
\documentclass{easychair}
\raggedbottom

\usepackage{doc}
\usepackage{xcolor}
\usepackage{mathpartir}
\usepackage{todonotes}
\presetkeys{todonotes}{inline}{}
\usepackage{scalerel}
\usepackage{bm}
\usepackage{amssymb}

\renewcommand{\mit}[1]{{\mathit{#1}}}
\newcommand{\msf}[1]{{\mathsf{#1}}}
\newcommand{\mbf}[1]{{\mathbf{#1}}}
\newcommand{\mbb}[1]{\mathbb{#1}}
\newcommand{\bs}[1]{\boldsymbol{#1}}
\newcommand{\wh}[1]{\widehat{#1}}
\newcommand{\mdo}{\mbf{do}\,}
\newcommand{\ind}{\hspace{1em}}
\newcommand{\bif}{\mbf{if}\,}
\newcommand{\bthen}{\mbf{then}\,}
\newcommand{\belse}{\mbf{else}\,}
\newcommand{\return}{\mbf{return}\,}
\newcommand{\pure}{\mbf{pure}\,}
\newcommand{\lam}{\lambda\,}
\newcommand{\data}{\mbf{data}\,}
\newcommand{\where}{\mbf{where}}
\newcommand{\M}{\msf{M}}
\newcommand{\letrec}{\mbf{letrec}\,}
\newcommand{\of}{\mbf{of}\,}
\newcommand{\go}{\mit{go}}
\newcommand{\add}{\mit{add}}
\newcommand{\letdef}{\mbf{let\,}}
\newcommand{\map}{\mit{map}}

\newcommand{\vas}{\mathsf{as}}
\newcommand{\vbs}{\mathsf{bs}}
\newcommand{\vcs}{\mathsf{cs}}
\newcommand{\vxs}{\mathsf{xs}}
\newcommand{\vys}{\mathsf{ys}}
\newcommand{\vsp}{\mathsf{sp}}
\newcommand{\vma}{\mathsf{ma}}
\newcommand{\vm}{\mathsf{m}}
\newcommand{\vn}{\mathsf{n}}
\newcommand{\vk}{\mathsf{k}}
\newcommand{\vA}{\mathsf{A}}
\newcommand{\vB}{\mathsf{B}}
\newcommand{\vC}{\mathsf{C}}
\newcommand{\vS}{\mathsf{S}}
\newcommand{\vF}{\mathsf{F}}
\newcommand{\vR}{\mathsf{R}}
\newcommand{\vM}{\mathsf{M}}
\newcommand{\vmb}{\mathsf{mb}}
\newcommand{\mAs}{\mathsf{As}}
\newcommand{\va}{\mathsf{a}}
\newcommand{\vb}{\mathsf{b}}
\newcommand{\vc}{\mathsf{c}}
\newcommand{\vd}{\mathsf{d}}
\newcommand{\vx}{\mathsf{x}}
\newcommand{\vy}{\mathsf{y}}
\newcommand{\vz}{\mathsf{z}}
\newcommand{\vf}{\mathsf{f}}
\newcommand{\vfs}{\mathsf{fs}}
\newcommand{\vg}{\mathsf{g}}
\newcommand{\vh}{\mathsf{h}}
\newcommand{\vt}{\mathsf{t}}
\newcommand{\vs}{\mathsf{s}}
\newcommand{\vr}{\mathsf{r}}
\newcommand{\vu}{\mathsf{u}}
\newcommand{\vl}{\mathsf{l}}
\newcommand{\vns}{\mathsf{ns}}

\newcommand{\SOP}{\msf{SOP}}
\newcommand{\El}{\msf{El}}
\newcommand{\USOP}{\msf{U}_{\msf{SOP}}}
\newcommand{\Uprod}{\msf{U_P}}
\newcommand{\Elprod}{\msf{El_{P}}}
\newcommand{\IsSOP}{\msf{IsSOP}}
\newcommand{\forEach}{\msf{forEach}}
\newcommand{\single}{\msf{single}}
\newcommand{\msplit}{\msf{split}}
\newcommand{\mapGen}{\msf{mapGen}}
\newcommand{\genPull}{\msf{gen_{Pull}}}
\newcommand{\Set}{\msf{Set}}
\newcommand{\casePull}{\msf{case_{Pull}}}
\newcommand{\appull}{\ap_{\Pull}}

\newcommand{\ext}{\triangleright}

\newcommand{\Int}{\msf{Int}}
\newcommand{\List}{\msf{List}}
\newcommand{\Tree}{\msf{Tree}}
\newcommand{\Node}{\msf{Node}}
\newcommand{\Leaf}{\msf{Leaf}}
\newcommand{\Nil}{\msf{Nil}}
\newcommand{\Cons}{\msf{Cons}}
\newcommand{\Reader}{\msf{Reader}}
\newcommand{\ReaderT}{\msf{ReaderT}}
\newcommand{\Monad}{\msf{Monad}}
\newcommand{\Applicative}{\msf{Applicative}}
\newcommand{\class}{\msf{class}}
\newcommand{\Functor}{\msf{Functor}}
\newcommand{\Bool}{\msf{Bool}}
\newcommand{\Statel}{\msf{State}}
\newcommand{\fro}{\leftarrow}
\newcommand{\case}{\mbf{case\,}}
\newcommand{\foldr}{\msf{foldr}}
\newcommand{\foldl}{\msf{foldl}}
\newcommand{\rep}{\msf{rep}}
\newcommand{\concatMap}{\msf{concatMap}}

\newcommand{\Lift}{{\Uparrow}}
\newcommand{\Up}{{\Uparrow}}
\newcommand{\spl}{{\bs{\sim}}}
\newcommand{\ql}{{\bs{\langle}}}
\newcommand{\qr}{{\bs{\rangle}}}
\newcommand{\bind}{\mathbin{>\!\!>\mkern-6.7mu=}}

\newcommand{\MTy}{\msf{MetaTy}}
\newcommand{\VTy}{\msf{ValTy}}
\newcommand{\Ty}{\msf{Ty}}
\newcommand{\CTy}{\msf{CompTy}}
\newcommand{\True}{\msf{True}}
\newcommand{\False}{\msf{False}}
\newcommand{\fst}{\msf{fst}}
\newcommand{\snd}{\msf{snd}}

\newcommand{\blank}{{\mathord{\hspace{1pt}\text{--}\hspace{1pt}}}}

\newcommand{\Nat}{\msf{Nat}}
\newcommand{\Zero}{\msf{Zero}}
\newcommand{\Suc}{\msf{Suc}}
\newcommand{\Maybe}{\msf{Maybe}}
\newcommand{\MaybeT}{\msf{MaybeT}}
\newcommand{\Nothing}{\msf{Nothing}}
\newcommand{\Just}{\msf{Just}}

\theoremstyle{remark}
\newtheorem{notation}{Notation}
\newtheorem*{axiom}{Axiom}

\newcommand{\id}{\mit{id}}
\newcommand{\mup}{\mbf{up}}
\newcommand{\mdown}{\mbf{down}}
\newcommand{\tyclass}{\mbf{class}}
\newcommand{\instance}{\mbf{instance}\,}
\newcommand{\Improve}{\msf{Improve}}
\newcommand{\Gen}{\msf{Gen}}
\newcommand{\unGen}{\mit{unGen}}
\renewcommand{\Vec}{\msf{Vec}}
\newcommand{\gen}{\mit{gen}}
\newcommand{\genRec}{\mit{genRec}}
\newcommand{\fmap}{<\!\!\$\!\!>}
\newcommand{\ap}
\newcommand{\runGen}{\mit{runGen}}
\newcommand{\qt}[1]{\ql#1\qr}
\newcommand{\lift}{\mit{lift}}
\newcommand{\liftGen}{\mit{liftGen}}
\newcommand{\MonadGen}{\msf{MonadGen}}
\newcommand{\MonadState}{\msf{MonadState}}
\newcommand{\MonadReader}{\msf{MonadReader}}
\newcommand{\RA}{\Rightarrow}
\newcommand{\EitherT}{\msf{EitherT}}
\newcommand{\Either}{\msf{Either}}
\newcommand{\Left}{\msf{Left}}
\newcommand{\Right}{\msf{Right}}
\newcommand{\StateT}{\msf{StateT}}
\newcommand{\Identity}{\msf{Identity}}

\newcommand{\Stop}{\msf{Stop}}
\newcommand{\Skip}{\msf{Skip}}
\newcommand{\Yield}{\msf{Yield}}

\newcommand{\runIdentity}{\mit{runIdentity}}
\newcommand{\runReaderT}{\mit{runReaderT}}
\newcommand{\newtype}{\mbf{newtype}\,}
\newcommand{\runMaybeT}{\mit{runMaybeT}}
\newcommand{\runStateT}{\mit{runStateT}}
\newcommand{\runState}{\mit{runState}}
\newcommand{\dlr}{\,\$\,}
\newcommand{\ImproveF}{\msf{ImproveF}}
\newcommand{\ExceptT}{\msf{ExceptT}}
\newcommand{\State}{\msf{State}}
\newcommand{\SumVS}{\msf{SumVS}}
\newcommand{\ProdCS}{\msf{ProdCS}}
\newcommand{\Here}{\msf{Here}}
\newcommand{\There}{\msf{There}}
\newcommand{\IsSumVS}{\msf{IsSumVS}}
\newcommand{\MonadJoin}{\msf{MonadJoin}}
\newcommand{\Stream}{\msf{Stream}}
\newcommand{\join}{\mit{join}}
\newcommand{\modify}{\mit{modify}}
\newcommand{\get}{\mit{get}}
\newcommand{\mput}{\mit{put}}
\newcommand{\Rep}{\mit{Rep}}
\newcommand{\encode}{\mit{encode}}
\newcommand{\decode}{\mit{decode}}
\newcommand{\mindex}{\mit{index}}
\newcommand{\mtabulate}{\mit{tabulate}}
\newcommand{\States}{\mit{States}}
\newcommand{\seed}{\mit{seed}}
\newcommand{\step}{\mit{step}}
\newcommand{\Step}{\msf{Step}}
\newcommand{\Pull}{\msf{Pull}}
\newcommand{\MkPull}{\msf{MkPull}}


\title{Polarized Lambda-Calculus at Runtime, Dependent Types at Compile Time}
\author{Andr\'as Kov\'acs}
\authorrunning{Kov\'acs}
\titlerunning{Polarized Lambda-Calculus at Runtime, Dependent Types at Compile Time}

\institute{
  University of Gothenburg
  Gothenburg, Sweden \\
  \email{andrask@chalmers.se}
}

\pagenumbering{gobble}
\begin{document}
\maketitle

We describe a particular two-level type theory \cite{staged2ltt}:
\begin{itemize}
\item The object level is a polarized simply-typed calculus, with
      computation types (functions, computational products) and value types
      (inductive types, closures).
\item The meta level is a standard dependent type theory.
\end{itemize}
This supports two-stage compilation, where we can use dependent types in the
surface language, but only object-level constructions remain after staging. It
appears to be an excellent setting for performance-focused staged programming.

The polarization lets us control closures and function arities in a fairly
lightweight way. In particular, if we do not use the explicit closure type
former, then all function calls can be compiled to calls and jumps to statically
known code locations. Hence, we migth ask: how much can we reproduce
from the abstraction tools of functional programming, without using any runtime
closures? Perhaps surprisingly, closures are rarely used in an essential way.
\begin{itemize}
\item A $\mit{map}$ function for lists is meant to be inlined, and after
      inlining no runtime closures remain.
\item Monadic binding in Haskell is a higher-order function. Without compiler optimizations,
      we get a big overhead from a deluge of runtime closures; but with optimizations
      we expect that no closures remain.
\end{itemize}
In the current work, we start to build up libraries in the mentioned two-level
theory, aiming to minimize the usage of runtime closures and eliminate
abstraction overheads. We want to shift work from general-purpose optimizing
compilers to deterministic \& extensible metaprogramming. For example, instead
of expecting GHC to sufficiently inline our monadic code, we implement efficient
monadic code generation ourselves.

\subsubsection*{Staged Monad Transformers}

Despite recent competition from a variety of algebraic effect systems, monads
and monad transformers remain the most widely used strongly-typed effect system,
so it make sense for us to develop their staged flavor.

In our approach, we \emph{don't use monads at all in the object
language}. Instead, we use meta-level monads, and only convert down to the
object level when runtime control dependencies force us to do so. Generally, the
staged version of a monad is obtained by extending it with \emph{code generation
as an effect}. The code generation monad is the following:
\begin{alignat*}{3}
  &\newtype \Gen : \MTy \to \MTy \\
  &\newtype \Gen\,A = \Gen\,(\{R : \msf{ObjTy}\} \to (A \to \Up R) \to \Up R)
\end{alignat*}
Here, $\msf{ObjTy}$ is the universe of object-level types, $\MTy$ is a
meta-level universe, and $\Up R$ is (intuitively) the type of metaprograms which
generate object expressions of type $R$.

This is a continuation monad where \emph{eventually} we have to return an
object-level value, so we can only ``run'' actions of type $\Gen\,(\Up A)$ to
extract a result with type $\Up A$. However, while working in the monad, we are
free to use both object-level and meta-level constructions, and introduce
object-level let-binders. In general, for some monad transformer stack $M$, we
obtain its staged version by putting a $\Gen$ at the bottom of the stack. For
example, $\State\,(\Up\msf{Int})$ becomes $\StateT\,(\Up\msf{Int})\,\Gen$. In
this monad in particular, we can modify an object expression in the state, and
also generate object code via $\Gen$.

Using our transformer library, we program in meta-monads, and insert conversions
to and from object code at the points of dynamic dependencies
(e.g.\ object-level function calls). Such conversions are defined in a
compositional manner, by recursion on transformer stacks. Overall, this style of
programming requires modest extra noise compared to Haskell, but it guarantees
high-quality code output by staging.

\subsubsection*{Staged Stream Fusion}

\emph{Stream fusion} \cite{DBLP:phd/ethos/Coutts11} is another application that we developed in this
setting. The idea here is to give a convenient list-like interface for
programming with meta-level state machines. Such machines can be turned into
efficient object-level code as blocks of mutually recursive functions, without
storing intermediate results in runtime lists. We demonstrate very concise
solutions to two well-known problems.

\textbf{First}, we need to generate mutually recursive blocks. One part of the
challenge is to have guaranteed well-scoping and well-typing, although the
number (and types) of definitions in a mutual block is only known at staging
time; see e.g.\ \cite{DBLP:conf/pepm/YallopK19}. We also want to avoid any runtime
overheads. We represent a mutual block as a single recursive definition with a
computational product type. The polarization guarantees that this can be
compiled without downstream overheads to an actual mutual block.

\textbf{Second}, fusion for arbitrary combinations of zipping and
$\mit{concatMap}$ has been a long-standing challenge. Currently, the
$\mit{strymonas}$ library \cite{DBLP:conf/pepm/0001K24} supports such fusion
but its solution is much more complex than ours, and it also relies on mutable
references in the object language. In contrast, we compile to pure object
code, which also enables us to parameterize streams over arbitrary monads
and embed monadic effects in stream definitions.

Our streams are tuples containing a type for an internal state, an initial
state and a representation of state transitions. Moreover, the internal state is
required to be a finite sum-of-products of object-level value types. Now, if
such sums-of-products are closed under $\Sigma$-types, $\concatMap$ is easily
definable by using a $\Sigma$-type of the family of internal states that we get
from an $A \to \msf{Stream}\,B$ function. We construct such $\Sigma$-types from
an internal \emph{generativity axiom}, which expresses that certain metaprograms
can't depend on object-level terms. This axiom does hold in the staging
semantics and we can erase it during staging. From this, we construct
\emph{transient} $\Sigma$-types which end up as non-dependent product types
after staging.

The ability to analyze object code has been often viewed as a desirable feature
in metaprogramming. In contrast, we demonstrate a use-case for the explicit
\emph{lack} of intensional analysis, through our generativity axiom. This might
be compared to \emph{internal parametricity} statements in type theories.

\bibliographystyle{plain}
\bibliography{references}

\end{document}
