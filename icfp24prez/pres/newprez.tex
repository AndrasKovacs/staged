%% build: latexmk -pdf -pvc -xelatex prez.tex

\documentclass[dvipsnames,aspectratio=169]{beamer}
\usetheme{Madrid}

%% kill footline
\setbeamertemplate{footline}[frame number]{}
\setbeamertemplate{navigation symbols}{}
\setbeamertemplate{footline}{}

%% bibliography
\bibliographystyle{alpha}
\setbeamerfont{bibliography item}{size=\footnotesize}
\setbeamerfont{bibliography entry author}{size=\footnotesize}
\setbeamerfont{bibliography entry title}{size=\footnotesize}
\setbeamerfont{bibliography entry location}{size=\footnotesize}
\setbeamerfont{bibliography entry note}{size=\footnotesize}
\setbeamertemplate{bibliography item}{}

%% kill ball enumeration
\setbeamertemplate{enumerate items}[circle]
\setbeamertemplate{section in toc}[circle]

%% kill block shadows
\setbeamertemplate{blocks}[rounded][shadow=false]
\setbeamertemplate{title page}[default][colsep=-4bp,rounded=true]

%% kill ball itemize
\setbeamertemplate{itemize items}[circle]

%% --------------------------------------------------------------------------------

\usepackage[utf8]{inputenc}
\usepackage{amsmath}
\usepackage{cite}
\usepackage{amsthm}
\usepackage{amssymb}
\usepackage{amsfonts}
\usepackage{mathpartir}
\usepackage{scalerel}
\usepackage{stmaryrd}
\usepackage{bm}
\usepackage{graphicx}
\usepackage{fontspec}
\usepackage{textcomp}
\usepackage{fancyvrb}
\usepackage{tikz-cd}

\tikzcdset{scale cd/.style={every label/.append style={scale=#1},
    cells={nodes={scale=#1}}}}

\setmonofont[Scale=0.8]{DejaVu Sans Mono Bold}

%% --------------------------------------------------------------------------------

%% HoTT style composition
\makeatletter
\DeclareRobustCommand{\sqcdot}{\mathbin{\mathpalette\morphic@sqcdot\relax}}
\newcommand{\morphic@sqcdot}[2]{%
  \sbox\z@{$\m@th#1\centerdot$}%
  \ht\z@=.33333\ht\z@
  \vcenter{\box\z@}%
}
\makeatother

\renewcommand{\mit}[1]{{\mathit{#1}}}
\newcommand{\ttt}[1]{{\texttt{#1}}}
\newcommand{\msf}[1]{{\mathsf{#1}}}
\newcommand{\mbf}[1]{{\mathbf{#1}}}
\newcommand{\mbb}[1]{\mathbb{#1}}
\newcommand{\U}{\mathsf{U}}
\newcommand{\code}{\mathsf{code}}
\newcommand{\bs}[1]{\boldsymbol{#1}}
\newcommand{\wh}[1]{\widehat{#1}}
\newcommand{\mdo}{\mbf{do}\,}
\newcommand{\ind}{\hspace{1em}}
\newcommand{\bif}{\mbf{if}\,}
\newcommand{\bthen}{\mbf{then}\,}
\newcommand{\belse}{\mbf{else}\,}
\newcommand{\return}{\mbf{return}\,}
\newcommand{\pure}{\mbf{pure}\,}
\newcommand{\lam}{\lambda\,}
\newcommand{\data}{\mbf{data}\,}
\newcommand{\where}{\mbf{where}}
\newcommand{\M}{\msf{M}}
\newcommand{\letrec}{\mbf{letrec}\,}
\newcommand{\of}{\mbf{of}\,}
\newcommand{\go}{\mit{go}}
\newcommand{\add}{\mit{add}}
\newcommand{\letdef}{\mbf{let\,}}
\newcommand{\map}{\mit{map}}
\newcommand{\emptycon}{\scaleobj{.75}\bullet}
\newcommand{\Tyo}{\msf{Ty}_{\mbbo}}
\newcommand{\Tmo}{\msf{Tm}_{\mbbo}}
\newcommand{\whset}{\wh{\Set}}
\newcommand{\ev}{\mbb{E}}
\newcommand{\re}{\mbb{R}}


\newcommand{\mbbc}{\mbb{C}}
\newcommand{\mbbo}{\mbb{O}}

\newcommand{\vas}{\mathsf{as}}
\newcommand{\vbs}{\mathsf{bs}}
\newcommand{\vcs}{\mathsf{cs}}
\newcommand{\vxs}{\mathsf{xs}}
\newcommand{\vys}{\mathsf{ys}}
\newcommand{\vsp}{\mathsf{sp}}
\newcommand{\vma}{\mathsf{ma}}
\newcommand{\vga}{\mathsf{ga}}
\newcommand{\vm}{\mathsf{m}}
\newcommand{\vn}{\mathsf{n}}
\newcommand{\vk}{\mathsf{k}}
\newcommand{\vA}{\mathsf{A}}
\newcommand{\vB}{\mathsf{B}}
\newcommand{\vC}{\mathsf{C}}
\newcommand{\vS}{\mathsf{S}}
\newcommand{\vF}{\mathsf{F}}
\newcommand{\vR}{\mathsf{R}}
\newcommand{\vM}{\mathsf{M}}
\newcommand{\vmb}{\mathsf{mb}}
\newcommand{\mAs}{\mathsf{As}}
\newcommand{\va}{\mathsf{a}}
\newcommand{\vb}{\mathsf{b}}
\newcommand{\vc}{\mathsf{c}}
\newcommand{\vd}{\mathsf{d}}
\newcommand{\vx}{\mathsf{x}}
\newcommand{\vy}{\mathsf{y}}
\newcommand{\vz}{\mathsf{z}}
\newcommand{\vf}{\mathsf{f}}
\newcommand{\vfs}{\mathsf{fs}}
\newcommand{\vg}{\mathsf{g}}
\newcommand{\vh}{\mathsf{h}}
\newcommand{\vt}{\mathsf{t}}
\newcommand{\vs}{\mathsf{s}}
\newcommand{\vr}{\mathsf{r}}
\newcommand{\vu}{\mathsf{u}}
\newcommand{\vl}{\mathsf{l}}
\newcommand{\vns}{\mathsf{ns}}
\newcommand{\vW}{\mathsf{W}}
\newcommand{\vsup}{\mathsf{sup}}
\newcommand{\vid}{\mathsf{id}}
\newcommand{\whW}{\wh{\vW}}


\newcommand{\SOP}{\msf{SOP}}
\newcommand{\El}{\msf{El}}
\newcommand{\USOP}{\msf{U}_{\msf{SOP}}}
\newcommand{\Uprod}{\msf{U_P}}
\newcommand{\Elprod}{\msf{El_{P}}}
\newcommand{\IsSOP}{\msf{IsSOP}}
\newcommand{\forEach}{\msf{forEach}}
\newcommand{\single}{\msf{single}}
\newcommand{\msplit}{\msf{split}}
\newcommand{\mapGen}{\msf{mapGen}}
\newcommand{\genPull}{\msf{gen_{Pull}}}
\newcommand{\Set}{\msf{Set}}
\newcommand{\casePull}{\msf{case_{Pull}}}
\newcommand{\appull}{\ap_{\Pull}}

\newcommand{\Con}{\msf{Con}}
\newcommand{\Sub}{\msf{Sub}}
\newcommand{\Tm}{\msf{Tm}}
\newcommand{\Mor}{\msf{Tm}}

\newcommand{\ext}{\triangleright}

\newcommand{\Int}{\msf{Int}}
\newcommand{\List}{\msf{List}}
\newcommand{\Tree}{\msf{Tree}}
\newcommand{\Node}{\msf{Node}}
\newcommand{\Leaf}{\msf{Leaf}}
\newcommand{\Nil}{\msf{Nil}}
\newcommand{\Cons}{\msf{Cons}}
\newcommand{\Reader}{\msf{Reader}}
\newcommand{\ReaderT}{\msf{ReaderT}}
\newcommand{\Monad}{\msf{Monad}}
\newcommand{\Applicative}{\msf{Applicative}}
\newcommand{\class}{\msf{class}}
\newcommand{\Functor}{\msf{Functor}}
\newcommand{\Bool}{\msf{Bool}}
\newcommand{\Statel}{\msf{State}}
\newcommand{\fro}{\leftarrow}
\newcommand{\case}{\mbf{case\,}}
\newcommand{\foldr}{\msf{foldr}}
\newcommand{\foldl}{\msf{foldl}}
\newcommand{\rep}{\msf{rep}}
\newcommand{\concatMap}{\msf{concatMap}}

\newcommand{\Lift}{{\Uparrow}}
\newcommand{\Up}{{\Uparrow}}
\newcommand{\spl}{{\bs{\sim}}}
\newcommand{\ql}{{\bs{\langle}}}
\newcommand{\qr}{{\bs{\rangle}}}
\newcommand{\bind}{\mathbin{>\!\!>\mkern-6.7mu=}}

\newcommand{\MTy}{\msf{MetaTy}}
\newcommand{\MTm}{\msf{MetaTm}}
\newcommand{\VTy}{\msf{ValTy}}
\newcommand{\Ty}{\msf{Ty}}
\newcommand{\CTy}{\msf{CompTy}}
\newcommand{\True}{\msf{True}}
\newcommand{\False}{\msf{False}}
\newcommand{\fst}{\msf{fst}}
\newcommand{\snd}{\msf{snd}}

\newcommand{\blank}{{\mathord{\hspace{1pt}\text{--}\hspace{1pt}}}}

\newcommand{\Nat}{\msf{Nat}}
\newcommand{\Zero}{\msf{Zero}}
\newcommand{\Suc}{\msf{Suc}}
\newcommand{\Maybe}{\msf{Maybe}}
\newcommand{\MaybeT}{\msf{MaybeT}}
\newcommand{\Nothing}{\msf{Nothing}}
\newcommand{\Just}{\msf{Just}}

\theoremstyle{remark}
\newtheorem{notation}{Notation}
\newtheorem*{axiom}{Axiom}

\newcommand{\id}{\mit{id}}
\newcommand{\mup}{\mbf{up}}
\newcommand{\mdown}{\mbf{down}}
\newcommand{\tyclass}{\mbf{class}}
\newcommand{\instance}{\mbf{instance}\,}
\newcommand{\Improve}{\msf{Improve}}
\newcommand{\Gen}{\msf{Gen}}
\newcommand{\unGen}{\mit{unGen}}
\renewcommand{\Vec}{\msf{Vec}}
\newcommand{\gen}{\mit{gen}}
\newcommand{\genRec}{\mit{genRec}}
\newcommand{\fmap}{<\!\!\$\!\!>}
\newcommand{\ap}
\newcommand{\runGen}{\mit{runGen}}
\newcommand{\qt}[1]{\ql#1\qr}
\newcommand{\lift}{\mit{lift}}
\newcommand{\liftGen}{\mit{liftGen}}
\newcommand{\MonadGen}{\msf{MonadGen}}
\newcommand{\MonadState}{\msf{MonadState}}
\newcommand{\MonadReader}{\msf{MonadReader}}
\newcommand{\RA}{\Rightarrow}
\newcommand{\EitherT}{\msf{EitherT}}
\newcommand{\Either}{\msf{Either}}
\newcommand{\Left}{\msf{Left}}
\newcommand{\Right}{\msf{Right}}
\newcommand{\StateT}{\msf{StateT}}
\newcommand{\Identity}{\msf{Identity}}

\newcommand{\Stop}{\msf{Stop}}
\newcommand{\Skip}{\msf{Skip}}
\newcommand{\Yield}{\msf{Yield}}

\newcommand{\runIdentity}{\mit{runIdentity}}
\newcommand{\runReaderT}{\mit{runReaderT}}
\newcommand{\newtype}{\mbf{newtype}\,}
\newcommand{\runMaybeT}{\mit{runMaybeT}}
\newcommand{\runStateT}{\mit{runStateT}}
\newcommand{\runState}{\mit{runState}}
\newcommand{\dlr}{\,\$\,}
\newcommand{\ImproveF}{\msf{ImproveF}}
\newcommand{\ExceptT}{\msf{ExceptT}}
\newcommand{\State}{\msf{State}}
\newcommand{\SumVS}{\msf{SumVS}}
\newcommand{\ProdCS}{\msf{ProdCS}}
\newcommand{\Here}{\msf{Here}}
\newcommand{\There}{\msf{There}}
\newcommand{\IsSumVS}{\msf{IsSumVS}}
\newcommand{\MonadJoin}{\msf{MonadJoin}}
\newcommand{\Stream}{\msf{Stream}}
\newcommand{\join}{\mit{join}}
\newcommand{\modify}{\mit{modify}}
\newcommand{\get}{\mit{get}}
\newcommand{\mput}{\mit{put}}
\newcommand{\Rep}{\mit{Rep}}
\newcommand{\encode}{\mit{encode}}
\newcommand{\decode}{\mit{decode}}
\newcommand{\mindex}{\mit{index}}
\newcommand{\mtabulate}{\mit{tabulate}}
\newcommand{\States}{\mit{States}}
\newcommand{\seed}{\mit{seed}}
\newcommand{\step}{\mit{step}}
\newcommand{\Step}{\msf{Step}}
\newcommand{\Pull}{\msf{Pull}}
\newcommand{\MkPull}{\msf{MkPull}}

\newcommand{\RED}[1]{{\color{BrickRed} #1}}
\newcommand{\GRN}[1]{{\color{OliveGreen} #1}}
\newcommand{\RBL}[1]{{\color{RoyalBlue} #1}}
\newcommand{\BLU}[1]{{\color{Blue} #1}}
\newcommand{\bsl}{\textbackslash}

%% --------------------------------------------------------------------------------

\title{Closure-Free Functional Programming in a Two-Level Type Theory}
\author{András Kovács}
\institute{University of Gothenburg}
\date{3 Sept 2024, ICFP 2024, Milan}
\begin{document}

\frame{\titlepage}

\begin{frame}[fragile]{Compiling monads in GHC Haskell}

\begin{columns}
\begin{column}{0.3\textwidth}
\textbf{Input:}
\begin{Verbatim}[commandchars=\\\{\}]

    \BLU{f} \RBL{::} \GRN{Reader Bool Int}
    \BLU{f} \RBL{=} \RED{do}
      b \RBL{<-} ask
      \RED{if} b \RED{then} return 10
           \RED{else} return 20






\end{Verbatim}
\end{column}
\pause
\begin{column}{0.5\textwidth}
\textbf{\texttt{-O0} output:}
\begin{Verbatim}[commandchars=\\\{\}]

    \BLU{dict1} :: \GRN{Monad} (\GRN{Reader Int})
    \BLU{dict1} = \GRN{MkMonad} ...

    \BLU{dict2} :: \GRN{MonadReader} (\GRN{Reader Int})
    \BLU{dict2} = \GRN{MkMonadReader} ...

    \BLU{f} :: \GRN{Reader Bool Int}
    \BLU{f} \RBL{=} (\RBL{>>=}) dict1 (ask dict2) (\RBL{\bsl}b \RBL{->}
      \RED{case} b \RED{of}
        \GRN{True } \RBL{->} return dict1 10
        \GRN{False} \RBL{->} return dict1 20)
\end{Verbatim}
\end{column}
\end{columns}

\end{frame}

%% f = >>=
%%       $dMonad_rHr
%%       (ask ($fMonadReaderrReaderT $fMonadIdentity))
%%       (\ b_aAg ->
%%          case b_aAg of {
%%            False -> return $dMonad_rHr (I# 20#);
%%            True -> return $dMonad_rHr (I# 10#)
%%          })


\begin{frame}[fragile]{Compiling monads in GHC Haskell}

\textbf{\texttt{-O1} output:}
\vspace{1em}
\begin{Verbatim}[commandchars=\\\{\}]
                             \BLU{f} \RBL{::} \GRN{Bool} \RBL{->} \GRN{Int}
                             \BLU{f} b = \RED{case} b \RED{of}
                               \GRN{True}  \RBL{->} 10
                               \GRN{False} \RBL{->} 20
\end{Verbatim}
\vspace{1em}
\emph{Optimization is hard!}
\vspace{1em}

Example: \ttt{mapM} is third-order \& rank-2 polymorphic, but almost all use cases should
compile to first-order monomorphic code.
\vspace{1em}
\begin{Verbatim}[commandchars=\\\{\}]
    \BLU{mapM} \RBL{::} \GRN{Monad} m \RBL{=>} (a \RBL{->} m b) \RBL{->} [a] \RBL{->} m [b]
\end{Verbatim}
\vspace{1em}

\end{frame}

\begin{frame}[fragile]{Compiling monads in GHC Haskell}

\[\begin{tikzcd}[scale cd=1.1, sep=normal]
	Source &&&&& Core &&&&& {...}
	\arrow["{Elaboration}"', from=1-1, to=1-6]
	\arrow["\begin{array}{c} Aggressive\\general\!-\!purpose\,optimization \end{array}", from=1-6, to=1-6, loop, in=55, out=125, distance=10mm]
	\arrow["Compilation"', from=1-6, to=1-11]
\end{tikzcd}\]

\end{frame}

\begin{frame}[fragile]{Proposal}

\[\begin{tikzcd}[scale cd=1.1, sep=normal]
	Source && Metaprograms &&&& Core &&& {...}
	\arrow["Elaboration"', from=1-1, to=1-3]
	\arrow["{Compile\!-\!time\,execution}"', from=1-3, to=1-7]
	\arrow["{Conservative\,optimization}", from=1-7, to=1-7, loop, in=55, out=125, distance=10mm]
	\arrow["Compilation"', from=1-7, to=1-10]
\end{tikzcd}\]
%% \begin{itemize}
%%   \item Programmers and the elaborator need to work
%%   \item Metaprograms are fast and robust

%% \end{itemize}


\end{frame}


\begin{frame}[fragile]{Proposal}

\begin{columns}
\begin{column}{0.3\textwidth}
\textbf{Input in WIP language:}
\begin{Verbatim}[commandchars=\\\{\}]
    \BLU{f} : \GRN{Reader Bool Int}
    \BLU{f} \RBL{:=} \RED{do}
      b \RBL{<-} ask
      \RED{if} b \RED{then} return 10
           \RED{else} return 20
\end{Verbatim}

\end{column}
\begin{column}{0.6\textwidth}
\begin{itemize}
\item Looks similar to Haskell.
\item Desugaring \& elaboration does slightly more work.
\item Compiles to efficient code \emph{with a formal guarantee, without
      general-purpose optimization}.
\end{itemize}
\end{column}
\end{columns}
\vspace{1em}
\pause

\begin{block}{}
\textbf{Setup}
\begin{itemize}
\item We use a \emph{two-level type theory (2LTT)}:
  \begin{itemize}
    \item Metalanguage (compile time): dependently typed, fancy features.
    \item Object language (runtime): simpler \& lower-level.
    \item The two are smoothly integrated.
  \end{itemize}
\item Monadic programs are \emph{metaprograms} which generate efficient
  runtime code.
\item Most optimizations are implemented in libraries instead of compiler internals.
\end{itemize}
\end{block}

\end{frame}

\begin{frame}[fragile]{Closure-freedom}

How do we know that an abstraction feature is ``low-cost''?
\vspace{1em}

It's a good indicator if the generated code is \textbf{free of closures}.
\vspace{1em}
\begin{itemize}
  \item Functional abstractions are usually implemented by closures:
        higher-order functions, classes, ML functors.
  \item A big part of GHC's \texttt{-O1} work is to get rid of closures.
\end{itemize}






\end{frame}

%% \begin{frame}[fragile]{The 2LTT}

%% Two type universes for the two levels.
%% \vspace{1em}
%% \begin{enumerate}
%% \item \textbf{\ttt{MetaTy}}: universe (hierarchy) of meta-level types. Supports
%%       $\Pi$, $\Sigma$, inductive families.
%% \item \textbf{\ttt{Ty}}: universe of object-level types.
%%   \begin{itemize}
%%   \item \ttt{Ty} is itself an element of \ttt{MetaTy}.
%%   \item No polymorphism or type dependency in \ttt{Ty}.
%%   \item Two sub-universes:
%%   \begin{itemize}
%%   \item \ttt{CompTy} contains \emph{computation types}: functions, computational products.
%%   \item \ttt{ValTy}  contains \emph{value types}: ADT-s and closure types.
%%   \item ADT constructors only store values, functions only take value inputs.
%%   \end{itemize}
%%   \end{itemize}
%% \end{enumerate}
%% \vspace{0.5em}

%% \end{frame}

\begin{frame}[fragile]{The 2LTT}

\begin{itemize}
  \item \textbf{\ttt{MetaTy}}: universe of meta-level types. Supports $\Pi$, $\Sigma$, inductive families.
  \item \textbf{\ttt{Ty}}: universe of object-level types. Only simple types. Polarized to \emph{computation} \& \emph{value} types.
\end{itemize}
\vspace{2em}

\begin{columns}
\begin{column}{0.4\textwidth}
\textbf{A meta-level program}:
\begin{verbatim}
  id : {A : MetaTy} → A → A
  id x = x




\end{verbatim}
\end{column}

\begin{column}{0.6\textwidth}
\textbf{An object-level program:}
\begin{verbatim}
  data List (A : ValTy) := Nil | Cons A List

  myMap : List Int → List Int
  myMap ns := case xs of
    Nil       → Nil
    Cons n ns → Cons (n + 10) (myMap ns)
\end{verbatim}
\end{column}
\end{columns}


\end{frame}

\begin{frame}[fragile]{The 2LTT - interaction between stages}

\begin{itemize}
\item \textbf{Lifting}: for \ttt{A : Ty}, we have \ttt{⇑A : MetaTy}, as the type of
      metaprograms that produce \ttt{A}-typed object programs.
\item \textbf{Quoting}: for \ttt{t : A} and \ttt{A : Ty}, we have \ttt{<t>} as the metaprogram
      which immediately returns \ttt{t}.
\item \textbf{Splicing}: for \ttt{t : ⇑A}, we have \ttt{\char`~t : A} which runs the
       metaprogram \ttt{t} and inserts its output in some object-level code.
\item Definitional equalities: \ttt{\char`~<t> ≡ t} and \ttt{<\char`~t> ≡ t}.
\end{itemize}

\end{frame}

\begin{frame}[fragile]{Staged example}

\begin{verbatim}
    map : {A B : ValTy} → (⇑A → ⇑B) → ⇑(List A) → ⇑(List B)
    map f as = <letrec go as := case as of
                  Nil       → Nil
                  Cons a as → Cons ~(f <a>) (go as)
                in go ~as>

    myMap : List Int → List Int
    myMap ns := ~(map (λ x. <~x + 10>) <ns>)
\end{verbatim}

\end{frame}

\begin{frame}[fragile]{Staged example - with stage inference}

\begin{verbatim}
    map : {A B : ValTy} → (A → B) → List A → List B
    map f = letrec go as := case as of
                  Nil       → Nil
                  Cons a as → Cons (f a) (go as)
            in go

    myMap : List Int → List Int
    myMap := map (λ x. x + 10)
\end{verbatim}
\end{frame}

\begin{frame}[fragile]{A monad for code generation}

Type classes (and monads) only exist in the metalanguage.
\begin{verbatim}
    class Monad (m : MetaTy → MetaTy) where
      return : a → m a
      (>>=)  : m a → (a → m b) → m b
\end{verbatim}
\pause
\ttt{Gen} is a Monad whose effect is \textbf{generating object code}:
\begin{verbatim}
    newtype Gen A = Gen {unGen : {R : Ty} → (A → ⇑R) → ⇑R}
    instance Monad Gen where ...

    runGen : Gen (⇑A) → ⇑A
    runGen (Gen f) = f id
\end{verbatim}
\pause
Generating an object-level \ttt{let}-definition:
\begin{verbatim}
    gen : {A : Ty} → ⇑A → Gen ⇑A
    gen {A} a = Gen $ λ k. <let x : A := ~a in ~(k <x>)>
\end{verbatim}

\end{frame}

\begin{frame}[fragile]{A monad for code generation}

\begin{columns}
\begin{column}{0.4\textwidth}
\textbf{Staged input:}
\begin{verbatim}
    myAction : ⇑Int → Gen ⇑Int
    myAction x = do
      y ← gen <~x + ~x>
      z ← gen <~y * ~y>
      return <~y * ~z>

    foo : Int
    foo := ~(runGen $ myAction <10>)
\end{verbatim}
\end{column}
\begin{column}{0.5\textwidth}
\textbf{Output:}
\begin{verbatim}
    foo : Int
    foo := let y := 10 + 10 in
           let z := y * y in
           y * z




\end{verbatim}
\end{column}
\end{columns}

\end{frame}

\begin{frame}[fragile]{Staging monads}

We only program in meta-level monads, but also have back-and-forth
translations between object-level types and metamonads.

\begin{verbatim}
    down : ReaderT (⇑R) Gen (⇑A) → ⇑(ReaderTₒ R Identityₒ A)
    up   : ⇑(ReaderTₒ R Identityₒ A) → ReaderT (⇑R) Gen (⇑A)

    f : ReaderTₒ Bool Identityₒ Int
    f := ~(down $ do
      b  ← ask
      b' ← split b
      case b' of
        MetaTrue  → return <10>
        MetaFalse → return <20>)
\end{verbatim}

\begin{block}{}
\textbf{In general:} up/down is defined by recursion on a transformer stack.
\alert{\ttt{Identityₒ} is related to \ttt{Gen}.}
\end{block}
\end{frame}

\begin{frame}[fragile]{Case splitting on object values}

\begin{verbatim}
                   split : MonadGen m => ⇑Bool → m MetaBool
                   split b = liftGen $ Gen $ λ k. <case ~b of
                     True  → ~(k MetaTrue)
                     False → ~(k MetaFalse)>

                   f : ReaderTₒ Bool Identityₒ Int
                   f := ~(down $ do
                     b  ← ask
                     b' ← split b
                     case b' of
                       MetaTrue  → return <10>
                       MetaFalse → return <20>)
\end{verbatim}

\end{frame}

\begin{frame}[fragile]{Polarization \& Closure-Freedom}

\emph{Computation} and \emph{value} types are tracked
in the object language.
\begin{verbatim}
    _→_     : ValTy → Ty → CompTy
    Closure : CompTy → ValTy
    List    : ValTy → ValTy
    ...
\end{verbatim}

Closures only appear at runtime if we use \texttt{Closure}!
\vspace{1em}

We have to use \ttt{Closure (A → B)} to store functions in ADTs
or pass them as function arguments.
\vspace{1em}

(It's rare that closures are \emph{really needed} in programming!)

\end{frame}

\begin{frame}[fragile]{Polarization \& Closure-Freedom}

How to compile this?
\vspace{1em}
\begin{verbatim}
                   f : Bool → Int → Int
                   f b = case b of True  → λ x. x + 10
                                   False → λ x. x * 10
\end{verbatim}
And this?
\begin{verbatim}
                   f : Int → Int
                   f x :=
                     let g y := x + y;
                     g x + 10
\end{verbatim}
\end{frame}

\begin{frame}{More things}

\begin{itemize}
  \item Conditionally accepted at ICFP 24: \emph{Closure-Free Functional Programming in a Two-Level Type Theory}.
  \item More things in paper: join points, stream fusion, semantics, more about polarized types.
  \item Implementations:
    \begin{itemize}
      \item In Agda and typed Template Haskell with some limitations.
      \item Standalone implementation early WIP.
    \end{itemize}
\end{itemize}
\vspace{2em}

\begin{center}

  \Large{\textbf{Thank you!}}

\end{center}

\end{frame}




\end{document}
