%% build: latexmk -pdf -pvc -xelatex prez.tex

\documentclass[dvipsnames,aspectratio=169]{beamer}
\usetheme{Madrid}

%% kill footline
\setbeamertemplate{footline}[frame number]{}
\setbeamertemplate{navigation symbols}{}
\setbeamertemplate{footline}{}

%% bibliography
\bibliographystyle{alpha}
\setbeamerfont{bibliography item}{size=\footnotesize}
\setbeamerfont{bibliography entry author}{size=\footnotesize}
\setbeamerfont{bibliography entry title}{size=\footnotesize}
\setbeamerfont{bibliography entry location}{size=\footnotesize}
\setbeamerfont{bibliography entry note}{size=\footnotesize}
\setbeamertemplate{bibliography item}{}

%% kill ball enumeration
\setbeamertemplate{enumerate items}[circle]
\setbeamertemplate{section in toc}[circle]

%% kill block shadows
\setbeamertemplate{blocks}[rounded][shadow=false]
\setbeamertemplate{title page}[default][colsep=-4bp,rounded=true]

%% kill ball itemize
\setbeamertemplate{itemize items}[circle]

%% --------------------------------------------------------------------------------

\usepackage[utf8]{inputenc}
\usepackage{amsmath}
\usepackage{cite}
\usepackage{amsthm}
\usepackage{amssymb}
\usepackage{amsfonts}
\usepackage{mathpartir}
\usepackage{scalerel}
\usepackage{stmaryrd}
\usepackage{bm}
\usepackage{graphicx}
\usepackage{fontspec}
\usepackage{textcomp}
\usepackage{fancyvrb}
\usepackage{tikz-cd}

\tikzcdset{scale cd/.style={every label/.append style={scale=#1},
    cells={nodes={scale=#1}}}}

\setmonofont[Scale=0.8]{DejaVu Sans Mono Bold}

%% --------------------------------------------------------------------------------

%% HoTT style composition
\makeatletter
\DeclareRobustCommand{\sqcdot}{\mathbin{\mathpalette\morphic@sqcdot\relax}}
\newcommand{\morphic@sqcdot}[2]{%
  \sbox\z@{$\m@th#1\centerdot$}%
  \ht\z@=.33333\ht\z@
  \vcenter{\box\z@}%
}
\makeatother

\renewcommand{\mit}[1]{{\mathit{#1}}}
\newcommand{\ttt}[1]{{\texttt{#1}}}
\newcommand{\msf}[1]{{\mathsf{#1}}}
\newcommand{\mbf}[1]{{\mathbf{#1}}}
\newcommand{\mbb}[1]{\mathbb{#1}}
\newcommand{\U}{\mathsf{U}}
\newcommand{\code}{\mathsf{code}}
\newcommand{\bs}[1]{\boldsymbol{#1}}
\newcommand{\wh}[1]{\widehat{#1}}
\newcommand{\mdo}{\mbf{do}\,}
\newcommand{\ind}{\hspace{1em}}
\newcommand{\bif}{\mbf{if}\,}
\newcommand{\bthen}{\mbf{then}\,}
\newcommand{\belse}{\mbf{else}\,}
\newcommand{\return}{\mbf{return}\,}
\newcommand{\pure}{\mbf{pure}\,}
\newcommand{\lam}{\lambda\,}
\newcommand{\data}{\mbf{data}\,}
\newcommand{\where}{\mbf{where}}
\newcommand{\M}{\msf{M}}
\newcommand{\letrec}{\mbf{letrec}\,}
\newcommand{\of}{\mbf{of}\,}
\newcommand{\go}{\mit{go}}
\newcommand{\add}{\mit{add}}
\newcommand{\letdef}{\mbf{let\,}}
\newcommand{\map}{\mit{map}}
\newcommand{\emptycon}{\scaleobj{.75}\bullet}
\newcommand{\Tyo}{\msf{Ty}_{\mbbo}}
\newcommand{\Tmo}{\msf{Tm}_{\mbbo}}
\newcommand{\whset}{\wh{\Set}}
\newcommand{\ev}{\mbb{E}}
\newcommand{\re}{\mbb{R}}


\newcommand{\mbbc}{\mbb{C}}
\newcommand{\mbbo}{\mbb{O}}

\newcommand{\vas}{\mathsf{as}}
\newcommand{\vbs}{\mathsf{bs}}
\newcommand{\vcs}{\mathsf{cs}}
\newcommand{\vxs}{\mathsf{xs}}
\newcommand{\vys}{\mathsf{ys}}
\newcommand{\vsp}{\mathsf{sp}}
\newcommand{\vma}{\mathsf{ma}}
\newcommand{\vga}{\mathsf{ga}}
\newcommand{\vm}{\mathsf{m}}
\newcommand{\vn}{\mathsf{n}}
\newcommand{\vk}{\mathsf{k}}
\newcommand{\vA}{\mathsf{A}}
\newcommand{\vB}{\mathsf{B}}
\newcommand{\vC}{\mathsf{C}}
\newcommand{\vS}{\mathsf{S}}
\newcommand{\vF}{\mathsf{F}}
\newcommand{\vR}{\mathsf{R}}
\newcommand{\vM}{\mathsf{M}}
\newcommand{\vmb}{\mathsf{mb}}
\newcommand{\mAs}{\mathsf{As}}
\newcommand{\va}{\mathsf{a}}
\newcommand{\vb}{\mathsf{b}}
\newcommand{\vc}{\mathsf{c}}
\newcommand{\vd}{\mathsf{d}}
\newcommand{\vx}{\mathsf{x}}
\newcommand{\vy}{\mathsf{y}}
\newcommand{\vz}{\mathsf{z}}
\newcommand{\vf}{\mathsf{f}}
\newcommand{\vfs}{\mathsf{fs}}
\newcommand{\vg}{\mathsf{g}}
\newcommand{\vh}{\mathsf{h}}
\newcommand{\vt}{\mathsf{t}}
\newcommand{\vs}{\mathsf{s}}
\newcommand{\vr}{\mathsf{r}}
\newcommand{\vu}{\mathsf{u}}
\newcommand{\vl}{\mathsf{l}}
\newcommand{\vns}{\mathsf{ns}}
\newcommand{\vW}{\mathsf{W}}
\newcommand{\vsup}{\mathsf{sup}}
\newcommand{\vid}{\mathsf{id}}
\newcommand{\whW}{\wh{\vW}}


\newcommand{\SOP}{\msf{SOP}}
\newcommand{\El}{\msf{El}}
\newcommand{\USOP}{\msf{U}_{\msf{SOP}}}
\newcommand{\Uprod}{\msf{U_P}}
\newcommand{\Elprod}{\msf{El_{P}}}
\newcommand{\IsSOP}{\msf{IsSOP}}
\newcommand{\forEach}{\msf{forEach}}
\newcommand{\single}{\msf{single}}
\newcommand{\msplit}{\msf{split}}
\newcommand{\mapGen}{\msf{mapGen}}
\newcommand{\genPull}{\msf{gen_{Pull}}}
\newcommand{\Set}{\msf{Set}}
\newcommand{\casePull}{\msf{case_{Pull}}}
\newcommand{\appull}{\ap_{\Pull}}

\newcommand{\Con}{\msf{Con}}
\newcommand{\Sub}{\msf{Sub}}
\newcommand{\Tm}{\msf{Tm}}
\newcommand{\Mor}{\msf{Tm}}

\newcommand{\ext}{\triangleright}

\newcommand{\Int}{\msf{Int}}
\newcommand{\List}{\msf{List}}
\newcommand{\Tree}{\msf{Tree}}
\newcommand{\Node}{\msf{Node}}
\newcommand{\Leaf}{\msf{Leaf}}
\newcommand{\Nil}{\msf{Nil}}
\newcommand{\Cons}{\msf{Cons}}
\newcommand{\Reader}{\msf{Reader}}
\newcommand{\ReaderT}{\msf{ReaderT}}
\newcommand{\Monad}{\msf{Monad}}
\newcommand{\Applicative}{\msf{Applicative}}
\newcommand{\class}{\msf{class}}
\newcommand{\Functor}{\msf{Functor}}
\newcommand{\Bool}{\msf{Bool}}
\newcommand{\Statel}{\msf{State}}
\newcommand{\fro}{\leftarrow}
\newcommand{\case}{\mbf{case\,}}
\newcommand{\foldr}{\msf{foldr}}
\newcommand{\foldl}{\msf{foldl}}
\newcommand{\rep}{\msf{rep}}
\newcommand{\concatMap}{\msf{concatMap}}

\newcommand{\Lift}{{\Uparrow}}
\newcommand{\Up}{{\Uparrow}}
\newcommand{\spl}{{\bs{\sim}}}
\newcommand{\ql}{{\bs{\langle}}}
\newcommand{\qr}{{\bs{\rangle}}}
\newcommand{\bind}{\mathbin{>\!\!>\mkern-6.7mu=}}

\newcommand{\MTy}{\msf{MetaTy}}
\newcommand{\MTm}{\msf{MetaTm}}
\newcommand{\VTy}{\msf{ValTy}}
\newcommand{\Ty}{\msf{Ty}}
\newcommand{\CTy}{\msf{CompTy}}
\newcommand{\True}{\msf{True}}
\newcommand{\False}{\msf{False}}
\newcommand{\fst}{\msf{fst}}
\newcommand{\snd}{\msf{snd}}

\newcommand{\blank}{{\mathord{\hspace{1pt}\text{--}\hspace{1pt}}}}

\newcommand{\Nat}{\msf{Nat}}
\newcommand{\Zero}{\msf{Zero}}
\newcommand{\Suc}{\msf{Suc}}
\newcommand{\Maybe}{\msf{Maybe}}
\newcommand{\MaybeT}{\msf{MaybeT}}
\newcommand{\Nothing}{\msf{Nothing}}
\newcommand{\Just}{\msf{Just}}

\theoremstyle{remark}
\newtheorem{notation}{Notation}
\newtheorem*{axiom}{Axiom}

\newcommand{\id}{\mit{id}}
\newcommand{\mup}{\mbf{up}}
\newcommand{\mdown}{\mbf{down}}
\newcommand{\tyclass}{\mbf{class}}
\newcommand{\instance}{\mbf{instance}\,}
\newcommand{\Improve}{\msf{Improve}}
\newcommand{\Gen}{\msf{Gen}}
\newcommand{\unGen}{\mit{unGen}}
\renewcommand{\Vec}{\msf{Vec}}
\newcommand{\gen}{\mit{gen}}
\newcommand{\genRec}{\mit{genRec}}
\newcommand{\fmap}{<\!\!\$\!\!>}
\newcommand{\ap}
\newcommand{\runGen}{\mit{runGen}}
\newcommand{\qt}[1]{\ql#1\qr}
\newcommand{\lift}{\mit{lift}}
\newcommand{\liftGen}{\mit{liftGen}}
\newcommand{\MonadGen}{\msf{MonadGen}}
\newcommand{\MonadState}{\msf{MonadState}}
\newcommand{\MonadReader}{\msf{MonadReader}}
\newcommand{\RA}{\Rightarrow}
\newcommand{\EitherT}{\msf{EitherT}}
\newcommand{\Either}{\msf{Either}}
\newcommand{\Left}{\msf{Left}}
\newcommand{\Right}{\msf{Right}}
\newcommand{\StateT}{\msf{StateT}}
\newcommand{\Identity}{\msf{Identity}}

\newcommand{\Stop}{\msf{Stop}}
\newcommand{\Skip}{\msf{Skip}}
\newcommand{\Yield}{\msf{Yield}}

\newcommand{\runIdentity}{\mit{runIdentity}}
\newcommand{\runReaderT}{\mit{runReaderT}}
\newcommand{\newtype}{\mbf{newtype}\,}
\newcommand{\runMaybeT}{\mit{runMaybeT}}
\newcommand{\runStateT}{\mit{runStateT}}
\newcommand{\runState}{\mit{runState}}
\newcommand{\dlr}{\,\$\,}
\newcommand{\ImproveF}{\msf{ImproveF}}
\newcommand{\ExceptT}{\msf{ExceptT}}
\newcommand{\State}{\msf{State}}
\newcommand{\SumVS}{\msf{SumVS}}
\newcommand{\ProdCS}{\msf{ProdCS}}
\newcommand{\Here}{\msf{Here}}
\newcommand{\There}{\msf{There}}
\newcommand{\IsSumVS}{\msf{IsSumVS}}
\newcommand{\MonadJoin}{\msf{MonadJoin}}
\newcommand{\Stream}{\msf{Stream}}
\newcommand{\join}{\mit{join}}
\newcommand{\modify}{\mit{modify}}
\newcommand{\get}{\mit{get}}
\newcommand{\mput}{\mit{put}}
\newcommand{\Rep}{\mit{Rep}}
\newcommand{\encode}{\mit{encode}}
\newcommand{\decode}{\mit{decode}}
\newcommand{\mindex}{\mit{index}}
\newcommand{\mtabulate}{\mit{tabulate}}
\newcommand{\States}{\mit{States}}
\newcommand{\seed}{\mit{seed}}
\newcommand{\step}{\mit{step}}
\newcommand{\Step}{\msf{Step}}
\newcommand{\Pull}{\msf{Pull}}
\newcommand{\MkPull}{\msf{MkPull}}

\newcommand{\RED}[1]{{\color{BrickRed} #1}}
\newcommand{\GRN}[1]{{\color{OliveGreen} #1}}
\newcommand{\RBL}[1]{{\color{RoyalBlue} #1}}
\newcommand{\BLU}[1]{{\color{Blue} #1}}
\newcommand{\bsl}{\textbackslash}
\newcommand{\brl}{\textbraceleft}
\newcommand{\brr}{\textbraceright}

%% --------------------------------------------------------------------------------

\title{Closure-Free Functional Programming in a Two-Level Type Theory}
\author{András Kovács}
\institute{University of Gothenburg}
\date{3rd Sept 2024, ICFP 2024, Milan}
\begin{document}

\frame{\titlepage}

\begin{frame}[fragile]{Compiling monads in GHC Haskell}

\begin{columns}
\begin{column}{0.3\textwidth}
\textbf{Source:}
\begin{Verbatim}[commandchars=\\\{\}]

    \BLU{f} \RBL{::} \GRN{Reader Bool Int}
    \BLU{f} \RBL{=} \RED{do}
      b \RBL{<-} ask
      \RED{if} b \RED{then} return 10
           \RED{else} return 20






\end{Verbatim}
\end{column}
\pause
\begin{column}{0.5\textwidth}
\textbf{\texttt{-O0} Core output:}
\begin{Verbatim}[commandchars=\\\{\}]

    \BLU{dict1} :: \GRN{Monad} (\GRN{Reader Int})
    \BLU{dict1} = \GRN{MkMonad} ...

    \BLU{dict2} :: \GRN{MonadReader} (\GRN{Reader Int})
    \BLU{dict2} = \GRN{MkMonadReader} ...

    \BLU{f} :: \GRN{Reader Bool Int}
    \BLU{f} \RBL{=} (\RBL{>>=}) dict1 (ask dict2) (\RBL{\bsl}b \RBL{->}
      \RED{case} b \RED{of}
        \GRN{True } \RBL{->} return dict1 10
        \GRN{False} \RBL{->} return dict1 20)
\end{Verbatim}
\end{column}
\end{columns}

\end{frame}

%% f = >>=
%%       $dMonad_rHr
%%       (ask ($fMonadReaderrReaderT $fMonadIdentity))
%%       (\ b_aAg ->
%%          case b_aAg of {
%%            False -> return $dMonad_rHr (I# 20#);
%%            True -> return $dMonad_rHr (I# 10#)
%%          })


\begin{frame}[fragile]{Compiling monads in GHC Haskell}

\textbf{\texttt{-O1} output:}
\vspace{1em}
\begin{Verbatim}[commandchars=\\\{\}]
                             \BLU{f} \RBL{::} \GRN{Bool} \RBL{->} \GRN{Int}
                             \BLU{f} b = \RED{case} b \RED{of}
                               \GRN{True}  \RBL{->} 10
                               \GRN{False} \RBL{->} 20
\end{Verbatim}
\vspace{1em}
\emph{Optimization is hard!}
\vspace{1em}

Example: \ttt{\BLU{mapM}} is third-order \& rank-2 polymorphic, but almost all use cases should
compile to first-order monomorphic code.
\vspace{1em}
\begin{Verbatim}[commandchars=\\\{\}]
    \BLU{mapM} \RBL{::} \GRN{Monad} m \RBL{=>} (a \RBL{->} m b) \RBL{->} [a] \RBL{->} m [b]
\end{Verbatim}
\vspace{1em}

\end{frame}

\begin{frame}[fragile]{Compiling monads in GHC Haskell}

\[\begin{tikzcd}[scale cd=1.1, sep=normal]
	Source &&&&& Core &&&&& {...}
	\arrow["{Elaboration}"', from=1-1, to=1-6]
	\arrow["\begin{array}{c} Aggressive\\general\!-\!purpose\,optimization \end{array}", from=1-6, to=1-6, loop, in=55, out=125, distance=10mm]
	\arrow["Compilation"', from=1-6, to=1-11]
\end{tikzcd}\]

\end{frame}

\begin{frame}[fragile]{Proposal}

\[\begin{tikzcd}[scale cd=1.1, sep=normal]
	Source && Metaprograms &&&& Core &&& {...}
	\arrow["Elaboration"', from=1-1, to=1-3]
	\arrow["{Compile\!-\!time\,execution}"', from=1-3, to=1-7]
	\arrow["{Conservative\,optimization}", from=1-7, to=1-7, loop, in=55, out=125, distance=10mm]
	\arrow["Compilation"', from=1-7, to=1-10]
\end{tikzcd}\]
\vspace{2em}

The paper explores \textbf{monad transformers} and stream fusion using this design.


\end{frame}


\begin{frame}[fragile]{Proposal}

\begin{columns}
\begin{column}{0.3\textwidth}
\textbf{Source in WIP language:}
\begin{Verbatim}[commandchars=\\\{\}]
    \BLU{f} : \GRN{Reader Bool Int}
    \BLU{f} \RBL{:=} \RED{do}
      b \RBL{<-} ask
      \RED{if} b \RED{then} return 10
           \RED{else} return 20
\end{Verbatim}

\end{column}
\begin{column}{0.6\textwidth}
\begin{itemize}
\item Looks similar to Haskell.
\item Desugaring \& elaboration does slightly more work.
\item Compilation to efficient code is formally guaranteed.
\end{itemize}
\end{column}
\end{columns}
\vspace{1em}
\pause

\begin{block}{}
\textbf{Setup}
\begin{itemize}
\item \emph{Two-level type theory (2LTT)}:
  \begin{itemize}
    \item Metalanguage (compile time): dependently typed.
    \item Object language (runtime): simply typed, \emph{polarized}.
    \item The two are smoothly integrated.
  \end{itemize}
\end{itemize}
\end{block}

\end{frame}

\begin{frame}[fragile]{The 2LTT}

\begin{itemize}
  \item \textbf{\ttt{\GRN{MetaTy}}}: universe of meta-level types.
  \item \textbf{\ttt{\GRN{Ty}}}: universe of object-level types. \alert{Polarization}:
    \textbf{\ttt{\GRN{ValTy}}} and \textbf{\ttt{\GRN{CompTy}}} are sub-universes of \textbf{\ttt{\GRN{Ty}}}.
\end{itemize}
\vspace{1em}

\begin{columns}
\begin{column}{0.4\textwidth}
\textbf{A meta-level program}:
\begin{Verbatim}[commandchars=\\\{\}]
  \BLU{id} \RBL{:} {\textbraceleft}A \RBL{:} \GRN{MetaTy}{\textbraceright} \RBL{->} A \RBL{->} A
  \BLU{id} x \RBL{=} x




\end{Verbatim}
\end{column}

\begin{column}{0.6\textwidth}
\textbf{An object-level program:}
\begin{Verbatim}[commandchars=\\\{\}]
  \RED{data} \GRN{List} (A \RBL{:} \GRN{ValTy}) \RBL{:=} \GRN{Nil} | \GRN{Cons} A (\GRN{List} A)

  \BLU{myMap} \RBL{:} \GRN{List Int} \RBL{->} \GRN{List Int}
  \BLU{myMap} ns \RBL{:=} \RED{case} xs \RED{of}
    \GRN{Nil}       \RBL{->} \GRN{Nil}
    \GRN{Cons} n ns \RBL{->} \GRN{Cons} (n + 10) (myMap ns)
\end{Verbatim}
\end{column}
\end{columns}
\vspace{2em}
\emph{Polarization lets us control closure creation through types.}
\end{frame}

%% \begin{frame}[fragile]{Closure-freedom}

%% In the object language:
%% \begin{itemize}
%%   \item Closures are \textbf{values}.
%%   \item Statically known functions are \textbf{computations}.
%%   \item If we don't ask for closures, we don't get them!
%% \end{itemize}
%% \vspace{1em}

%% Essential usage of closures is surprisingly rare!

%% \vspace{1em}
%% Closure-freedom is a good indicator of low-cost abstraction.


%% \end{frame}

\begin{frame}[fragile]{The 2LTT - interaction between stages}

\begin{itemize}
\item \textbf{Lifting}: for \ttt{A \RBL{:} \GRN{Ty}}, we have \ttt{\RED{⇑}A \RBL{:} \GRN{MetaTy}}, as the type of
      metaprograms that produce \ttt{A}-typed object programs.
\item \textbf{Quoting}: for \ttt{t \RBL{:} A} and \ttt{A \RBL{:} \GRN{Ty}}, we have \ttt{\RED{<}t\RED{>} : \RED{⇑}A} as the metaprogram
      which immediately returns \ttt{t}.
\item \textbf{Splicing}: for \ttt{t : \RED{⇑}A}, we have \ttt{\RED{\char`~}t \RBL{:} A} which runs the
       metaprogram \ttt{t} and inserts its output in some object-level code.
\item Definitional equalities: \ttt{\RED{\char`~<}t\RED{>} ≡ t} and \ttt{\RED{<\char`~}t\RED{>} ≡ t}.
\end{itemize}

\end{frame}

\begin{frame}[fragile]{Staged example}

\begin{Verbatim}[commandchars=\\\{\}]
    \BLU{map} \RBL{:} {\textbraceleft}A B \RBL{:} \GRN{ValTy}{\textbraceright} \RBL{->} (\RED{⇑}A \RBL{->} \RED{⇑}B) \RBL{->} \RED{⇑}(\GRN{List} A) \RBL{->} \RED{⇑}(\GRN{List} B)
    \BLU{map} f as \RBL{=} \RED{<letrec} go as \RBL{:=} \RED{case} as \RED{of}
                  \GRN{Nil}       \RBL{->} \GRN{Nil}
                  \GRN{Cons} a as \RBL{->} \GRN{Cons} \RED{~}(f \RED{<}a\RED{>}) (go as)
                \RED{in} go \RED{~}as\RED{>}

    \BLU{myMap} \RBL{:} \GRN{List Int} \RBL{->} \GRN{List Int}
    \BLU{myMap} ns \RBL{:=} \RED{~}(map (\RBL{\bsl}x\RBL{.} \RED{<~}x \RBL{+} 10\RED{>}) \RED{<}ns\RED{>})
\end{Verbatim}

\end{frame}

\begin{frame}[fragile]{Staged example - with stage inference}

\begin{Verbatim}[commandchars=\\\{\}]
    \BLU{map} \RBL{:} {\textbraceleft}A B \RBL{:} \GRN{ValTy}{\textbraceright} \RBL{->} (A \RBL{->} B) \RBL{->} \GRN{List} A \RBL{->} \GRN{List} B
    \BLU{map} f \RBL{=} \RED{letrec} go as \RBL{:=} \RED{case} as \RED{of}
               \GRN{Nil}       \RBL{->} \GRN{Nil}
               \GRN{Cons} a as \RBL{->} \GRN{Cons} (f a) (go as)
            \RED{in} go

    \BLU{myMap} \RBL{:} \GRN{List Int} \RBL{->} \GRN{List Int}
    \BLU{myMap} \RBL{:=} map (\RBL{\bsl}x\RBL{.} x \RBL{+} 10)
\end{Verbatim}
\end{frame}

\begin{frame}[fragile]{A monad for code generation}

Type classes (and monads) only exist in the metalanguage.
\begin{Verbatim}[commandchars=\\\{\}]
    \RED{class} \GRN{Monad} (M \RBL{:} \GRN{MetaTy} \RBL{->} \GRN{MetaTy}) \RED{where}
      return \RBL{:} A \RBL{->} M A
      (\RBL{>>=})  \RBL{:} M A \RBL{->} (A \RBL{->} M B) \RBL{->} M B
\end{Verbatim}
\pause
\vspace{0.5em}
\ttt{\GRN{Gen}} is a Monad whose effect is \textbf{generating object code}:
\vspace{0.5em}
\begin{Verbatim}[commandchars=\\\{\}]
    \RED{newtype} \GRN{Gen} A \RBL{=} \GRN{Gen} {\brl}unGen \RBL{:} {\brl}R \RBL{:} \GRN{Ty}\brr \RBL{->} (A \RBL{->} \RED{⇑}R) \RBL{->} \RED{⇑}R\brr
    \RED{instance} \GRN{Monad} \GRN{Gen} \RED{where} ...

    \BLU{runGen} \RBL{:} \GRN{Gen} (\RED{⇑}A) \RBL{->} \RED{⇑}A
    \BLU{runGen} (\GRN{Gen} f) \RBL{=} f id
\end{Verbatim}
\pause
\vspace{0.3em}
Generating an object-level \ttt{let}-definition:
\begin{Verbatim}[commandchars=\\\{\}]
    \BLU{gen} \RBL{:} {\brl}A \RBL{:} \GRN{Ty}\brr \RBL{->} \RED{⇑}A \RBL{->} \GRN{Gen} (\RED{⇑}A)
    \BLU{gen} {\brl}A\brr a \RBL{=} \GRN{Gen} \RBL{$} \RBL{{\bsl}}k\RBL{.} \RED{<let} x \RBL{:} A \RBL{:=} \RED{~}a \RED{in} \RED{~}(k \RED{<}x\RED{>})\RED{>}
\end{Verbatim}

\end{frame}

\begin{frame}[fragile]{A monad for code generation}

\begin{columns}
\begin{column}{0.4\textwidth}
\textbf{Metaprogram:}
\begin{Verbatim}[commandchars=\\\{\}]
    \BLU{foo} \RBL{:} \GRN{Int}
    \BLU{foo} \RBL{:=} \RED{~}(runGen \RBL{$} \RED{do}
      x \RBL{<-} gen \RED{<}10 \RBL{+} 10\RED{>}
      y \RBL{<-} gen \RED{<~}x \RBL{*} \RED{~}x\RED{>}
      return \RED{<~}x \RBL{*} \RED{~}y\RED{>})
\end{Verbatim}
\end{column}

\begin{column}{0.5\textwidth}
\textbf{$\hspace{1em}$Code output:}
\begin{Verbatim}[commandchars=\\\{\}]
     \BLU{foo} \RBL{:} \GRN{Int}
     \BLU{foo} \RBL{:=} \RED{let} x \RBL{:=} 10 \RBL{+} 10 \RED{in}
            \RED{let} y \RBL{:=} x \RBL{*} x \RED{in}
            x \RBL{*} y

\end{Verbatim}
\end{column}
\end{columns}

\end{frame}

\begin{frame}[fragile]{Generating monadic code}

We want to define efficient code generation for a monad \texttt{M}.
{\center
\begin{block}{}
\center\emph{\texttt{M} extended with \texttt{\GRN{Gen}} at the bottom yields
the corresponding code generator monad.}
\end{block}}

\vspace{1em}

For example:
\begin{itemize}
\item \texttt{\GRN{ReaderT} (\RED{⇑}R) \GRN{Gen} (\RED{⇑}A)} actions
      are code generators for \texttt{\GRN{Reader} R A}.
\item \texttt{\GRN{StateT} (\RED{⇑}S) \GRN{Gen} (\RED{⇑}A)} actions
      are code generators for \texttt{\GRN{State} S A}.
\end{itemize}
\vspace{0.5em}

\pause
In each case, we can convert back-and forth, e.g.\
\vspace{0.5em}
\begin{Verbatim}[commandchars=\\\{\}]
    \BLU{up}   : \RED{⇑}(\GRN{Reader} R A) \RBL{->} \GRN{ReaderT} (\RED{⇑}R) \GRN{Gen} (\RED{⇑}A)
    \BLU{down} : \GRN{ReaderT} (\RED{⇑}R) \GRN{Gen} (\RED{⇑}A) \RBL{->} \RED{⇑}(\GRN{Reader} R A)
\end{Verbatim}

\end{frame}

\begin{frame}[fragile]{Generating monadic code}

\begin{columns}
\begin{column}{0.4\textwidth}
\textbf{Metaprogram:}
\begin{Verbatim}[commandchars=\\\{\}]
    \BLU{action} \RBL{:} \GRN{Reader} \GRN{Int} \GRN{Int}
    \BLU{action} \RBL{:=} \RED{~}(down \RBL{$} \RED{do}
      x \RBL{<-} ask
      x \RBL{<-} ask
      return \RED{<~}x \RBL{+} \RED{~}x\RED{>})
\end{Verbatim}
\end{column}

\begin{column}{0.5\textwidth}
\textbf{$\hspace{1em}$Output:}
\begin{Verbatim}[commandchars=\\\{\}]
    \BLU{action} \RBL{:} \GRN{Reader} \GRN{Int} \GRN{Int}
    \BLU{action} \RBL{:=} \GRN{Reader} (\RBL{\bsl}x\RBL{.} x \RBL{+} x)



\end{Verbatim}
\end{column}
\end{columns}
\vspace{2em}

\emph{We get ``fusion'' for monadic code.}

\end{frame}

\begin{frame}[fragile]{Pattern matching on object-level values}

How do we inspect the structure of object-level values at compile time?
\vspace{1em}

We don't directly support looking inside object expressions (it breaks some things).
\vspace{1em}

But we can generate object-level pattern matches in \texttt{\GRN{Gen}}:
\vspace{0.5em}
\begin{Verbatim}[commandchars=\\\{\}]
    \BLU{split} \RBL{:} \RED{⇑}\GRN{Bool} \RBL{->} \GRN{Gen} \GRN{MetaBool}
    \BLU{split} b \RBL{=} \GRN{Gen} \RBL{$} \RBL{\bsl}k\RBL{.} \RED{<case ~}b \RED{of}
      \GRN{True } \RBL{->} \RED{~}(k \GRN{MetaTrue})
      \GRN{False} \RBL{->} \RED{~}(k \GRN{MetaFalse})\RED{>}
\end{Verbatim}

\vspace{1em}
\texttt{\BLU{split}} generalizes to all object ADT-s and all \texttt{\GRN{Gen}}-based monads.

\end{frame}

\begin{frame}[fragile]{Compiling monads - example}

\begin{Verbatim}[commandchars=\\\{\}]
      \BLU{f} \RBL{:} \GRN{Reader Bool Int}            \BLU{f} \RBL{:} \GRN{Reader Bool Int}
      \BLU{f} \RBL{:=} \RED{do}                        \BLU{f} \RBL{:=} \RED{~}(down \RBL{$} \RED{do}
        b \RBL{<-} ask              ==>      b \RBL{<-} ask
        \RED{if} b \RED{then} return 10            split b \RBL{>>=} \RBL{\bsl}\RED{case}
             \RED{else} return 20               \GRN{MetaTrue}  \RBL{->} return \RED{<}10\RED{>}
                                          \GRN{MetaFalse} \RBL{->} return \RED{<}20\RED{>})


                        \BLU{f} \RBL{:} \GRN{Reader Bool Int}
                  ==>   \BLU{f} \RBL{:=} \GRN{Reader} (\RBL{\bsl}b\RBL{.} \RED{case} b \RED{of}
                           \GRN{True}  \RBL{->} 10
                           \GRN{False} \RBL{->} 20)
\end{Verbatim}


\end{frame}




\begin{frame}{More things}

More in the paper and artifact:
\begin{itemize}
  \item Handling join points in monads.
  \item Handling mutually recursive blocks.
  \item Stream fusion.
  \item More metatheory.
  \item Adaptation as Agda and Typed Template Haskell libraries.
\end{itemize}
\vspace{1em}

Work in progress:
\begin{itemize}
\item Standalone prototype targeting LLVM.
\item Deploying the Template Haskell library in the Agda source code,
      in high-performance generics.
\end{itemize}

\pause
\begin{center}

  \Large{\textbf{Thank you!}}

\end{center}
\end{frame}

\end{document}
