
%% build: latexmk -pdf -pvc paper.tex

\documentclass[acmsmall,screen,review,anonymous]{acmart}
%% \documentclass[nonacm,acmsmall,review]{acmart}
%% \raggedbottom

%% \BibTeX command to typeset BibTeX logo in the docs
\AtBeginDocument{%
  \providecommand\BibTeX{{%
    \normalfont B\kern-0.5em{\scshape i\kern-0.25em b}\kern-0.8em\TeX}}}


%% \setcopyright{rightsretained}
%% \acmPrice{}
%% \acmDOI{10.1145/3547641}
%% \acmYear{2022}
%% \copyrightyear{2022}
%% \acmSubmissionID{icfp22main-p52-p}
%% \acmJournal{PACMPL}
%% \acmVolume{6}
%% \acmNumber{ICFP}
%% \acmArticle{110}
%% \acmMonth{8}


%% These commands are for a JOURNAL article.

%% \acmJournal{JACM}
%% \acmVolume{37}
%% \acmNumber{4}
%% \acmArticle{111}
%% \acmMonth{8}

%%
%% Submission ID.
%% Use this when submitting an article to a sponsored event. You'll
%% receive a unique submission ID from the organizers
%% of the event, and this ID should be used as the parameter to this command.
%%\acmSubmissionID{123-A56-BU3}

\bibliographystyle{ACM-Reference-Format}
%%
%% The majority of ACM publications use numbered citations and
%% references.  The command \citestyle{authoryear} switches to the
%% "author year" style.
%%
%% If you are preparing content for an event
%% sponsored by ACM SIGGRAPH, you must use the "author year" style of
%% citations and references.
%% Uncommenting
%% the next command will enable that style.
\citestyle{acmauthoryear}

%% --------------------------------------------------------------------------------

\usepackage{xcolor}
\usepackage{mathpartir}
\usepackage{todonotes}
\presetkeys{todonotes}{inline}{}
\usepackage{scalerel}
\usepackage{bm}
\usepackage{mathtools}
\usepackage{stmaryrd}
\usepackage[title]{appendix}

\newcommand{\mit}[1]{{\mathsf{#1}}}
\newcommand{\msf}[1]{{\mathsf{#1}}}
\newcommand{\mbf}[1]{{\mathbf{#1}}}
\newcommand{\mbb}[1]{\mathbb{#1}}
\newcommand{\rel}{^{\approx}}
\newcommand{\U}{\mathsf{U}}
\newcommand{\p}{\mathsf{p}}
\newcommand{\q}{\mathsf{q}}
\newcommand{\code}{\mathsf{code}}
\newcommand{\bs}[1]{\boldsymbol{#1}}
\newcommand{\wh}[1]{\widehat{#1}}
\newcommand{\mdo}{\mbf{do}\,}
\newcommand{\ind}{\hspace{1em}}
\newcommand{\bif}{\mbf{if}\,}
\newcommand{\bthen}{\mbf{then}\,}
\newcommand{\belse}{\mbf{else}\,}
\newcommand{\return}{\mbf{return}\,}
\newcommand{\pure}{\mbf{pure}\,}
\newcommand{\lam}{\lambda\,}
\newcommand{\data}{\mbf{data}\,}
\newcommand{\where}{\mbf{where}}
\newcommand{\M}{\msf{M}}
\newcommand{\letrec}{\mbf{letrec}\,}
\newcommand{\of}{\mbf{of}\,}
\newcommand{\go}{\mit{go}}
\newcommand{\add}{\mit{add}}
\newcommand{\letdef}{\mbf{let\,}}
\newcommand{\map}{\mit{map}}
\newcommand{\emptycon}{\scaleobj{.75}\bullet}
\newcommand{\Tyo}{\msf{Ty}_{\mbbo}}
\newcommand{\Tmo}{\msf{Tm}_{\mbbo}}
\newcommand{\whset}{\wh{\Set}}
\newcommand{\ev}{\mbb{E}}
\newcommand{\re}{\mbb{R}}
\newcommand{\welim}{\vW{-}\msf{elim}}

\newcommand{\mbbc}{\mbb{C}}
\newcommand{\mbbo}{\mbb{O}}

\newcommand{\vas}{\mathsf{as}}
\newcommand{\vbs}{\mathsf{bs}}
\newcommand{\vcs}{\mathsf{cs}}
\newcommand{\vxs}{\mathsf{xs}}
\newcommand{\vys}{\mathsf{ys}}
\newcommand{\vsp}{\mathsf{sp}}
\newcommand{\vma}{\mathsf{ma}}
\newcommand{\vga}{\mathsf{ga}}
\newcommand{\vm}{\mathsf{m}}
\newcommand{\vn}{\mathsf{n}}
\newcommand{\vk}{\mathsf{k}}
\newcommand{\vA}{\mathsf{A}}
\newcommand{\vB}{\mathsf{B}}
\newcommand{\vC}{\mathsf{C}}
\newcommand{\vS}{\mathsf{S}}
\newcommand{\vF}{\mathsf{F}}
\newcommand{\vR}{\mathsf{R}}
\newcommand{\vM}{\mathsf{M}}
\newcommand{\vmb}{\mathsf{mb}}
\newcommand{\mAs}{\mathsf{As}}
\newcommand{\va}{\mathsf{a}}
\newcommand{\vb}{\mathsf{b}}
\newcommand{\vc}{\mathsf{c}}
\newcommand{\vd}{\mathsf{d}}
\newcommand{\vx}{\mathsf{x}}
\newcommand{\vy}{\mathsf{y}}
\newcommand{\vz}{\mathsf{z}}
\newcommand{\vf}{\mathsf{f}}
\newcommand{\vfs}{\mathsf{fs}}
\newcommand{\vg}{\mathsf{g}}
\newcommand{\vh}{\mathsf{h}}
\newcommand{\vt}{\mathsf{t}}
\newcommand{\vs}{\mathsf{s}}
\newcommand{\vr}{\mathsf{r}}
\newcommand{\vu}{\mathsf{u}}
\newcommand{\vl}{\mathsf{l}}
\newcommand{\vns}{\mathsf{ns}}
\newcommand{\vW}{\mathsf{W}}
\newcommand{\vsup}{\mathsf{sup}}
\newcommand{\vid}{\mathsf{id}}
\newcommand{\whW}{\wh{\vW}}


\newcommand{\SOP}{\msf{SOP}}
\newcommand{\El}{\msf{El}}
\newcommand{\USOP}{\msf{U}_{\msf{SOP}}}
\newcommand{\Uprod}{\msf{U_P}}
\newcommand{\Elprod}{\msf{El_{P}}}
\newcommand{\IsSOP}{\msf{IsSOP}}
\newcommand{\forEach}{\msf{forEach}}
\newcommand{\single}{\msf{single}}
\newcommand{\msplit}{\msf{split}}
\newcommand{\mapGen}{\msf{mapGen}}
\newcommand{\genPull}{\msf{gen_{Pull}}}
\newcommand{\Set}{\msf{Set}}
\newcommand{\casePull}{\msf{case_{Pull}}}
\newcommand{\appull}{\ap_{\Pull}}

\newcommand{\Con}{\msf{Con}}
\newcommand{\Sub}{\msf{Sub}}
\newcommand{\Tm}{\msf{Tm}}

\newcommand{\ext}{\triangleright}

\newcommand{\Int}{\msf{Int}}
\newcommand{\List}{\msf{List}}
\newcommand{\Tree}{\msf{Tree}}
\newcommand{\Node}{\msf{Node}}
\newcommand{\Leaf}{\msf{Leaf}}
\newcommand{\Nil}{\msf{Nil}}
\newcommand{\Cons}{\msf{Cons}}
\newcommand{\Reader}{\msf{Reader}}
\newcommand{\ReaderT}{\msf{ReaderT}}
\newcommand{\Monad}{\msf{Monad}}
\newcommand{\Applicative}{\msf{Applicative}}
\newcommand{\class}{\msf{class}}
\newcommand{\Functor}{\msf{Functor}}
\newcommand{\Bool}{\msf{Bool}}
\newcommand{\Statel}{\msf{State}}
\newcommand{\fro}{\leftarrow}
\newcommand{\case}{\mbf{case\,}}
\newcommand{\foldr}{\msf{foldr}}
\newcommand{\foldl}{\msf{foldl}}
\newcommand{\rep}{\msf{rep}}
\newcommand{\concatMap}{\msf{concatMap}}

\newcommand{\Lift}{{\Uparrow}}
\newcommand{\Up}{{\Uparrow}}
\newcommand{\spl}{{\bs{\sim}}}
\newcommand{\ql}{{\bs{\langle}}}
\newcommand{\qr}{{\bs{\rangle}}}
\newcommand{\bind}{\mathbin{>\!\!>\mkern-6.7mu=}}

\newcommand{\MTy}{\msf{MetaTy}}
\newcommand{\MTm}{\msf{MetaTm}}
\newcommand{\VTy}{\msf{ValTy}}
\newcommand{\Ty}{\msf{Ty}}
\newcommand{\CTy}{\msf{CompTy}}
\newcommand{\True}{\msf{True}}
\newcommand{\False}{\msf{False}}
\newcommand{\fst}{\msf{fst}}
\newcommand{\snd}{\msf{snd}}

\newcommand{\blank}{{\mathord{\hspace{1pt}\text{--}\hspace{1pt}}}}

\newcommand{\Nat}{\msf{Nat}}
\newcommand{\Zero}{\msf{Zero}}
\newcommand{\Suc}{\msf{Suc}}
\newcommand{\Maybe}{\msf{Maybe}}
\newcommand{\MaybeT}{\msf{MaybeT}}
\newcommand{\Nothing}{\msf{Nothing}}
\newcommand{\Just}{\msf{Just}}

\theoremstyle{remark}
\newtheorem{notation}{Notation}
\newtheorem*{axiom}{Axiom}

\newcommand{\id}{\mit{id}}
\newcommand{\mup}{\mbf{up}}
\newcommand{\mdown}{\mbf{down}}
\newcommand{\tyclass}{\mbf{class}}
\newcommand{\instance}{\mbf{instance}\,}
\newcommand{\Improve}{\msf{Improve}}
\newcommand{\Gen}{\msf{Gen}}
\newcommand{\unGen}{\mit{unGen}}
\renewcommand{\Vec}{\msf{Vec}}
\newcommand{\gen}{\mit{gen}}
\newcommand{\genRec}{\mit{genRec}}
\newcommand{\fmap}{<\!\!\$\!\!>}
\newcommand{\ap}
\newcommand{\runGen}{\mit{runGen}}
\newcommand{\qt}[1]{\ql#1\qr}
\newcommand{\lift}{\mit{lift}}
\newcommand{\liftGen}{\mit{liftGen}}
\newcommand{\MonadGen}{\msf{MonadGen}}
\newcommand{\MonadState}{\msf{MonadState}}
\newcommand{\MonadReader}{\msf{MonadReader}}
\newcommand{\RA}{\Rightarrow}
\newcommand{\EitherT}{\msf{EitherT}}
\newcommand{\Either}{\msf{Either}}
\newcommand{\Left}{\msf{Left}}
\newcommand{\Right}{\msf{Right}}
\newcommand{\StateT}{\msf{StateT}}
\newcommand{\Identity}{\msf{Identity}}

\newcommand{\Stop}{\msf{Stop}}
\newcommand{\Skip}{\msf{Skip}}
\newcommand{\Yield}{\msf{Yield}}

\newcommand{\runIdentity}{\mit{runIdentity}}
\newcommand{\runReaderT}{\mit{runReaderT}}
\newcommand{\newtype}{\mbf{newtype}\,}
\newcommand{\runMaybeT}{\mit{runMaybeT}}
\newcommand{\runStateT}{\mit{runStateT}}
\newcommand{\runState}{\mit{runState}}
\newcommand{\dlr}{\,\$\,}
\newcommand{\ImproveF}{\msf{ImproveF}}
\newcommand{\ExceptT}{\msf{ExceptT}}
\newcommand{\State}{\msf{State}}
\newcommand{\SumVS}{\msf{SumVS}}
\newcommand{\ProdCS}{\msf{ProdCS}}
\newcommand{\Here}{\msf{Here}}
\newcommand{\There}{\msf{There}}
\newcommand{\IsSumVS}{\msf{IsSumVS}}
\newcommand{\MonadJoin}{\msf{MonadJoin}}
\newcommand{\Stream}{\msf{Stream}}
\newcommand{\join}{\mit{join}}
\newcommand{\modify}{\mit{modify}}
\newcommand{\get}{\mit{get}}
\newcommand{\mput}{\mit{put}}
\newcommand{\Rep}{\mit{Rep}}
\newcommand{\encode}{\mit{encode}}
\newcommand{\decode}{\mit{decode}}
\newcommand{\mindex}{\mit{index}}
\newcommand{\mtabulate}{\mit{tabulate}}
\newcommand{\States}{\mit{States}}
\newcommand{\seed}{\mit{seed}}
\newcommand{\step}{\mit{step}}
\newcommand{\Step}{\msf{Step}}
\newcommand{\Pull}{\msf{Pull}}
\newcommand{\MkPull}{\msf{MkPull}}

%% --------------------------------------------------------------------------------

%%
%% end of the preamble, start of the body of the document source.
%% \hypersetup{draft}
\begin{document}

\title{Closure-Free Functional Programming in a Two-Level Type Theory}
%% \titlenote{}

%% The "author" command and its associated commands are used to define
%% the authors and their affiliations.
%% Of note is the shared affiliation of the first two authors, and the
%% "authornote" and "authornotemark" commands
%% used to denote shared contribution to the research.
\author{András Kovács}
\email{andrask@chalmers.se}
\orcid{0000-0002-6375-9781}
\affiliation{%
  \institution{University of Gothenburg}
  \country{Sweden}
  \city{Gothenburg}
}

\begin{abstract}
Many abstraction tools in functional programming rely heavily on general-purpose
compiler optimization to achieve adequate performance. For example, monadic
binding is a higher-order function which yields runtime closures in the absence
of sufficient compile-time inlining and beta-reductions, thereby significantly
degrading performance. In current systems such as the Glasgow Haskell Compiler,
there is no strong guarantee that general-purpose optimization can eliminate
abstraction overheads, and users only have indirect and fragile control over
code generation through inlining directives and compiler options. We propose a
two-stage language to simultaneously get strong guarantees about code generation
and strong abstraction features. The object language is a simply-typed
first-order language which can be compiled without runtime closures. The
compile-time language is a dependent type theory. The two are integrated in a
two-level type theory.

We demonstrate two applications of the system. First, we develop monads and
monad transformers. Here, abstraction overheads are eliminated by staging and we
can reuse almost all definitions from the existing Haskell ecosystem. Second,
we develop pull-based stream fusion. Here we make essential use of dependent
types to give a concise definition of a $\mathsf{concatMap}$ operation with
guaranteed fusion. We provide an Agda implementation and a typed Template
Haskell implementation of these developments.

\end{abstract}

\begin{CCSXML}
<ccs2012>
   <concept>
       <concept_id>10003752.10003790.10011740</concept_id>
       <concept_desc>Theory of computation~Type theory</concept_desc>
       <concept_significance>500</concept_significance>
       </concept>
   <concept>
       <concept_id>10011007.10011006.10011041.10011047</concept_id>
       <concept_desc>Software and its engineering~Source code generation</concept_desc>
       <concept_significance>500</concept_significance>
       </concept>
 </ccs2012>
\end{CCSXML}

\ccsdesc[500]{Theory of computation~Type theory}
\ccsdesc[500]{Software and its engineering~Source code generation}

\keywords{two-level type theory, staged compilation}

\maketitle


\section{Introduction}\label{sec:introduction}

Modern functional programming supports many convenient abstractions. These often
come with significant runtime overheads. Sometimes the overheads are acceptable,
but in other cases compiler optimization is crucial. Monads in Haskell is an
example for the latter. Even the $\Reader$ monad, which is one of the simplest
in terms of implementation, yields large overheads when compiled without
optimizations. Consider the following:
\begin{alignat*}{3}
  &\vf :: \Int \to \Reader\,\Bool\,\Int \\
  &\vf\,\vx = \mdo\{\vb \fro \msf{ask}; \bif \vx\, \bthen \return (\vx + 10)\, \belse \return (\vx + 20)\}
\end{alignat*}
With optimizations enabled, GHC compiles this roughly to the code below:
\begin{alignat*}{3}
  &\vf :: \Int \to \Bool \to \Int \\
  &\vf = \lam \vx\,\vb.\, \bif \vb\, \bthen \vx + 10\, \belse \vx + 20
\end{alignat*}
Without optimizations we roughly get:
\begin{alignat*}{3}
  &\vf = \lam \vx.\,(\bind)\,\msf{MonadReaderDict}\,(\msf{ask}\,\msf{MonadReaderDict})\,(\lam \vb.\,\bif \vb\\
  &\ind \bthen\return\,\msf{MonadReaderDict}\,(\vx + 10)\\
  &\ind \belse\hspace{0.25em}\return\,\msf{MonadReaderDict}\,(\vx + 20))
\end{alignat*}
Here, $\msf{MonadReaderDict}$ is a runtime dictionary, containing the methods of
the $\Monad$ instance for $\Reader$, and $(\bind)\,\msf{MonadReaderDict}$ is the
projection of the binding function. Here, a runtime closure will be created for
the $\lam \vb.\,...$ function, and $(\bind)$, $\msf{ask}$ and $\mbf{return}$
will create additional dynamic closures.

The difference between optimized and unoptimized code is already large here, and
it gets even larger when we consider monad transformers or code that is
polymorphic over monads. In Haskell, such code is pervasive, even in fairly
basic programs which do not use fancy abstractions. Consider the $\msf{mapM}$
function from the Haskell Prelude:
\begin{alignat*}{3}
  & \msf{mapM} :: \Monad\,\vm \Rightarrow (\va \to \vm\,\vb) \to [\va] \to \vm\,[\vb]
\end{alignat*}
This is a third-order and second-rank polymorphic function in disguise, because
its monad dictionary argument contains the polymorphic second-order method
$(\bind)$.  Compiling $\msf{mapM}$ efficiently relies on inlining the instance
dictionary, then inlining the methods contained there, and also inlining the
functions that the higher-order binding is applied to.

GHC's optimization efforts are respectable, and it has gotten quite adept over
its long history of development. However, there is no strong guarantee that
certain optimizations will happen. Control over optimizations remains tricky,
fragile and non-compositional. \texttt{INLINE} and \texttt{REWRITE} pragmas can
be used to control code generation \cite{ghcdocs}, but without strong
guarantees, and their advanced usage requires knowledge of GHC internals. For
example, correctly specifying the \emph{ordering} of certain rule applications
is often needed. We also have to care about formal function arities. Infamously,
the function composition operator is defined as $(.)\,\vf\,\vg = \lam \vx \to
\vf\,(\vg\,\vx)$ in the base libraries, instead of as $(.)\,\vf\,\vg\,\vx =
\vf\,(\vg\,\vx)$, to get better inlining behavior \cite{ghcbase}. It is common
practice in high-performance Haskell programming to visually review GHC's
optimized code output.

\subsection{Closure-Free Staged Compilation}

In this paper we use staged compilation to address issues of robustness. The
idea is to shift as much as possible work from general-purpose optimization to
metaprograms.

Metaprograms can be deterministic, transparent, and can be run efficiently,
using fast interpreters or machine code compilation. In contrast,
general-purpose optimizers are slower to run, less transparent and less
robust. Also, metaprogramming allows library authors to exploit
domain-specific optimizations, while it is not realistic for general-purpose
optimizers to know about all domains.

On the other hand, metaprogramming requires some additional care and input from
programmers. Historically, there have been problems with ergonomics as well. In
weakly-typed staged systems, code generation might fail \emph{too late} in the
pipeline, producing incomprehensible errors. Or, tooling that works for an
object language (like debugging, profiling, IDEs) may not work for metaprogramming, or
metaprogramming may introduce heavy noise and boilerplate, obscuring the logic
of programs and imposing restrictions on code structure.

\textbf{Two-level type theory} (2LTT) \cite{twolevel,staged2ltt} makes it
possible to use expressive dependent type theories for metaprogramming
for a wide range of object languages. In this paper, we use 2LTT to sweeten the
deal of staged compilation, aiming for a combination of strong guarantees, good
ergonomics, high level of abstraction and easy-to-optimize code output.

We develop a particular two-level type theory for this purpose, which we
call \textbf{CFTT}, short for ``closure-free type theory''. This consists
of:
\begin{itemize}
\item A simply-typed object theory with first-order functions, general recursion and
      finitary algebraic data types. This language is easy to optimize and compile
      in the downstream pipeline, but it lacks many convenience features.
\item A dependent type theory for the compile-time language. This
      allows us to recover many features by metaprogramming.
\end{itemize}
Since the object language is first-order, we guarantee that all programs in CFTT
can be ultimately compiled without any dynamic closures, using only calls and
jumps to statically known code. Why emphasize closures?  They are the
foundation to almost all abstraction tools in functional programming:
\begin{itemize}
\item Higher-order functions in essentially all functional languages are implemented with closures.
\item Type classes in Haskell use dictionary-passing, which relies on closures for function methods \cite{DBLP:conf/popl/WadlerB89}.
\item Functors and first-class modules in OCaml \cite{leroy2023ocaml} and other ML-s rely on closures.
\end{itemize}
Hence, doing functional programming without closures is a clear demonstration
that we can get rid of abstraction overheads.

Perhaps surprisingly, little practical programming relies essentially on
closures. Most of the time, programmers use higher-order functions for
\emph{abstraction}, such as when mapping over lists, where it is expected that
the mapping function will be inlined. We note though that our setup is
compatible with closures as well, and it can support two separate type formers
for closure-based and non-closure-based ``static'' functions. Having both of
these would be desirable in a practical system. In the current work we focus on
the closure-free case because it is much less known and developed, and it is
interesting to see how far we can go with it.

%% For example, difference lists
%% \cite{TODO} are often implemented with closures, but they can be also
%% implemented as binary trees, with the same programming interface.

%% \todo{Essential closures: CPS? Threaded interpreters?}
%% Whole-program defunctionalization is also possible, and it is notably used by
%% the MLton compiler \cite{TODO}. While this can be practical and effective, it
%% can be also expensive and require whole-program processing. Also, conceptually
%% speaking, it does not eliminate closures but instead makes them more transparent
%% to optimizations. In this paper we do not use defunctionalization.

\subsection{Contributions}

\begin{itemize}
\item In Section \ref{sec:overview-of-cftt} we present the two-level type theory
  CFTT, where the object level is first-order simply-typed and the meta level is
  dependently typed. The object language supports an operational semantics
  without runtime closures, and can be compiled with only statically known
  function calls. We provide a supplementary Agda formalization of the
  operational semantics of the object language.
\item In Section \ref{sec:monad-transformers} we build a monad transformer library. We believe
  that this is a good demonstration, because monads and monad transformers are
  the most widely used effect system in Haskell, and at the same time their
  compilation to efficient code can be surprisingly difficult. The main idea is
  to have monads and monad transformers \emph{only} in the metalanguage, and try
  to express as much as possible at the meta level, and funnel the object-meta
  interactions through specific \emph{binding-time improvements}.  We show how
  almost all definitions from the Haskell ecosystem of monad transformers can be
  reused in the meta level of CFTT. We also develop let-insertion, join points
  and case switching on object-level data.
\item In Section \ref{sec:stream-fusion} we build a pull-based stream fusion library. Here,
  we demonstrate essential usage of dependent types, in providing guaranteed
  fusion for arbitrary combinations of $\concatMap$ and $\mit{zip}$. We use a
  state machine representation that is based on \emph{sums-of-products} of
  object-level values. We show that CFTT is compatible with a
  \emph{generativity} axiom, which internalizes the fact that metaprograms
  cannot inspect the structure of object-level terms. We use this to show that
  the universe of sums-of-products is closed under $\Sigma$-types. This in turn
  enables a very concise definition of $\concatMap$.
\item
  We adapt the contents of the paper to typed Template Haskell
  \cite{DBLP:journals/pacmpl/XiePLWYW22}, with some modifications,
  simplifications and fewer guarantees about generated code.  In particular,
  Haskell does not have enough dependent types for the simple $\concatMap$
  definition, but we can still work around this limitation. We also provide a
  precise Agda embedding of CFTT and our libraries as described in this
  paper. Here, the object theory is embedded as a collection of postulated
  operations, and we can use Agda's normalization command to print out generated
  object code.
\end{itemize}

\section{Overview of CFTT}\label{sec:overview-of-cftt}

In the following we give an overview of CFTT features. We first review the
meta-level language, then the object-level one, and finally the staging
operations which bridge between the two.

\subsection{The Meta Level}\label{sec:the-meta-level}

$\bs{\MTy}$ is the universe of types in the compile-time language. We will often
use the term ``metatype'' to refer to inhabitants of $\MTy$, and use
``metaprogram'' for inhabitants of metatypes. $\MTy$ supports dependent
functions, $\Sigma$-types and indexed inductive types \cite{inductivefamilies}.

Formally, $\MTy$ is additionally indexed by universes levels (orthogonally to
staging), and we have $\MTy_i : \MTy_{i+1}$. However, universe levels add noise
and they are not too relevant to the current paper, so we will omit them.

Throughout this paper we use a mix of Agda and Haskell syntax for
CFTT. Dependent functions and implicit arguments follow Agda. A basic example:
\begin{alignat*}{3}
  &\id : \{\vA : \MTy\} \to \vA \to \vA\\
  &\id = \lam \vx.\, \vx
\end{alignat*}
Here, the type argument is implicit, and it gets inferred when we use the
function. For example, $\id\,\True$ is elaborated to $\id\,\{\Bool\}\,\True$,
where the braces mark an explicit application for the implicit argument.
Inductive types can be introduced using a Haskell-like ADT notation, or with a
GADT-style one:
\begin{alignat*}{3}
  &                                           &&\hspace{4em}\data\,\Bool_\M &&: \MTy\,\where\\
  & \data\,\Bool_\M : \MTy = \True_\M\,|\,\False_\M &&\hspace{4em}\ind\ind \True_\M &&: \Bool_\M\\
  &                                           &&\hspace{4em}\ind\ind \False_\M &&: \Bool_\M
\end{alignat*}
Note that we added an $_\M$ subscript to the type; when analogous types can be
defined both on the meta and object levels, we will sometimes use this subscript
to disambiguate the meta-level version.

We use Haskell-like newtype notation, such as in $\newtype \msf{Wrap}\,\vA =
\msf{Wrap}\,\{\mit{unWrap} : \vA \}$, and also a similar notation for (dependent)
record types, for instance as in
\[\data \msf{Record} = \msf{Record}\,\{\mit{field1} : \vA,\,\mit{field2} : \vB\}.\]
All construction and elimination rules for type formers in $\MTy$ stay within
$\MTy$. For example, induction on meta-level values can only produce meta-level
values.

\subsection{The Object Level}\label{sec:the-object-level}

$\bs{\Ty}$ is the universe of types in the object language. It is itself a
metatype, so so we have $\Ty : \MTy$. All construction and elimination rules of
type formers in $\Ty$ stay within $\Ty$. We further split $\Ty$ to two
sub-universes.

First, $\bs{\VTy} : \MTy$ is the universe of \emph{value types}. $\VTy$ supports
parameterized algebraic data types, where parameters can have arbitrary types,
but all constructor field types must be in $\VTy$. Since $\VTy$ is a
sub-universe of $\Ty$, we have that when $\vA : \VTy$ then also $\vA :
\Ty$. Formally, this is specified as an explicit embedding operation, but we
will use implicit subtyping for convenience.

Second, $\bs{\CTy} : \MTy$ is the universe of \emph{computation types}. This is
also a sub-universe of $\Ty$ with implicit coercions. For now, we only specify
that $\CTy$ contains functions whose domains are value types:
\[ \blank\to\blank : \VTy \to \Ty \to \CTy \]
For instance, if $\Bool : \VTy$ is defined as an object-level ADT, then $\Bool
\to \Bool : \CTy$, hence also $\Bool \to \Bool : \Ty$. However, $(\Bool \to
\Bool) \to \Bool$ is ill-formed, since the domain is not a value type. Let us look at an
example for an object-level program, where we already have natural numbers
declared as $\data \Nat := \Zero\,|\,\Suc\,\Nat$:
\begin{alignat*}{3}
  &\hspace{-1em}\rlap{$\add : \Nat \to \Nat \to \Nat$}\\
  &\hspace{-1em}\add :=\,&& \mathrlap{\letrec \go\,\vn\,\vm := \case \vn\,\of}\\
  &\hspace{-1em}         && \ind \Zero     &&\to \vm;\\
  &\hspace{-1em}         && \ind \Suc\,\vn &&\to \Suc\,(\go\,\vn\,\vm);\\
  &\hspace{-1em}         && \go
\end{alignat*}
Recursive definitions are introduced with $\letrec$. The general syntax is
$\letrec \vx : \vA := \vt; \vu$, where the $\vA$ type annotation can be
omitted. $\mbf{letrec}$ can be only used to define computations, not values
(hence, only functions can be recursive so far).

Object-level definitions use $:=$ as notation, instead of the $=$ that is used
for meta-level ones. We also have non-recursive $\mbf{let}$, which can be used
to define computations and values alike, and can be used to shadow binders:
\begin{alignat*}{3}
  &\vf : \Nat \to \Nat\\
  &\vf\,\vx := \letdef \vx := \vx + 10; \letdef \vx := \vx + 20; \vx * 10
\end{alignat*}
We also allow $\mbf{newtype}$ definitions, both in $\VTy$ and $\CTy$. These are
assumed to be erased at runtime. In the Haskell implementation they are
important for guiding type class resolution, and we think that the explicit
wrapping makes many definitions more comprehensible in CFTT as well.

Values are call-by-value at runtime; they are computed eagerly in function
applications and $\mbf{let}$-s. $\mbf{let}$-definitions can be used to define
inhabitants of any type, and the type of the $\mbf{let}$ body can be also
arbitrary. Additionally, the right hand sides of $\mbf{case}$ branches can also
have arbitrary types. So the following is well-formed:
\begin{alignat*}{3}
  &\vf : \mathrlap{\Bool \to \Nat \to \Nat}\\
  &\vf\,\vb := \case \vb\,\of \True \to (\lam \vx.\, \vx + 10); \False &&\to (\lam \vx.\, \vx * 10)
\end{alignat*}
In contrast, computations are call-by-name, and the only way we can compute with
functions is to apply them to value arguments. The call-by-name strategy is
fairly benign here and does not lead to significant duplication of computation,
because functions cannot escape their scope; they cannot be passed as arguments
or stored in data constructors. This makes it possible to run object programs
without using dynamic closures. This point is not completely straightforward;
consider the previous $\vf$ function which has $\lambda$-expressions under a
$\mbf{case}$.

However, the call-by-name semantics lets us transform $\vf$ to $\lam
\vb\,\vx.\,\case \vb\,\of \True \to \vx + 10; \False \to \vx * 10$, and more
generally we can transform programs so that every function call becomes
\emph{saturated}. This means that every function call is of the form
$\vf\,\vt_1\,\vt_2\,...\,\vt_n$, where $\vf$ is a function variable and the
definition of $\vf$ immediately $\lambda$-binds $n$ arguments. We do not detail
this here. We provide formal syntax and operational semantics of the object
language in the Agda supplement. We formalized the specific translation steps
that are involved in call saturation, but only specified the full translation
informally.

\subsubsection{Object-level definitional equality} This is a distinct notion
from runtime semantics. Object programs are embedded in CFTT, which is a
dependently typed language, so sometimes we need to decide definitional equality of
object programs during type checking. The setup is simple: we have no $\beta$ or
$\eta$ rules for object programs at all, nor any rule for $\mbf{let}$-unfolding.
The main reason is the following: we care about the size and efficiency of
generated code, and these properties are not stable under $\beta\eta$-conversion
and $\mbf{let}$-unfolding. Moreover, since the object language has general
recursion, we do not have a sensible and decidable notion of program equivalence
anyway.

\subsubsection{Comparison to call-by-push-value}
We took inspiration from call-by-push-value (CBPV) \cite{DBLP:conf/tlca/Levy99},
and there are similarities to our object language, but there are also
significant differences. Both systems have a value-computation distinction, with
call-by-name computations and call-by-value values. However, our object theory
supports variable binding at arbitrary types while CBPV only supports value
variables. In CBPV, a let-definition for a function is only possible by first
packing it up as a closure value (or ``thunk''), which clearly does not suit our
applications.  Investigating the relation between CBPV and our object language
could be future work.

%% \subsubsection{Spectrum of possible object languages}
%% There is a trade-off between making the object language more restricted, and
%% thus easier to compile, and making metaprogramming more convenient. We will see
%% that the ability to insert $\mbf{let}$-s without restriction is convenient, and
%% likewise the ability to have $\lambda$-abstraction under $\mbf{case}$ bodies,
%% although these features necessitate more downstream processing, to bring
%% programs to saturated form. On the other side of the spectrum, one might imagine
%% generating object code in low-level A-normal form. This is more laborious, but
%% it could be also interesting, because it forces us to invent abstractions for
%% manipulating ANF in the metalanguage. In a similar vein, Allais recently
%% proposed metaprogramming quantum circuits in a two-level type theory
%% \cite{TODO}.

%% more restricted, and thus easier to compile, and making \emph{metaprogramming}
%% more convenient. We will see that the ability to insert $\mbf{let}$-s without
%% restriction is very convenient in code generation, and likewise the ability to
%% have arbitrary object expressions in $\mbf{case}$ bodies. In this paper we go
%% with the most liberal object syntax, at the cost of needing more downstream
%% processing. The call-by-name nature of computation types requires a bit of a
%% change of thinking from programmers, but we believe that it is well worth to
%% have for the metaprogramming convenience.

%% On the other side of the spectrum, one might imagine generating object code in
%% low-level A-normal form. This is more laborious, but it could be also
%% interesting, because it forces us to invent abstractions for manipulating ANF in
%% the metalanguage. In a similar vein, Allais recently proposed metaprogramming
%% quantum circuits in a two-level type theory \cite{TODO}. We leave such setups
%% with low-level object languages to future investigation.


%% Finally, one might compare our object language to call-by-push-value
%% (CBPV). Indeed, we took inspiration from CBPV, and there are similarities, but
%% also differences. In both systems there is a value-computation distinction, and
%% values are call-by-value, and computations are call-by-name. However, our object
%% language allows variable binding at arbitrary types, while CBPV only supports it
%% at value types. In CBPV, a let-definition for a function is only possible by
%% first packing it up as a closure value (or: ``thunk''), which clearly does not
%% work for us.  Also, CBPV makes a judgment-level structural distinction between
%% values and computations, while we use type universes for that. Generally
%% speaking, type-based restrictions are easier to work with in dependent type
%% theories than structural restrictions.

%% The reasons are the following. First, since the object
%% language has general recursion, most $\beta$-rules are only valid in the
%% operational semantics up to some restrictions, and $\beta$-conversion is not
%% decidable to begin with. Second, in metaprogramming we care about the size and
%% efficiency of generated code, and these properties are not stable under $\beta$
%% and $\eta$. Hence, the sensible choice is to do metaprogramming up to strict
%% syntactic equality of object programs.

%% Let us discuss the object language. First, notice that there is no polymorphism
%% or any kind of type dependency. Although we can define lists as $\data \List\,A
%% : \VTy := \Nil\,|\,\Cons\,A\,(\List\,A)$, the parameterization is just a
%% shorthand; all concrete instantiations of the type are distinct. This
%% monomorphism makes it easy to use different memory layouts for different types.
%% For example, types which look like products may be unboxed. We could also make a
%% distinction between boxed and unboxed sum types. We do not explore this in
%% detail, we just note that monomorphic types make it easy to control memory
%% layouts, and we believe that this is an important part of performance
%% optimization.

%% Second, there are no higher-order functions, and functions also cannot be stored
%% in data structures. Hence, locally defined functions can never escape their
%% scope, and all function calls are to functions defined in the current
%% scope. This makes it possible to run object programs without using dynamic
%% closures. This latter point might not be completely straightforward; what
%% about the previous $f$ function which has $\lambda$-expressions under a
%% $\mbf{case}$, should that require closures?

%% We say that it should not. We make this formal formal in Section \ref{TODO};
%% here we only give an intuitive explanation. In short, we choose a call-by-name
%% runtime semantics for functions, which means that the only way we can compute
%% with a function is by applying it to all arguments and extracting the resulting
%% value.  Hence, the only way to compute with $f$ is to apply it to \emph{two}
%% arguments, so $f$ is operationally equivalent to $\lam b\,x.\,\case b\,\of \True
%% \to x + 10; \False \to x * 10$.  In Section \ref{TODO} we show that all object
%% programs can be transformed to a \emph{saturated} form, where the arity of every
%% function call matches the number of topmost $\lambda$-binders in the definition
%% of the called function. Then, the local functions can be compiled either to
%% top-level functions by lambda-lifting, or if they are only tail-called, left in
%% place as join points \cite{TODO}.

%% Why not just make the object language less liberal, e.g.\ by disallowing
%% $\lambda$ under $\mbf{case}$ or $\mbf{let}$, thereby making call saturation
%% easier or more obvious? There is a trade-off between making the object language
%% more restricted, and thus easier to compile, and making \emph{metaprogramming}
%% more convenient. We will see that the ability to insert $\mbf{let}$-s without
%% restriction is very convenient in code generation, and likewise the ability to
%% have arbitrary object expressions in $\mbf{case}$ bodies. In this paper we go
%% with the most liberal object syntax, at the cost of needing more downstream
%% processing. The call-by-name nature of computation types requires a bit of a
%% change of thinking from programmers, but we believe that it is well worth to
%% have for the metaprogramming convenience.

%% On the other side of the spectrum, one might imagine generating object code in
%% low-level A-normal form. This is more laborious, but it could be also
%% interesting, because it forces us to invent abstractions for manipulating ANF in
%% the metalanguage. In a similar vein, Allais recently proposed metaprogramming
%% quantum circuits in a two-level type theory \cite{TODO}. We leave such setups
%% with low-level object languages to future investigation.

%% Finally, one might compare our object language to call-by-push-value
%% (CBPV). Indeed, we took inspiration from CBPV, and there are similarities, but
%% also differences. In both systems there is a value-computation distinction, and
%% values are call-by-value, and computations are call-by-name. However, our object
%% language allows variable binding at arbitrary types, while CBPV only supports it
%% at value types. In CBPV, a let-definition for a function is only possible by
%% first packing it up as a closure value (or: ``thunk''), which clearly does not
%% work for us.  Also, CBPV makes a judgment-level structural distinction between
%% values and computations, while we use type universes for that. Generally
%% speaking, type-based restrictions are easier to work with in dependent type
%% theories than structural restrictions.

\subsection{Staging}\label{sec:staging}

With what we have seen so far, there is no interaction between the meta and
object levels. We make such interaction possible with \emph{staging primitives}.

\begin{itemize}
\item For $\vA : \Ty$, we have $\Lift \vA : \MTy$, pronounced as ``lift $\vA$''. This is
      the type of metaprograms that produce $\vA$-typed object programs.
\item For $\vA : \Ty$ and $\vt : \vA$, we have $\ql \vt \qr : \Lift \vA$, pronounced ``quote $\vt$''. This
      is the metaprogram which immediately returns $\vt$.
\item For $\vt : \Lift \vA$, we have $\spl \vt : A$, pronounced ``splice
  $\vt$''. This inserts the result of a metaprogram into an object
  term. \emph{Notation:} splicing binds stronger than function application, so
  $\vf\,\spl \vx$ is parsed as $\vf\,(\spl \vx)$. We borrow this from
  MetaML \cite{metaml}.
\item We have $\ql \spl \vt \qr \equiv \vt$ and $\spl \ql \vt \qr \equiv \vt$ as definitional equalities.
\end{itemize}

A CFTT program is a mixture of object-level and meta-level top-level definitions
and declarations. We use \emph{unstaging} to refer to the process of evaluating
metaprograms in a CFTT program in order to extract purely object-level code.
The output is guaranteed to be well-typed.

Let us look at some basic examples. Recall the meta-level identity function; it
can be used at the object-level too, by applying it to quoted terms:
  \[ \letdef \vn : \Nat := \spl(\id\,\ql 10 + 10 \qr); ... \]
Here, $\id$ is used at type $\Lift \Nat$. During unstaging, the expression in
the splice is evaluated, so we get $\spl\ql 10 + 10 \qr$, which is
definitionally the same as $10 + 10$, which is our code output here. Boolean
short-circuiting is another basic use-case:
\begin{alignat*}{3}
  &\mit{and} : \Up\Bool \to \Up\Bool \to \Up\Bool\\
  &\mit{and}\,\vx\,\vy = \ql\case \spl \vx\,\of \True \to \spl \vy; \False \to \False\qr
\end{alignat*}
Since the $y$ expression is inlined under a $\case$ branch at every use site, it
is only computed at runtime when $x$ evaluates to $\True$. In many situations,
staging can be used instead of laziness to implement short-circuiting, and often
with better runtime performance, avoiding the overhead of thunking. Consider the
$\map$ function now:
\begin{alignat*}{3}
  & \hspace{-5em}\mathrlap{\map : \{\vA\,\vB : \VTy\} \to (\Up \vA \to \Up \vB) \to \Up (\List\,\vA) \to \Up(\List\,\vB)}\\
  & \hspace{-5em}\mathrlap{\map\,\vf\,\vas = \ql\letrec \go\,\vas := \case \vas\,\of}\\
  & \hspace{-5em}\ind\ind\ind\ind\ind \ind\Nil             &&\to \Nil;\\
  & \hspace{-5em}\ind\ind\ind\ind\ind \ind\Cons\,\va\,\vas &&\to \Cons\,\spl(\vf\,\ql \va \qr)\,(\go\,\vas);\\
  & \hspace{-5em}\ind\ind\ind\ind\ind\go\,\spl\vas \qr
\end{alignat*}
For example, this can be used as $\letdef \vf\,\vas : \List\,\Nat \to
\List\,\Nat := \spl(\map\,(\lam \vx.\,\ql \spl \vx + 10 \qr)\,\qt{\vas}) $.
This is unstaged to a recursive definition where the mapping function is inlined
into the $\Cons$ case as $\Cons\,\va\,\vas \to \Cons\,(\va + 10)\,(\go\,\vas)$.
Note that $\map$ has to abstract over value types, since lists can only contain
values, not functions. Also, the mapping function has type $\Up \vA \to \Up
\vB$, instead of $\Up (\vA \to \vB)$. The former type is often preferable to the
latter; the former is a metafunction with useful computational content, while
the latter is merely a black box that computes object code. If we have $\vf :
\Up(\vA \to \vB)$, and $\vf$ computes to $\ql \lam \vx.\,\vt\qr$, then $\spl
\vf\,\vu$ is unstaged to an undesirable ``administrative'' $\beta$-redex $(\lam
\vx.\,\vt)\,\vu$.

\subsection{Staged Semantics of CFTT}\label{sec:semantics-of-staging}

We give an overview of the unstaging algorithm together with its correctness.
We reuse the concepts and most of the results from \cite{staged2ltt}. Our main
task is to extend ibid.\,with identity types and inductive types in the
metalanguage, since we use them in this paper. For inductive types, it is the
most sensible to add W-types, since all inductive families can be faithfully
derived from them in our setting \cite{whynotw}.

This however requires a substantial amount of technical background from
\cite{staged2ltt}, so we relegate it to Appendix \ref{appendix:A}. Here we only
review the basics.
\begin{itemize}
\item The input of unstaging is a CFTT term with an object-level type, which
      only depends on object-level free variables. The output is a term in the
      object theory. In the other direction, there is an evident embedding of
      object-theoretic terms as CFTT terms.
\item \emph{Soundness} means that unstaging followed by embedding is the
      identity map, up to conversion. In other words, unstaging respects conversion
      of CFTT terms.
\item \emph{Stability} means that embedding followed by unstaging is the
      identity map. In other words, unstaging has no action on terms which are
      already purely object-level (i.e.\ contain no splices).
\end{itemize}
In \cite{staged2ltt}, there is one more property, called \emph{strictness},
which means that unstaging does not perform $\beta\eta$-conversion in the
object theory, but this holds trivially in our case since we do not have such
conversion rules.

However, for CFTT we slightly depart from the above notion of soundness, for the
following reason. We would like to assume certain propositional identities as
\emph{axioms} in the metalanguage, without blocking unstaging as a
computation. We use such an axiom to good effect in Section
\ref{sec:concatmap-for-streams}.

Axioms clearly block computation, so they are incompatible with the mentioned
notion of soundness. But we are only interested in equations which hold
definitionally during unstaging. Hence, we can do the following: first, we erase
all identity proofs and their transports from CFTT programs, then we proceed
with unstaging as in \cite{staged2ltt}. This erasure can be formalzied as a
syntactic translation \cite{next700} from CFTT to CFTT extended with the
equality reflection principle \cite{TODO}. Thus, soundness of staging for CFTT
holds up to erasure of identity proofs.

\section{Monads \& Monad Transformers}\label{sec:monad-transformers}

In this section we build a library for monads and monad transformers. We believe
that this is a good demonstration of CFTT-s abilities, since monads are
ubiquitous in Haskell programming and they also introduce a great amount of
abstraction that should be optimized away.

\subsection{Binding-Time Improvements}\label{sec:binding-time-improvements}

We start with some preparatory work before getting to monads. We saw that $\Up
\vA \to \Up \vB$ is usually preferable to $\Up(\vA \to \vB)$. The two types are
equivalent during unstaging, up to the runtime equivalence of object programs,
and we can convert back and forth in CFTT:
\begin{alignat*}{3}
  &\mup : \Up (\vA \to \vB) \to \Up \vA \to \Up \vB && \ind\ind \mdown : (\Up \vA \to \Up \vB) \to \Up (\vA \to \vB) \\
  &\mup\,\vf\,\va = \ql \spl \vf\, \spl \va\qr   && \ind\ind \mdown\,\vf = \ql \lam \va.\,\spl(\vf\,\ql \va \qr) \qr
\end{alignat*}
We cannot show internally, using propositional equality, that these functions are
inverses, since we do not have $\beta\eta$-rules for object functions; but we
will not need this proof in the rest of the paper.

In the staged compilation and partial evaluation literature, the term
\emph{binding time improvement} is used to refer to such conversions, where the
``improved'' version supports more compile-time computation
\cite{partial-evaluation}. A general strategy for generating efficient ``fused''
programs, is to try to work as much as possible with improved representations,
and only convert back to object code at points where runtime dependencies are
unavoidable. Let us look at binding-time-improvement for product types now:
\begin{alignat*}{3}
  &\mup : \Up (\vA,\,\vB) \to (\Up \vA,\,\Up \vB) && \ind\ind \mdown : (\Up \vA,\,\Up \vB) \to \Up(\vA,\,\vB) \\
  &\mup\,\vx = (\qt{\fst\,\spl \vx},\, \qt{\snd\,\spl \vx})   && \ind\ind \mdown\,(\vx,\,\vy) = \qt{(\spl \vx,\,\spl \vy)}
\end{alignat*}

Here we overload Haskell-style product type notation, both for types and the
pair constructor, at both levels. There is a problem with this conversion
though: $\mup$ uses $\vx : \Up(\vA,\,\vB)$ twice, which can increase code size
and duplicate runtime computations. For example, $\mdown\,({\mup\,\ql \vf\,\vx
  \qr})$ is staged to $\ql (\fst\,(\vf\,\vx),\,\snd\,(\vf\,\vx)) \qr$. It would
be safer to first let-bind an expression with type $\Up(\vA,\,\vB)$, and then
only use projections of the newly bound variable. This is called
\emph{let-insertion} in staged compilation. But it is impossible to use
let-insertion in $\mup$ because the return type is in $\MTy$, and we cannot
introduce object binders in meta-level code.  Fortunately, there is a principled
solution.

\subsection{The Code Generation Monad}\label{sec:the-code-generation-monad}

The solution is to use a monad which extends $\MTy$ with the ability to
freely generate object-level code and introduce object binders.
\begin{alignat*}{3}
  & \newtype \Gen\,(\vA : \MTy) = \Gen\,\{\unGen : \{\vR : \Ty\} \to (\vA \to \Up \vR) \to \Up \vR\}
\end{alignat*}
This is a monad in $\MTy$ in a standard sense:
\begin{alignat*}{3}
  &\mathrlap{\instance \Monad\,\Gen\,\where}\\
  &\ind \return \va        &&= \Gen \dlr \lam \vk.\,\vk\,\va\\
  &\ind \mit{ga} \bind \vf &&= \Gen \dlr \lam \vk.\,\unGen\,\mit{ga}\,(\lam \va.\,\unGen\,(\vf\,\va)\,\vk))
\end{alignat*}
From now on, we reuse Haskell-style type classes and $\mbf{do}$-notation in
CFTT. We will use type classes in an informal way, without precisely specifying
how they work. However, type classes are used in essentially the same way in the
Agda and Haskell implementations, with modest technical differences.

From the $\Monad$ instance, the $\Functor$ and $\Applicative$ instances can be
also derived. We reuse $({\fmap})$ and $({\ap})$ for applicative notation as well.
An inhabitant of $\Gen\,\vA$ can be viewed as an action whose effect is to produce
some object code. $\Gen$ can be only ``run'' when the result type is an object
type:
\begin{alignat*}{3}
  &\runGen : \Gen\,(\Up \vA) \to \Up \vA\\
  &\runGen\,\vma = \unGen\,\vma\,\id
\end{alignat*}
We can let-bind object expressions in $\Gen$:
\begin{alignat*}{3}
  & \gen : \{\vA : \Ty\} \to \Up \vA \to \Gen\,(\Up \vA) \\
  & \gen\,\va = \Gen \dlr \lam \vk.\,\ql \letdef \vx : \vA := \spl \va; \spl(\vk\,\ql \vx \qr) \qr
\end{alignat*}
And also recursive definitions of computations:
\begin{alignat*}{3}
  & \genRec : \{\vA : \CTy\} \to (\Up \vA \to \Up \vA) \to \Gen\,(\Up \vA) \\
  & \genRec\,\vf = \Gen \dlr \lam \vk.\,\qt{\letrec \vx : \vA := \spl(\vf\,\qt{\vx}); \spl(\vk\,\qt{\vx})}
\end{alignat*}
Now, using do-notation, we may write $\mdo \{\vx \fro \gen\,\qt{10 + 10}; \vy \fro
\gen\,\qt{20 + 20}\};\return \qt{\vx + \vy}\}$, for a $\Gen\,(\Up \Nat)$
action. Running this with $\runGen$ yields $\qt{\letdef \vx := 10 + 10; \letdef \vy
  := 20 + 20; \vx + \vy}$. We can also define a ``safer'' binding-time improvement for
products, using let-insertion:
\begin{alignat*}{6}
  &\hspace{-2em}\mathrlap{\mup : \Up (\vA,\,\vB) \to \Gen\,(\Up \vA,\,\Up \vB)            \hspace{5em} \mdown : \Gen\,(\Up \vA,\, \Up \vB) \to \Up(\vA,\,\vB)} \\
  &\hspace{-2em}\mup\,\vx = \mdo &&\vx \fro \gen\,\vx                                    && \ind\ind \mdown\,\vx = \mdo &&(\va,\,\vb) \fro \vx\\
  &\hspace{-2em}               &&\return (\qt{\fst\,\spl \vx},\,\qt{\snd\,\spl \vx})     &&                           &&\return \qt{(\spl \va,\, \spl \vb)}
\end{alignat*}

Working in $\Gen$ is convenient, since we can freely generate object code and
also have access to the full metalanguage. Also, the point of staging is that
\emph{eventually} all metaprograms will be used for the purpose of code
generation, so we eventually $\runGen$ all of our actions. So why not
just always work in $\Gen$? The implicit nature of $\Gen$ may make it harder to
reason about the size and content of generated code. This is a bit similar to
the $\msf{IO}$ monad in Haskell, where eventually everything needs to run in
$\msf{IO}$, but we may not want to write all of our code in $\msf{IO}$.

\subsection{Monads}\label{sec:monads}

Let us start with the $\Maybe$ monad. We have $\data \Maybe\,\vA :=
\Nothing\,|\,\Just\,\vA$, and $\Maybe$ itself is available as a $\VTy \to \VTy$
metafunction. However, we cannot directly fashion a monad out of $\Maybe$, since
we do not have enough type formers in $\VTy$. We could try to use the following
type for binding:
\[ (\bind) : \Up(\Maybe\,\vA) \to (\Up\vA \to \Up(\Maybe\,\vB)) \to \Up(\Maybe\,\vB) \]
This works, but the definition necessarily uses runtime case splits on $\Maybe$
values, many of which could be optimized away during staging. Also, not having
a ``real'' monad is inconvenient for the purpose of code reuse.

Instead, our strategy is to only use proper monads in $\MTy$, and convert
between object types and meta-monads when necessary, as a form of binding-time
improvement. We define a class for this conversion:
\begin{alignat*}{3}
  &\hspace{-9em}\mathrlap{\tyclass\,\MonadGen\,\vM \RA \Improve\,(\vF : \VTy \to \Ty)\,(\vM : \MTy \to \MTy)\,\where} \\
  &\hspace{-9em}\ind \mup   &&: \{\vA : \VTy\} \to \Up(\vF\,\vA) \to \vM\,(\Up \vA)\\
  &\hspace{-9em}\ind \mdown &&: \{\vA : \VTy\} \to \vM\,(\Up \vA) \to \Up(\vF\,\vA)
\end{alignat*}
Assume that $\Maybe_\M$ is the standard meta-level monad, and $\MaybeT_\M$ is
the standard monad transformer, defined as follows:
\[ \newtype \MaybeT_\M\,\vM\,\vA = \MaybeT_\M\,\{\runMaybeT_\M : \vM\,(\Maybe_\M\,\vA)\} \]
Now, the binding-time improvement of $\Maybe$ is as follows:
\begin{alignat*}{3}
  &\instance \Improve\,\Maybe\,(\MaybeT_\M\,\Gen)\,\where\\
  %% &\ind \mup : \Up\,(\Maybe\,\vA) \to \MaybeT_\M\,\Gen\,(\Up\,\vA) \\
  &\ind \mup\,\vma = \MaybeT_\M \dlr \Gen \dlr \lam \vk.\\
  &\ind\ind \qt{\case \spl \vma\,\of \Nothing \to \spl(\vk\,\Nothing_\M);\Just\,\va \to \spl(\vk\,(\Just_\M\qt{\va}))}\\
  %% &\ind \mdown : \MaybeT_\M\,\Gen\,(\Up\,A) \to \Up\,(\Maybe\,A) \\
  &\ind \mdown\,(\MaybeT_\M\,(\Gen\,\vma)) = \\
  &\ind\ind \vma\,(\lam \vx.\,\case \vx\,\of \Nothing_\M \to \qt{\Nothing}; \Just_\M\,\va \to \qt{\Just\,\spl \va})
\end{alignat*}
With this, we get the $\Monad$ instance for free from $\MaybeT_\M$ and $\Gen$. A small example:
\[ \letdef \vn : \Maybe\,\Nat := \spl(\mdown\,\$\,\mdo \{\vx \fro \return\,\qt{10}; \vy \fro \return\,\qt{20}; \return\,\qt{\vx + \vy})\});\,... \]
Since $\MaybeT_\M$ is meta-level, its monadic binding fully computes at unstaging time. Thus, the above code is computed to
\[ \letdef \vn : \Maybe\,\Nat := \Just\,(10 + 20);\;... \]
Assume also a $\lift : \Monad\,\vM \RA \vM\,\vA \to \MaybeT_\M\,\vM\,\vA$ operation which
comes from $\MaybeT_\M$ being a monad transformer. We can do let-insertion in
$\MaybeT_\M\,\Gen$ by simply lifting:
\begin{alignat*}{3}
  &\gen' : \Up \vA \to \MaybeT_\M\,\Gen\,(\Up \vA) \\
  &\gen'\,\va = \lift\,(\gen\,\va)
\end{alignat*}
However, it is more convenient to proceed in the style of Haskell's monad
transformer library \texttt{mtl} \cite{mtl}, and have a class for monads that
support code generation:
\begin{alignat*}{3}
&\tyclass\,\Monad\,\vM \RA \MonadGen\,\vM\,\where\\
&\ind \liftGen : \Gen\,\vA \to \M\,\vA\\
&\instance \MonadGen\,\Gen\,\where\,\liftGen = \id \\
&\instance \MonadGen\,\vM \RA \MonadGen\,(\MaybeT_\M\,\vM)\,\where\,\liftGen = \lift \circ \liftGen
\end{alignat*}
The $\MonadGen$ instance can be defined uniformly for every monad transformer as
$\liftGen = \lift \circ \liftGen$. We also redefine $\gen$ and $\genRec$ to work
in any $\MonadGen$, so from now on we have:
\begin{alignat*}{3}
 &\gen   &&: \MonadGen\,\vM \RA \Up \vA \to \vM\,(\Up \vA)\\
 &\genRec &&: \MonadGen\,\vM \RA (\Up \vA \to \Up \vA) \to \vM\,(\Up \vA)
\end{alignat*}

\subsubsection{Case splitting in monads}
We often want to case-split on object-level data inside a monadic action, and
perform different actions in different branches. At first, this may seem
problematic, because we cannot directly compute metaprograms from object-level
case splits. Fortunately, with a little bit more work, this is actually possible
in any $\MonadGen$, for any value type.

We demonstrate this for lists. An object-level $\mbf{case}$ on lists introduces
two points where code generation can continue. We define a metatype which gives
us a ``view'' on these points:
\[ \data \msf{SplitList}\,\vA = \Nil'\,|\,\Cons'\,(\Up \vA)\,(\Up (\List\,\vA)) \]
We can generate code for a case split, returning a view on it:
\begin{alignat*}{3}
  &\mit{split} : \Up (\List\,\vA) \to \Gen\,(\msf{SplitList}\,\vA)\\
  &\mit{split}\,\vas = \Gen \dlr \lam \vk.\,\qt{\case \spl \vas\,\of \Nil \to \spl(\vk\,\Nil'); \Cons\,\va\,\vas \to \spl(\vk\,(\Cons'\,\qt{a}\,\qt{\vas}))}
\end{alignat*}
Now, in any $\MonadGen$, assuming $\vas : \Up(\List\,\vA)$, we may write
\begin{alignat*}{3}
  &\mdo \{\mit{sp} \fro \liftGen\;(\!\mit{split}\,\vas);(\case \mit{sp}\,\of \Nil' \to ...;\Cons'\,\va\,\vas \to ...)\}
\end{alignat*}
This can be generalized to splitting on any object value. In the Agda and
Haskell implementations, we overload $\mit{split}$ with a class similar to the
following:
\begin{alignat*}{3}
  & \hspace{-2em}\mathrlap{\tyclass\,\mit{Split}\,(\vA : \VTy)\,\where}\\
  & \hspace{-2em}\ind \mit{SplitTo} &&: \MTy \\
  & \hspace{-2em}\ind \mit{split}   &&: \Lift \vA \to \Gen\,\mit{SplitTo}
\end{alignat*}
In a native implementation of CFTT it may make sense to extend do-notation, so
that we elaborate $\mbf{case}$ on object values to an application of the
appropriate $\mit{split}$ function. We adopt this in the rest of the paper, so
when working in a $\MonadGen$, we can write $\case \vas\,\of \Nil \to ...;
\Cons\,\va\,\vas \to ...$, binding $\va : \Up \vA$ and $\vas : \Up(\List\,\vA)$
in the $\Cons$ case.

\subsection{Monad Transformers}\label{monad-transformers}

At this point, it makes sense to aim for a monad transformer library where
binding-time improvement is defined compositionally, by recursion on the
transformer stack. The base case is the following:
\begin{alignat*}{3}
  & \hspace{-2em}\mathrlap{\newtype \Identity\,\vA := \Identity\,\{\runIdentity : \vA\} }\\
  & \hspace{-2em}\mathrlap{\instance \Improve\,\Identity\,\Gen\,\where}\\
  & \hspace{-2em}\ind \mup\,\vx   &&= \return \qt{\runIdentity\,\spl \vx}\\
  & \hspace{-2em}\ind \mdown\,\vx &&= \unGen\,\vx \dlr \lam \va.\,\qt{\Identity\,\spl \va}
\end{alignat*}
We recover the object-level $\Maybe$ as $\MaybeT\,\Identity$, from the following $\MaybeT$:
\[ \newtype \MaybeT\,(\vM : \VTy \to \Ty)\,(\vA : \VTy) := \MaybeT\,\{\runMaybeT : \vM\,(\Maybe\,\vA)\} \]
With this, improvement can be generally defined for $\MaybeT$:
\begin{alignat*}{3}
  &\instance \Improve\,\vF\,\vM \RA \Improve\,(\MaybeT\,\vF)\,(\MaybeT_\M\,\vM)\,\where\\
  &\ind \mup\,\vx = \MaybeT_\M \dlr \mdo\\
  &\ind\ind \vma \fro \mup\,\qt{\runMaybeT\,\spl \vx}\\
  &\ind\ind \case \vma\,\of \Nothing \to \return \Nothing_\M\\
  &\hspace{6.5em}         \Just\,\va \hspace{1.15em}\to \return (\Just_\M\,\va)\\
  &\ind \mdown\,(\MaybeT_\M\,\vx) = \qt{\MaybeT\,\spl(\mdown \dlr \vx \bind \lam\case\\
  &\ind\ind\Nothing_\M \to \return \qt{\Nothing}\\
  &\ind\ind\Just_\M\,\va\hspace{1.15em} \to \return \qt{\Just\,\spl \va})}
\end{alignat*}
In the $\mbf{case}$ in $\mup$, we use our syntax sugar for matching on a
$\Maybe$ value inside an $\vM$ action. This is legal, since we know from the
$\Improve\,\vF\,\vM$ assumption that $\vM$ is a $\MonadGen$. In $\mdown$ we also use
$\lam\case ...$ to shorten $\lam \vx.\,\case \vx\,\of ...$.

In the meta level, we can reuse essentially all definitions from \texttt{mtl}.
From there, only the continuation monad transformer fails to support
binding-time-improvement in CFTT, because of the obvious need for dynamic
closures. In the following we only present $\StateT$ and $\ReaderT$.

We start with $\StateT$. We assume $\StateT_\M$ as the standard meta-level
definition. The object-level $\StateT$ has type $(\vS : \VTy)(\vF : \VTy \to
\Ty)(\vA : \VTy) \to \VTy$; the state parameter $\vS$ has to be a value type,
since it is an input to an object-level function.
\begin{alignat*}{3}
  &\instance \Improve\,\vF\,\vM \RA \Improve\,(\StateT\,\vS\,\vF)\,(\StateT_\M\,(\Up \vS)\,\vM)\,\where\\
  &\ind \mup\,\vx = \StateT_\M \dlr \lam \vs.\,\mdo\\
  &\ind\ind \mit{as} \fro \mup\,\qt{\runStateT\,\spl \vx\,\spl \vs}\\
  &\ind\ind \case \mit{as}\,\,\of (\va,\,\vs) \to \return (\va,\,\vs)\\
  &\ind \mdown\,\vx = \qt{ \StateT\,(\lam \vs.\, \spl(\mdown \dlr \mdo\\\
  &\ind \ind (\va,\,\vs) \fro \runStateT_\M\,\vx\,\qt{\vs}\\
  &\ind \ind \return \qt{(\spl \va,\, \spl \vs)}))}
\end{alignat*}
Like before in $\MaybeT$, we rely on object-level case splitting in the
definition of $\mup$. For $\Reader$, the environment parameter also has to be a
value type, and we define improvement as follows.
\begin{alignat*}{3}
  &\hspace{-5em}\mathrlap{\instance \Improve\,\vF\,\vM \RA \Improve\,(\ReaderT\,\vR\,\vF)\,(\ReaderT_\M\,(\Up \vR)\,\vM)\,\where}\\
  &\hspace{-5em}\ind \mup\,\vx   &&= \ReaderT_\M \dlr \lam \vr.\, \mup\,\qt{\runReaderT\,\spl \vx\,\spl \vr}\\
  &\hspace{-5em}\ind \mdown\,\vx &&= \qt{\ReaderT\,(\lam \vr.\,\spl(\mdown\,(\runReaderT_\M\,\vx\,\qt{\vr})))}
\end{alignat*}

\subsubsection{$\State$ and $\Reader$ operations} If we use
the $\modify$ function that we already have in $\State_\M$, a curious thing
happens. The meaning of $\modify\,(\lam \vx.\,\qt{\spl \vx + \spl \vx})$ is to
replace the current state $\vx$, as an object expression, with the expression
$\qt{\spl \vx + \spl \vx}$, and this happens at unstaging time. This behaves as an
``inline'' modification which replaces every subsequent mention of the state
with a different expression. For instance, ignoring newtype wrappers for now,
\[ \mdown \dlr \mdo \{\modify\,(\lam \vx.\,\qt{\spl \vx + \spl \vx}); \modify\,(\lam \vx.\,\qt{\spl \vx + \spl \vx});\return\,\qt{()}\} \]
is unstaged to
\[ \qt{\lam \vx.\,((),\,(\vx + \vx) + (\vx + \vx))} \]
which duplicates the evaluation of $\vx + \vx$. The solution is to force the
evaluation of the new state in the object language, by let-insertion. A similar
phenomenon happens with the $\mit{local}$ function in $\Reader$. So we define
``stricter'' versions of these operations. We also return $\Up ()$ from
actions instead of $()$ --- the former is more convenient, because the $\mdown$
operation can be immediately used on it.
\begingroup
\allowdisplaybreaks
\begin{alignat*}{3}
  & \mput' : (\MonadState\,(\Up \vS)\,\vM,\,\MonadGen\,\vM) \RA \Up \vS \to \vM\,(\Up ()) \\
  & \mput'\,\vs = \mdo \{\vs \fro \gen\,\vs; \mput\,\vs; \return \qt{()}\}\\
  &\\
  & \modify' : (\MonadState\,(\Up \vS)\,\vM,\,\MonadGen\,\vM) \RA (\Up \vS \to \Up \vS) \to \vM\,(\Up ()) \\
  & \modify'\,\vf = \mdo \{\vs \fro \get; \mput'\,(\vf\,\vs)\}\\
  &\\
  & \mit{local'} : (\MonadReader\,(\Up \vR)\,\vM,\,\MonadGen\,\vM) \RA (\Up \vR \to \Up \vR) \to \vM\,\vA \to \vM\,\vA\\
  & \mit{local'}\,\vf\,\vma = \mdo \{\vr \fro \mit{ask}; \vr \fro \gen\,(\vf\,\vr);\mit{local}\,(\lam \_.\,\vr)\,\vma\}
\end{alignat*}
\endgroup
Now,
\[ \mdown \dlr \mdo \{\modify'\,(\lam \vx.\,\qt{\spl \vx + \spl \vx}); \modify'\,(\lam \vx.\,\qt{\spl \vx + \spl \vx})\} \]
is unstaged to
\[ \qt{\lam \vx.\,\letdef x := \vx + \vx; \letdef \vx := \vx + \vx; ((),\,\vx)} \]

\subsection{Joining Control Flow in Monads}\label{sec:joining-control-flow-in-monads}

There is a deficiency in our library so far. Assuming $\vb : \Up\Bool$, consider:
\begin{alignat*}{5}
  & \hspace{-1em}\mdo && (\case \vb\,\of\,\True\to \mput' \qt{10}; \False &&\to \mput' \qt{20})\\
  & \hspace{-1em}     && \mathrlap{\modify' (\lam \vx.\,\qt{\spl \vx + 10})}
\end{alignat*}
The $\modify' (\lam \vx. \qt{\spl \vx + 10})$ action gets inlined in both case
branches during unstaging. This follows from the definition of monadic binding in
$\Gen$ and the $\mit{split}$ function in the desugaring of $\mbf{case}$. Code
generation is continued in both branches with the same action. If we have
multiple Boolean $\mbf{case}$ splits sequenced after each other, that yields
exponential code size. Right now, we can fix this by let-binding the problematic
action:
%% \begin{alignat*}{5}
%%   & \mdo && (\gen \dlr \mdown\,(\case \vb\,\of \True \to \mput' \qt{10};\False &&\to \mput' \qt{20})) \bind \mup \\
%%   &      && \mathrlap{\modify' (\lam \vx.\,\qt{\spl \vx + 10})}
%% \end{alignat*}
\begin{alignat*}{5}
  & \mdo && \msf{action} \fro \gen \dlr \mdown\,(\case \vb\,\of \True \to \mput' \qt{10};\False &&\to \mput' \qt{20}) \\
  &      && \mup\,\msf{action}\\
  &      && \mathrlap{\modify' (\lam \vx.\,\qt{\spl \vx + 10})}
\end{alignat*}
This removes code duplication by round-tripping through an object-level
$\mbf{let}$. This solution is fairly good in a state monad, where $\mdown$ only
introduces a runtime pair constructor, and it is feasible to compile
object-level pairs as unboxed data, without overheads. However, for $\Maybe$,
$\mdown$ introduces a runtime $\Just$ or $\Nothing$, and $\mup$ introduces a
runtime case split. A better solution would be to introduce two let-bound
\emph{join points} before the offending $\mbf{case}$, one for returning a
$\Just$ and one for returning $\Nothing$, but fusing away the actual runtime
constructors. So, for example, we would like to produce object code which looks
like the following:
\begin{alignat*}{3}
  & \letdef\,\mit{joinJust}\,\vn := ...\\
  & \letdef\,\mit{joinNothing}\,() := ...\\
  & \case \vx == 10\,\of\\
  & \ind \True \hspace{0.2em}\to \mit{joinJust}\,(\vx + 10) \\
  & \ind \False \to \mit{joinNothing}\,()
\end{alignat*}
Such fused returns are possible whenever we have a $\Gen\,\vA$ action at the
bottom of the transformer stack, such that $\vA$ is isomorphic to a meta-level
finite sum of value types. Recall that $\Gen\,\vA$ is defined as $\{\vR : \VTy\} \to
(\vA \to \Up \vR) \to \Up \vR$. Here, if $\vA$ is a finite sum, we can rearrange $\vA \to
\Up \vR$ to a finite product of functions.

We could proceed with finite sums, but we will need finite
\emph{sums-of-products} ($\SOP$ in short) later in Section in \ref{sec:stream-fusion}, so we
develop SOP-s. We view $\SOP$-s as a Tarski-style universe consisting of a type of
descriptions and a way of interpreting descriptions into $\MTy$ (``$\El$'' for
``elements'' of the type).
\begin{alignat*}{3}
  &\USOP : \MTy                    &&\ind \El_\SOP : \USOP \to \MTy \\
  &\USOP = \List\,(\List\,\VTy)    &&\ind \El_\SOP\,\vA = ...
\end{alignat*}
A type description is a list of lists of value types. We decode this to a sum of
products of value types. We omit the definition here. $\USOP$ is closed under
value types, finite product types and finite sum types. For instance, we have
$\Either_\SOP : \USOP \to \USOP \to \USOP$ together with $\Left_\SOP : \El_\SOP\,\vA
\to \El_\SOP\,(\Either_\SOP\,\vA\,\vB)$, $\Right_\SOP : \El_\SOP\,\vB \to
\El_\SOP\,(\Either_\SOP\,\vA\,\vB)$ and a case splitting operation. It is more
convenient to work with type formers in $\MTy$, and only convert to $\SOP$
representations when needed, so we define a class for the representable types:
\begin{alignat*}{3}
  & \hspace{-5em}\mathrlap{\tyclass\,\IsSOP\,(\vA : \MTy)\,\where} \\
  & \hspace{-5em}\ind \Rep      &&: \USOP \\
  & \hspace{-5em}\ind \mit{rep} &&: \vA \simeq \El_\SOP\,\Rep
\end{alignat*}
Above, $\vA \simeq \Rep$ denotes a record containing a pair of back-and-forth
functions, together with proofs (as propositional equalities) that they are
inverses. We will overload $\mit{rep}$ as the forward conversion function with
type $\vA \to \El_\SOP\,\Rep$, and write $\mit{rep}^{-1}$ for its inverse. Now we define
an isomorphic presentation of $\El_\SOP\,\vA \to \Lift \vR$ as a product of
object-level functions:
\begin{alignat*}{3}
  &\mathrlap{\mit{Fun}_{\SOP\Lift} : \USOP \to \Ty \to \MTy}\\
  &\mit{Fun}_{\SOP\Lift}\,\Nil\,&& \vR = ()\\
  &\mit{Fun}_{\SOP\Lift}\,(\Cons\,\vA\,\vB)\,&& \vR = (\Lift(\mit{foldr}\,(\to)\,\vR\,\vA),\,\mit{Fun}_{\SOP\Lift}\,\vB\,\vR)
\end{alignat*}
\begin{alignat*}{3}
  &\mtabulate &&: (\El_\SOP\,\vA \to \Lift \vR) \to \mit{Fun}_{\SOP\Lift}\,\vA\,\vR\\
  &\mindex    &&: \mit{Fun}_{\SOP\Lift}\,\vA\,\vR \to (\El_\SOP\,\vA \to \Lift \vR)
\end{alignat*}
We omit here the definitions of $\mtabulate$ and $\mindex$. We will also need to
let-bind all functions in a $\mit{Fun}_{\SOP\Lift}$:
\begin{alignat*}{3}
  & \mathrlap{\mit{genFun}_{\SOP\Lift} : \{\vA : \USOP\} \to \mit{Fun}_{\SOP\Lift}\,\vA\,\vR \to \mit{Fun}_{\SOP\Lift}\,\vA\,\vR}\\
  & \mit{genFun}_{\SOP\Lift}\,\{\Nil\}        && ()             &&= \return\,()\\
  & \mit{genFun}_{\SOP\Lift}\,\{\Cons\,\_\,\vA\}&& (\vf,\,\mit{\vfs}) &&= ({,})\,{\fmap}\,\gen\,\vf\,{\ap}\,\mit{genFun}_{\SOP\Lift}\,\{\vA\}\,\vfs
\end{alignat*}
We introduce a class for monads that support control flow joining.
\begin{alignat*}{3}
  & \tyclass\,\Monad\,\vM \RA \MonadJoin\,\vM\,\where\\
  & \ind \join : \IsSOP\,\vA \RA \vM\,\vA \to \vM\,\vA
\end{alignat*}
The most interesting instance is the ``base case'' for $\Gen$:
\begin{alignat*}{3}
  &\instance \MonadJoin\,\Gen\,\where\\
  &\ind \join\,\vma = \Gen \dlr \lam \vk.\,\runGen \dlr \mdo\\
  &\ind \ind \mit{joinPoints} \fro \mit{genFun}_{\SOP\Lift}\,(\mtabulate\,(\vk \circ \mit{rep}^{-1}))\\
  &\ind \ind \va \fro \vma\\
  &\ind \ind \return \dlr \mindex\,\mit{joinPoints}\,(\mit{rep}\,\va)
\end{alignat*}
Here we first convert $\vk : \vA \to \Lift \vR$ to a product of join points and
let-bind each one of them. Then we generate code that returns to the appropriate
join point for each return value. Other $\MonadJoin$ instances are straightforward. In
$\StateT_\M$, we need the extra $\IsSOP\,\vS$ constraint because $\vS$ is returned
as a result value.
\begin{alignat*}{3}
  & \instance\,(\MonadJoin\,\vM) \RA \MonadJoin\,(\MaybeT_\M\,\vM)\,\where\\
  & \ind \join\,(\MaybeT_\M\,\vma) = \MaybeT_\M\,(\join\,\vma)\\
  & \instance\,(\MonadJoin\,\vM) \RA \MonadJoin\,(\ReaderT_\M\,\vR\,\vM)\,\where\\
  & \ind \join\,(\ReaderT_\M\,\vma) = \ReaderT_\M\,(\join \circ \vma)\\
  & \instance\,(\MonadJoin\,\vM,\,\IsSOP\,\vS) \RA \MonadJoin\,(\StateT_\M\,\vS\,\vM)\,\where\\
  & \ind \join\,(\StateT_\M\,\vma) = \StateT_\M\,(\join \circ \vma)
\end{alignat*}

Now, whenever the return value of a $\mbf{case}$-splitting action is a sum of
products of values (this includes just returning an object value, which is by
far the most common situation), we can use $\join \dlr \case \vx\,\of ...$ to
eliminate code duplication, without creating runtime sum types.

\subsection{Code Example}

We look at a slightly larger code example. We define annotated binary trees as
\[ \data \Tree\,\vA := \Leaf\,|\,\Node\,\vA\,(\Tree\,\vA)\,(\Tree\,\vA). \]
We write $\msf{fail : M\,A}$ for returning $\Nothing$ in any monad transformer
stack that contains $\MaybeT$. We define a function which traverses a tree and
replaces values with values taken from a list. If the tree contains $0$, we
throw an error. If we run out of list elements, we leave values unchanged.
\begin{alignat*}{4}
  &  \letrec &&\vf : \Tree\,\Nat \to \StateT\,(\List\,\Nat)\,(\MaybeT\,\Identity)\,(\Tree\,\Nat) \\
  &          &&\vf\,\vt := \spl \big(\mdown \dlr \case \vt\,\of\\
  &          &&\ind \Leaf \to \return \qt{\Leaf} \\
  &          &&\ind \Node\,\vn\,\vl\,\vr \to \mdo\\
  &          &&\ind \ind \case \qt{\spl \vn == 0}\,\of\,\True \hspace{0.2em}\to \msf{fail}\\
  &          &&\ind \ind \ind \hspace{6.7em}             \False \to \return () \\
  &          &&\ind \ind \vns \fro \get \\
  &          &&\ind \ind \vn\hspace{0.35em}\fro\join \dlr \case \vns\,\of\,\Nil \hspace{2.65em}\to \return \vn \\
  &          &&\ind \ind \ind \hspace{8.2em}\Cons\,\vn\,\vns \to \mdo \{\msf{put}\,\vns; \return \vn\} \\
  &          &&\ind \ind \vl\hspace{0.1em} \fro \mup\,\qt{\vf\,\spl \vl} \\
  &          &&\ind \ind \vr \fro \mup\,\qt{\vf\,\spl \vr} \\
  &          &&\ind \ind \return \qt{\Node\,\spl \vn\,\spl \vl\,\spl \vr}\big)
\end{alignat*}
Here, all of the $\case$ matches are done on object-level values, so they are
all desugared to applications of $\msf{split}$.
\begin{itemize}
\item Note the $\join$: without it, the recursive $\Node$ traversal code would be inlined in both case branches.
\item We do not use $\join$ when checking $\qt{\spl n == 0}$. Here, it
      happens to be superfluous, since the $\msf{fail}$ ``destroys'' all subsequent
      code generation in the branch, by short-circuiting at the meta-level.
\item Also, we can use $\msf{put}$ instead of the ``stricter'' $\msf{put'}$ because
      the state modification immediately gets sequenced by the enclosing join point.
\item To make object-level recursive calls, we just need to wrap them in $\mup$.
\end{itemize}
Here, we intentionally tuned the definition for a nice-looking staging output.
However, we could also just default to ``safe'' choices everywhere: we could use
a joint point in every $\mbf{case}$ with two or more branches, and use the
strict state modifications everywhere. This would only add a small amount of
administrative noise that can be easily cleaned up by the downstream compiler.

Let us look at the unstaging output. We omit newtype wrappers and use nested
pattern matching on pairs, but otherwise this is exactly the code that we get in
our Agda implementation.
\begin{alignat*}{4}
  &  \letrec &&\vf : \Tree\,\Nat \to \StateT\,(\List\,\Nat)\,(\MaybeT\,\Identity)\,(\Tree\,\Nat) \\
  &          &&\vf := \lam \vt\,\vns.\, \case\,\vt\,\of\\
  &          &&\ind \Leaf \to \Just\,(\Leaf,\,\vns) \\
  &          &&\ind \Node\,\vn\,\vl\,\vr \to \case (\vn == 0)\,\of\\
  &          &&\ind \ind \True \hspace{0.25em}\to \Nothing \\
  &          &&\ind \ind \False \to \\
  &          &&\ind \ind \ind \letdef \msf{joinNothing} := \lam \_.\,\Nothing;\\
  &          &&\ind \ind \ind \letdef \msf{joinJust} := \lam \vn\,\vns.\,\case\,\vf\,\vl\,\vns\,\of\\
  &          &&\ind \ind \ind \ind \ind \ind \Nothing\hspace{0.9em} \to \Nothing\\
  &          &&\ind \ind \ind \ind \ind \ind \Just\,(\vl,\,\vns) \to \case\,\vf\,\vr\,\vns\,\of\\
  &          &&\ind \ind \ind \ind \ind \ind \ind \Nothing\hspace{.95em} \to \Nothing\\
  &          &&\ind \ind \ind \ind \ind \ind \ind \Just\,(\vr,\,\vns) \to \Just\,(\Node\,\vn\,\vl\,\vr,\,\vns);\\
  &          &&\ind \ind \ind \case \vns\,\of\\
  &          &&\ind \ind \ind \ind \Nil \hspace{2.65em}\to \msf{joinJust}\,\vn\,\vns\\
  &          &&\ind \ind \ind \ind \Cons\,\vn\,\vns \to \msf{joinJust}\,\vn\,\vns
\end{alignat*}
The only flaw in this code is the $\msf{joinNothing}$, which is never called,
becase we joined an action that never throws an error. But this is also easy to
clean up afterwards.
\subsection{Discussion}

So far, we have a monad transformer library with the following features:
\begin{itemize}
\item Almost all definitions from the well-known Haskell ecosystem of monads and monad transformers
      can be directly reused, in the meta level.
\item We can pattern match on object-level values in monadic code, insert object-level $\mbf{let}$-s
      with $\gen$ and avoid code duplication with $\join$.
\item In monadic code, object-level data constructors are only ever created by
      $\mdown$, and matching on object-level data is only created by $\mit{split}$
      and $\mup$. Monadic operations are fully fused, and all function calls can be
      compiled to statically known saturated calls.
\end{itemize}

As to potential weaknesses, first, the system as described in this section has
some syntactic noise and requires extra attention from programmers.  We believe
that the noise can be mitigated very effectively in a native CFTT
implementation. It was demonstrated in a 2LTT implementation in
\cite{staged-demo} that almost all quotes and splices are unambiguously
inferable, if we require that stages of let-definitions are always specified (as
we do here). Moreover, $\mup$ and $\mdown$ should be also effectively inferable,
using bidirectional elaboration. With such inference, monadic code in CFTT would
look only slightly more complicated than in Haskell.

Second, in CFTT we cannot store computations (e.g.\ functions or $\State$
actions) in runtime data structures, nor can we have computations in $\State$
state or in $\Reader$ environments. However, it would be possible to extend CFTT
with a closure type former that converts computations to values, in which case
there is no such limitation anymore. Here, closure-freedom would be still
available; we would be able to pick where to use or avoid the closure type
former.

\subsection{Agda \& Haskell Implementations}

We implemented everything in this section in both Agda and typed Template
Haskell. We summarize features and differences:
\begin{itemize}
\item The Haskell implementation can be used to generate code that can be
      further compiled by GHC; here the object language is taken to be Haskell
      itself. Since Haskell does not distinguish value and computation types, we do
      not track them in the library, and we do not get guaranteed closure-freedom
      from GHC.
\item In Agda, we postulate all types and terms of the object theory in a
  faithful way (i.e.\ equivalently to the CFTT syntax presented here), and take
  Agda itself to be the metalanguage. Here, we can test ``staging'' by running Agda
  programs which compute object expressions. However, we can only inspect
  staging output and cannot compile or run object programs.
\item For sums-of-products in Haskell, we make heavy use of \emph{singleton
  types} \cite{DBLP:conf/haskell/EisenbergW12} to emulate dependent types. This adds significant
  noise. Also, in $\IsSOP$ instances we can only define the conversion functions
  and cannot prove that they are inverses, because Haskell does not have enough support
  for dependent types.
\end{itemize}

\section{Stream Fusion}\label{sec:stream-fusion}

Stream fusion refers to a collection of techniques for generating efficient code
from declarative definitions involving streams of values, where intermediate
data structures are eliminated. Stream fusion can be broadly grouped into
\emph{push} fusion, which is based on Church-encodings of inductive lists, and
$\emph{pull}$ fusion, which is based on Church-encodings of coinductive lists;
see e.g.\ \cite{DBLP:conf/ifl/HinzeHJ10}. The two styles have different
trade-offs, and in practical programming it is a good idea to support both, but
in this section we focus on pull streams.

The reason is that pull streams have been historically more difficult to
efficiently compile, and we can demonstrate significant improvements in CFTT.
We also use dependent types in a more essential way than in the previous
section.


\subsection{Streams}

A pull stream is a meta-level specification of a state machine:
\begin{alignat*}{3}
  & \data \Step\,\vS\,\vA = \Stop\,|\,\Skip\,\vS\,|\,\Yield\,\vA\,\vS\\
  & \data \Pull\,(\vA : \MTy) : \MTy\,\where\\
  & \ind \Pull : (\vS : \MTy) \to \IsSOP\,\vS \RA \Gen\,\vS \to (\vS \to \Gen\,(\Step\,\vS\,\vA)) \to \Pull\,\vA
\end{alignat*}
In the $\Pull$ constructor, $\vS$ is the type of the internal state which is
required to be a sum-of-products of value types by the $\IsSOP\,\vS$
constraint. Borrowing terminology from Coutts \cite{DBLP:phd/ethos/Coutts11}, we
call each product in the state a \emph{state shape}.

The next field with type $\Gen\,\vS$ is the initial state, while transitions
are represented by the $\vS \to \Gen\,(\Step\,\vS\,\vA)$ field. Possible transitions
are stopping ($\Stop$), transitioning to a new state while outputting a value
($\Yield$) and making a silent transition to a new state ($\Skip$).

Let us see some operations on streams now. $\Pull$ is evidently a $\Functor$. It
is not quite a ``zippy'' applicative functor, because its application operator
requires an extra $\IsSOP$ constraint:
\begin{alignat*}{3}
  &\mit{repeat} : \vA \to \Pull\,\vA\\
  &\mit{repeat}\,\va = \Pull\,()\,(\return\,())\,(\lam \_.\,\return \dlr \Yield\,\va\,())
\end{alignat*}
\begin{alignat*}{3}
  &\hspace{-7em}\mathrlap{({\ap_{\Pull}}) : \IsSOP\,\vA \RA \Pull\,(\vA \to \vB) \to \Pull\,\vA \to \Pull\,\vB }\\
  &\hspace{-7em}\mathrlap{({\ap_{\Pull}})\,(\Pull\,\vS\,\seed\,\step)\,(\Pull\,\vS'\,\seed'\,\step') =}\\
  &\hspace{-7em}\mathrlap{ \ind \Pull\,(\vS,\,\vS',\,\Maybe\,\vA)\,(({,}) \fmap \seed \ap ((,\Nothing) \fmap \seed') \dlr \lam \case }\\
  &\hspace{-7em}\mathrlap{ \ind \ind (\vs,\,\vs',\,\Just\,\va) \to \step\,\vs \bind \lam \case}\\
  &\hspace{-7em} \ind \ind \ind \Stop        &&\to \return \Stop \\
  &\hspace{-7em} \ind \ind \ind \Skip\,\vs     &&\to \return \dlr \Skip\,(\vs,\,\vs',\,\Just\,\va) \\
  &\hspace{-7em} \ind \ind \ind \Yield\,\vf\,\vs &&\to \return \dlr \Yield\,(\vf\,\va)\,(\vs,\,\vs',\,\Nothing) \\
  &\hspace{-7em} \mathrlap{\ind \ind (\vs,\,\vs',\,\Nothing) \to \step'\,\vs' \bind \lam \case }\\
  &\hspace{-7em} \ind \ind \ind \Stop         &&\to \return \Stop \\
  &\hspace{-7em} \ind \ind \ind \Skip\,\vs'     &&\to \return \dlr \Skip\,(\vs,\,\vs',\,\Nothing) \\
  &\hspace{-7em} \ind \ind \ind \Yield\,\va\,\vs' &&\to \return \dlr \Skip\,(\vs,\,\vs',\,\Just\,\va)
\end{alignat*}

In $\mit{repeat}$, the state is the unit type, while in $({\appull})$ we take
the product of states, and also add another $\Maybe\,\vA$ that is used to buffer
a single value while we are stepping the $\Pull\,(\vA \to \vB)$ stream.
The $\IsSOP$ constraints for the new machine states are implicitly dispatched
by instance resolution; this works in the Agda and Haskell versions too. We can
derive $\mit{zip}$ and $\mit{zipWith}$ from $({\appull})$ and $({\fmap})$ for
streams, with the restriction that the zipped streams must all produce $\IsSOP$
values.

$\Pull$ is also a monoid with stream appending as the binary operation.
\begin{alignat*}{3}
  & \mit{empty} : \Pull\,\vA\\
  & \mit{empty} = \Pull\,()\,(\return\,())\,(\lam \_.\,\return \Stop)
\end{alignat*}
\begin{alignat*}{5}
  & \hspace{-5em}\mathrlap{({<>}) : \Pull\,\vA \to \Pull\,\vA \to \Pull\,\vA} \\
  & \hspace{-5em}\mathrlap{({<>})\,(\Pull\,\vS\,\seed\,\step)\,(\Pull\,\vS'\,\seed'\,\step') = \Pull\,(\Either\,\vS\,\vS')\,(\Left \fmap \seed)\,\lam \case} \\
  & \hspace{-5em}\ind \Left\,\vs &&\to \step\,s \bind \lam \case &&\Stop             &&\to (\Skip \circ \Right) \fmap \seed'\\
  & \hspace{-5em}                    &&                          &&\Skip\,\vs        &&\to \return \dlr \Skip\,(\Left\,\vs) \\
  & \hspace{-5em}                    &&                        &&\Yield\,\va\,\vs    &&\to \return \dlr \Yield\,\va\,(\Left\,\vs)\\
  & \hspace{-5em}\ind \Right\,\vs' &&\to \step\,s' \bind \lam \case &&\Stop          &&\to \return \Stop\\
  & \hspace{-5em}                    &&                        &&\Skip\,\vs'         &&\to \return \dlr \Skip\,(\Right\,\vs') \\
  & \hspace{-5em}                    &&                        &&\Yield\,\va\,\vs'   &&\to \return \dlr \Yield\,\va\,(\Right\,\vs')
\end{alignat*}
These definitions are standard for streams; compared to the non-staged
definitions in previous literature, the only additional noise is just the $\Gen$
monad in the initial states and the transitions. Likewise, we can give standard
definitions for usual stream functions such as $\msf{filter}$, $\msf{take}$ or
$\msf{drop}$.

\subsection{Running Streams with Mutual Recursion}

How do we generate object code from streams? The state $\vS$ is given as a
finite sums of products, but the sums and the products are on the meta level, so
we cannot directly use $\vS$ in object code. Similarly as in the the treatment
of join points, we tabulate the $\vS \to \Gen\,\Step\,\vS\,\vA$ transition
function to a product of functions. However, these functions need to be
\emph{mutually recursive}, since it is possible to transition from any state to
any other state, and each such transition is represented as a function call.
This problem of generating well-typed mutual blocks was addressed by Yallop and
Kiselyov in \cite{DBLP:conf/pepm/YallopK19}. In contrast to this work, which
used control effects and mutable references in MetaOCaml, we present a solution
that does not use side effects in the metalanguage.

The solution is to extend $\CTy$ with finite products of computations,
i.e.\ assume $() : \CTy$ and $({,}) : \CTy \to \CTy \to \CTy$, together with
pairing and projections. These types, like functions, are call-by-name in the
runtime semantics, and they also cannot escape the scope of their definition.
Hence, we can also ``saturate'' programs that involve computational product types:
every computation definition at type $(\vA,\,\vB)$ can be translated to a pair,
and every projection of a let-defined variable can be statically matched up with
a pairing in the variable definition. Thus, a recursive let-definition at type
$(\vA \to \vB,\,\vA \to \vB)$ can be always compiled to a pair of mutual
functions.

We redefine the previous $\mit{Fun}_{\SOP\Lift}$ to return a computation type
instead:
\begin{alignat*}{3}
  &\mathrlap{\mit{Fun}_{\SOP\Lift} : \USOP \to \Ty \to \CTy}\\
  &\mit{Fun}_{\SOP\Lift}\,\Nil\,&& \vR = ()\\
  &\mit{Fun}_{\SOP\Lift}\,(\Cons\,\vA\,\vB)\,&& \vR = (\mit{foldr}\,(\to)\,\vR\,\vA,\,\mit{Fun}_{\SOP\Lift}\,\vB\,\vR)
\end{alignat*}
\begin{alignat*}{3}
  &\mtabulate &&: (\El_\SOP\,\vA \to \Lift \vR) \to \Lift(\mit{Fun}_{\SOP\Lift}\,\vA\,\vR)\\
  &\mindex    &&: \Lift(\mit{Fun}_{\SOP\Lift}\,\vA\,\vR) \to (\El_\SOP\,\vA \to \Lift \vR)
\end{alignat*}
Even for join points, this is just as efficient as the previous
$\mit{Fun}_{\SOP\Lift}$ version, since a definition of a product of functions
will get compiled to a sequence of function definitions. In addition, we can use
it to generate object code from streams. There are several choices for this, but
in CFTT the $\msf{foldr}$ function is as good as we can get.
\begin{alignat*}{3}
  &\hspace{-13em}\mathrlap{\foldr : \{\vA : \MTy\}\{\vB : \Ty\} \to (\vA \to \Lift \vB \to \Lift \vB) \to \Lift \vB \to \Pull\,\vA \to \Lift\,\vB}\\
  &\hspace{-13em}\mathrlap{\foldr\,\{\vA\}\,\{\vB\}\,\vf\,\vb\,(\Pull\,\vS\,\seed\,\step) = \ql}\\
  &\hspace{-13em}\mathrlap{\ind \letrec \vfs : \mit{Fun}_{\SOP\Lift}\,(\Rep\,\{\vS\})\,\vB := \spl\big(\mtabulate \dlr \lam \vs.\,\unGen\,(\step\,(\rep^{-1}\,\vs)) \dlr \lam \case} \\
  &\hspace{-13em}\ind \ind \ind \Stop            &&\to \vb \\
  &\hspace{-13em}\ind \ind \ind \Skip\,\vs       &&\to \mindex\,\qt{\vfs}\,(\rep\,\vs) \\
  &\hspace{-13em}\ind \ind \ind \Yield\,\va\,\vs &&\to \vf\,\va\,(\mindex\,\qt{\vfs}\,(\rep\,\vs))\big);\\
  &\hspace{-13em}\mathrlap{\ind \spl\big(\unGen\,\seed \dlr \lam s.\,\mindex\,\qt{\vfs}\,(\rep\,\vs)\big)\qr}
\end{alignat*}
This $\foldr$ is quite flexible because it can produce object terms with computation types.
For instance, we can define $\foldl$ from $\foldr$:
\begin{alignat*}{3}
  &\foldl : \{\vA : \MTy\}\{\vB : \VTy\} \to (\Lift \vB \to \vA \to \Lift \vB) \to \Lift \vB \to \Pull\,\vA \to \Lift\,\vB\\
  &\foldl\,\vf\,\vb\,\vas = \spl(\foldr\,(\lam \va\,\vg.\,\qt{\lam \vb.\,\spl \vg\,\spl(\vf\,\qt{\vb}\,\spl \va)})\,\qt{\lam \vb.\,\vb}\,\vas)\,\spl \vb
\end{alignat*}
Note that since we abstract $\vB$ in a runtime function, it must be a value
type. Here, each $\Stop$ and $\Yield$ in the transition function gets
interpreted as a $\lambda$-expression in the output. However, those $\lambda$-s
will be lifted out in the scope, yielding a proper mutually tail-recursive
definition with an accummulator for $\vB$. Contrast this to the GHC base
library, where $\foldl$ for lists is also defined from $\foldr$, to enable
push-based fusion, but where a substantial \emph{arity analysis} is used in
the compiler to (incompletely) eliminate the intermediate closures
\cite{DBLP:conf/sfp/Breitner14}.

\subsection{$\mbf{concatMap}$ for Streams}\label{sec:concatmap-for-streams}

Can we have a list-like $\Monad$ instance for streams, with singleton streams
for $\return$ and $\concatMap$ for binding? This is not possible. Aiming at
\[ \concatMap : (\vA \to \Pull\,\vB) \to \Pull\,\vA \to \Pull\,\vB, \]
the $\vA \to \Pull\,\vB$ function can contain an infinite number of different
machine state types, which cannot be represented in a finite amount of object
code. Here by ``infinite'' we mean the notion that is internally available in
the meta type theory. For instance, we can define a $\Nat_\M \to \Pull\,\vB$
function which for each $\vn : \Nat_\M$ produces a concatenation of $\vn$
streams. Hence, we shall have the following function instead:
\[ \concatMap : \IsSOP\,\vA \RA (\vA \to \Pull\,\vB) \to \Pull\,\vA \to \Pull\,\vB \]
The idea is the following: if $\USOP$ is closed under $\Sigma$-types, we can directly
define this $\concatMap$, by taking the appropriate dependent sum of the $\vA
\to \USOP$ family of machine states, which we extract from the $\vA \to
\Pull\,\vB$ function. Let us write $\Sigma\,\vA\,\vB : \MTy$ for dependent sums,
for $\vA : \MTy$ and $\vB : \vA \to \MTy$, and reuse $({,})$ for pairing.  We
also use the following field projection functions: $\mit{projS} : \Pull\,\vA \to
\MTy$, $\mit{projSeed} : (\vas : \Pull\,\vA) \to \mit{projS}\,\vas$, and
$\mit{projStep} : (\vas : \Pull\,\vA) \to \mit{projS}\,\vas \to
\Gen\,(\Step\,(\mit{projS}\,\vas)\,\vA)$.  For now, we assume that the following
instance exists:
\[ \instance (\IsSOP\,\vA,\,\{\va : \vA\} \to \IsSOP\,(\vB\,\va)) \RA \IsSOP\,(\Sigma\,\vA\,\vB) \]
Above, $\{\va : \vA\} \to \IsSOP\,(\vB\,\va)$ is a universally quantified
instance constraint; this can be also represented in Agda. The definition of
$\concatMap$ is as follows.
\begin{alignat*}{3}
  & \mathrlap{\concatMap : \IsSOP\,\vA \RA (\vA \to \Pull\,\vB) \to \Pull\,\vA \to \Pull\,\vB }\\
  & \mathrlap{\concatMap\,\{\vA\}\,\{\vB\}\,\vf\,(\Pull\,\vS\,\seed\,\step) = }\\
  & \mathrlap{\ind \Pull\,(\vS,\,\Maybe\,(\Sigma\,\vA\,(\mit{projS} \circ \vf)))\,(({,}) \fmap \seed\,\ap\,\return\,\Nothing) \dlr \lam \case }\\
  & \mathrlap{\ind \ind (\vs,\,\Nothing) \to \step\,\vs \bind \lam \case }\\
  & \ind \ind \ind \ind \Stop                 &&\to \return \Stop\\
  & \ind \ind \ind \ind \Skip\,\vs            &&\to \return \dlr \Skip\,(\vs,\, \Nothing)\\
  & \ind \ind \ind \ind \Yield\,\va\,\vs      &&\to \mdo \{\vs' \fro \mit{projSeed}\,(\vf\,\va); \return \dlr \Skip\,(\vs,\,\Just\,(\va,\,\vs'))\}\\
  & \mathrlap{\ind \ind (\vs,\,\Just\,(\va,\,\vs')) \to \mit{projStep}\,(\vf\,\va)\,\vs' \bind \lam \case }\\
  & \ind \ind \ind \ind \Stop                &&\to \return \dlr \Skip\,(\vs,\,\Nothing) \\
  & \ind \ind \ind \ind \Skip\,\vs'          &&\to \return \dlr \Skip\,(\vs,\,\Just\,(\va,\,\vs'))\\
  & \ind \ind \ind \ind \Yield\,\vb\,\vs'    &&\to \return \dlr \Yield\,\vb\,(\vs,\, \Just\,(\va,\,\vs'))
\end{alignat*}
Here, $\Nothing$ marks the states where we are in the ``outer'' loop, running
the $\Pull\,\vA$ stream until we get its next value. $\Just$ marks the states of
the ``inner'' loop, where we have a concrete $\va : \vA$ value and we run the
$(\vf\,\va)$ stream until it stops. In the inner loop, the machine state type
depends on the $\va : \vA$ value, hence the need for $\Sigma$. How do we get
$\IsSOP$ for $\Sigma$? The key observations are:
\begin{itemize}
 \item Metaprograms cannot inspect the structure of object terms.
 \item Object types do not depend on object terms.
\end{itemize}
Hence, we expect that during staging, every $\vf : \Lift\,\vA \to \VTy$ function
has to be constant. This is actually true in the staging semantics of
CFTT. There, all metafunctions are stable under object-level parallel
substitutions. Also, object types are untouched by substitution. Hence, a
straightforward unfolding of semantic definitions validates the constancy of
$\vf$.

Generally speaking, in the semantics, every function whose domain is a product
of object types and whose codomain is a constant presheaf, is a constant
function. We may call these constancy statements \emph{generativity axioms},
since they reflect the inability of metaprograms to inspect terms, and
``generativity'' often refers to this property in the staged compilation
literature.

Let us write $\Uprod = \List\,\VTy$ and $\Elprod : \Uprod \to \MTy$ for a
universe of finite products of value types. Concretely, in CFTT and the Agda
implementation, we assume the following:
\\\\
\noindent \textbf{Axiom (generativity).}\textit{ For every $\vf : \Elprod\,\vA \to \USOP$ and $\va,\,\va' : \Elprod\,\vA$,
we have that $\vf\,\va = \vf\,\va'$.}
\\\\
We remark that there is no risk of staging getting stuck on this axiom, because
propositional equality proofs get erased, as we described in Section
\ref{sec:semantics-of-staging}.

%% We also remark that although generativity is inconsistent with inspecting the
%% structure of object terms, it is consistent with inspecting the structure of
%% object \emph{types}.

From generativity, we derive $\Sigma_\SOP : (\vA : \USOP) \to (\vB : \El_\SOP\,\vA \to \USOP)
\to \USOP$ as follows. First, for each $\vA : \Uprod$, we define $\msf{loop_{\vA}} :
\Elprod\,\vA$ as a product of non-terminating object programs. This is only needed
to get arbitrary inhabitants with which we can call $\Elprod\,A \to \USOP$
functions.

Then, $\Sigma_\SOP\,\vA\,\vB$ is defined as the concatenation of $\vA_i \times
\vB\,(\msf{inject_i}\,\msf{loop_{\vA_i}})$ for each $\vA_i \in \vA$, where
$({\times})$ is the product type former in $\USOP$ and $\msf{inject_i}$
has type $\Elprod\,\vA_i \to \El_\SOP\,\vA$. This is similar to the definition of
non-dependent products in $\USOP$, except that we have to get rid of the type
dependency by instantiating $\vB$ with arbitrary programs.

Then, we can show using the generativity axiom that $\Sigma_\SOP$ supports
projections, pairing and the $\beta\eta$-rules. These are all needed when we
define the $\IsSOP$ instance, when we have to prove that encoding via $\rep$
is an isomorphism. Concretely, assuming $\IsSOP\,\vA$ and $\{\va : \vA\} \to
\IsSOP\,(\vB\,\va)$, we define $\Rep$ and $\msf{rep}$ for $\Sigma\,\vA\,\vB$ as follows:
\[ \Rep\,\{\Sigma\,\vA\,\vB\} = \Sigma\,(\Rep\,\{\vA\})\,(\lam \vx.\,\Rep\,\{\vB\,(\rep^{-1}\,\vx)\}) \]
\[  \rep\,\{\Sigma\,\vA\,\vB\}\,(\va,\,\vb) = (\rep\,\{\vA\}\,\va,\,\rep\,\{\vB\,(\rep^{-1}\,(\rep\,\vx))\}\,\vb) \]
Note that $\rep$ is only well-typed up to the fact that $\rep^{-1} \circ \rep =
\id$; the Agda definition contains an additional transport that we omit here.
This is, in fact, our reason for including the isomorphism equations in
$\IsSOP$. This, in turn, necessitates using SOP instead of finite sums. We can
define a product type former for finite sums of value types, by taking the
pairwise products of components. However, we can only take the
\emph{object-level} products here, and since object-level products have no
$\beta\eta$-rules, we cannot prove $\beta\eta$ for the derived product type in
finite sums, and likewise for the derived $\Sigma$-type. When we use SOP
instead, we do not have to use object-level products and this issue does not
appear.

We skip the full definition of $\Sigma_\SOP$ here. The reader may refer
to the $\SOP$ module in the Agda implementation, which is altogether 260 lines with
all $\IsSOP$ instances.

\subsection{Let-Insertion \& Case Splitting in Monadic Style}

While $\Pull$ is not a monad, and hence also not a $\MonadGen$, we can still use
a monadic style of stream programming with good ergonomics. First, we need
singleton streams for ``returning'':
\begin{alignat*}{3}
  & \mit{single} : \vA \to \Pull\,\vA \\
  & \mit{single}\,\va = \Pull\,\Bool_\M\,\True \dlr \lam \vb.\,\return \dlr\mbf{if}\,\vb\,\mbf{then}\,\Yield\,\va\,\False\,\mbf{else}\,\Stop
\end{alignat*}
Now, we should use this operation with care, since it has two states and can
contribute to a code size blow-up. For example, $\concatMap$ over $\single$
introduces at least a doubling of the number of state shapes. Although
let-insertion and case splitting could be derived as $\concatMap$ over
$\single$, to avoid the size explosion we give more specialized definitions
instead. First, we define an embedding of $\Gen$ operations in streams.
\begin{alignat*}{3}
  & \hspace{-11em}\mathrlap{\mit{bind_{\Gen}} : \IsSOP\,\vA \RA \Gen\,\vA \to (\vA \to \Pull\,\vB) \to \Pull\,\vB}\\
  & \hspace{-11em}\mathrlap{\mit{bind_{\Gen}}\,\vga\,\vf\,=}\\
  & \hspace{-11em}\mathrlap{\ind \Pull\,(\Sigma\,\vA\,(\mit{projS} \circ \vf))\,(\mdo \{\va \fro \vga; \vs \fro \mit{projSeed}\,(\vf\,\va); \return (\va,\,\vs)) \dlr \lam \case}\\
  & \hspace{-11em}\mathrlap{\ind \ind (\va,\,\vs) \to \mit{projStep}\,(\vf\,\va)\,\vs \bind \lam \case }\\
  & \hspace{-11em}\ind \ind \ind \Stop            &&\to \return \Stop\\
  & \hspace{-11em}\ind \ind \ind \Skip\,\vs       &&\to \return \dlr \Skip\,(\va,\,\vs) \\
  & \hspace{-11em}\ind \ind \ind \Yield\,\vb\,\vs &&\to \return \dlr \Yield\,\vb\,(\va,\,\vs)
\end{alignat*}
We recover let-insertion and case splitting as follows:
\begin{alignat*}{3}
  &\mit{gen_\Pull} : \Lift \vA \to (\Lift \vA \to \Pull\,\vB) \to \Pull\,\vB\\
  &\mit{gen_\Pull}\,\va = \mit{bind_{\Gen}}\,(\gen\,\va)
\end{alignat*}
\begin{alignat*}{3}
  &\mit{case_\Pull} : (\mit{Split}\,\vA,\,\IsSOP\,(\mit{SplitTo}\,\{\vA\})) \RA \Lift \vA \to (\mit{SplitTo}\,\{\vA\} \to \Pull\,\vB) \to \Pull\,\vB\\
  &\mit{case_\Pull}\,\va = \mit{bind_{\Gen}}\,(\mit{split}\,\va)
\end{alignat*}
Let us look at small example. We write $\mit{forEach}$ for $\concatMap$ with
flipped arguments.
\begin{alignat*}{3}
  &\hspace{-4em}\mathrlap{\mit{forEach}\,(\mit{take}\,\qt{100}\,(\mit{countFrom}\,\qt{0})) \dlr \lam \vx.}\\
  &\hspace{-4em}\mathrlap{\gen_\Pull\, \qt{\spl \vx * 2} \dlr \lam \vy.}\\
  &\hspace{-4em}\mathrlap{\mit{case_\Pull}\,\qt{\spl \vx < 50} \dlr \lam \case}\\
  &\hspace{-4em} \ind \True  &&\to \mit{take}\,\vy\,(\mit{countFrom}\,x)\\
  &\hspace{-4em} \ind \False &&\to \single\,\vy
\end{alignat*}
Here, in every $\forEach$ iteration, $\gen_\Pull\,\qt{\spl \vx * 2}$ evaluates
the given expression and saves the result to the machine state. Then,
$\mit{case_\Pull}$ branches on an object-level Boolean expression. If we use
$\foldr$ to generate object code from this definition, we get \emph{four}
mutually defined functions. We get two state shapes from $\single\,\vy$ and one
from $\mit{take}\,\vy\,(\mit{countFrom}\,x)$; we sum these to get three for the
$\mit{case_\Pull}$, then finally we get an extra shape from $\mit{forEach}$ which
introduces an additional $\Maybe$.

\subsection{Discussion}

Our stream library has a strong support for programming in a monadic style, even
though $\Pull$ is not literally a monad. We can bind object values with
$\concatMap$, and we can also do let-insertion and case splitting for them. We
also get guaranteed well-typing, closure-freedom, and arbitrary mixing of
zipping and $\concatMap$.

We highlight the usage of the generativity axiom as well. Previously in staged
compilation, intensional analysis (i.e. the ability to analyze object code) has
been viewed as a desirable feature that increases the expressive power of a
system. To our knowledge, our work is the first one that exploits the
\emph{lack} of intensional analysis in metaprogramming. This is a bit similar to
parametricity in type theories, where the inability to analyze types has a
payoff in the form of ``free theorems'' \cite{wadler89theoremsforfree}.

Regarding the practical application of our stream library, we think that it
would make sense to support both push and pull fusion in a realistic
implementation, and allow users to benefit from the strong points of both. Push
streams, which we do not present in this paper, have proper $\Monad$ and
$\MonadGen$ instances and are often more convenient to use in CFTT. They are
also better for deep traversals of structures where they can utilize unbounded
stack allocations, while pull streams need heap allocations for unbounded space
in the machine state.

\subsection{Agda \& Haskell Implementations}


In Agda, to avoid computation getting stuck on the generativity axiom, we use
the \texttt{primTrustMe} built-in \cite{agdadocs} to automatically erase the axiom
when the sides of the equation are definitionally equal. Otherwise the
implementation closely follows this section.

In Haskell there are some limitations. First, in $\concatMap$, the
projection function $\mit{projS}$ cannot be defined, because Haskell is not
dependently typed, and the other field projections are also out of reach. We
only have a weaker ``positive'' recursion principle for existential types. It
might be the case that a strongly typed $\concatMap$ is possible with only weak
existentials, but we attempted this and found that it introduces too much
technical complication.

So instead of giving a single generic definition for $\concatMap$ for $\IsSOP$
types, we define $\concatMap$ just for object types,\footnote{Recall that we do
not distinguish value and computation types in Haskell.} and define
$\mit{case_\Pull}$ separately for each object type, as an overloaded class
method. In each of these definitions, we only need to deal with a concrete
finite number of machine state types, which is feasible to handle with weak
existentials.

Also, the generativity axiom is \emph{false} in Template Haskell, since it is
possible to look inside quoted expressions. Instead, we use type coercions that
can fail at unstaging time. If library users do not violate generativity, these
coercions disappear and the code output will not contain unsafe coercions.

\section{Related Work}\label{sec:related-work}

\subsubsection*{Two-level type theory.} Two-level lambda calculi were first developed in
the context of abstract interpretation and partial evaluation
\cite{DBLP:phd/ethos/Nielson84a,DBLP:books/daglib/0071307}.  This line of
research is characterized by simple types, having the same language features at
different stages, and an emphasis on \emph{binding time analysis},
i.e.\ automatically inferring stage annotations as part of a pipeline for
partial evaluation and program optimization.

Later and independently, the same notion of level appeared in homotopy type
theory, first in Voevodsky's Homotopy Type System \cite{hts}, and subsequently developed as
2LTT in \cite{twolevel}. Here, dependent types are essential, and we have different
theories at the two stages. The application of 2LTT to staged compilation was
developed in \cite{staged2ltt}. Binding-time analysis has not been developed in this
setting; 2LTT-s have been meant to be used as surface languages to directly work
in.

We build on both traditions. CFTT itself is a 2LTT and makes heavy use of
dependent type theory. However, we are mostly concerned with code generation and
optimization and borrow concepts from the partial evaluation literature.

\subsubsection*{Tracking function arities and closures in types} Downen et al.'s
intermediate language $\mathcal{IL}$ has similar motivations as our object
language \cite{DBLP:journals/pacmpl/DownenAJE20}. In both systems, function
types are distinguished from closure types, enjoy universal $\eta$-conversion
and have an explicit calling arity. In fact, our object language can be almost
viewed as a small simply-typed fragment of $\mathcal{IL}$, with only
$\mbf{letrec}$ missing from $\mathcal{IL}$.

\subsubsection*{Sums-of-products} $\SOP$ was proposed for generic Haskell programming
by De Vries and Löh \cite{DBLP:conf/icfp/VriesL14}. Our own $\SOP$
implementation in Haskell is mostly the same as in ibid. The
$\SOP$-of-object-code representation appeared as well in typed Template Haskell
in \cite{DBLP:conf/haskell/PickeringLW20}.

\subsubsection*{CPS and binding-time improvement}
Our $\Gen$ monad can be viewed as a variation on the continuation-passing-style
that has been often used in staged programming. Flanagan et al.'s ANF
translation algorithm uses let-insertion in CPS that works the same way as our
$\gen$ \cite{DBLP:conf/pldi/FlanaganSDF93}. Kiselyov et al.\ likewise uses
essentially the same CPS, in MetaOCaml
\cite{DBLP:conf/popl/KiselyovBPS17}. Jones et al.\ discuss CPS and binding-time
improvement in partial evaluation \cite{partial-evaluation}. $\eta$-expansion
for object-level finite sums is known as ``the trick'' \cite{eta-expansion-trick}.

\subsubsection*{Join points}
The use join points in GHC's core language is partly motivated by avoiding code
duplication in case-of-case transformations
\cite{DBLP:conf/pldi/MaurerDAJ17}. In our monad library, case-of-case is
implicitly and eagerly computed during staging, and we similarly use join points
to avoid code duplication.

\subsubsection*{Stream fusion}
The staged stream fusion library $\mathit{strymonas}$ by Kiselyov et al.\ is the
primary prior art \cite{DBLP:conf/popl/KiselyovBPS17}. Here, streams are represented as a sum type of
push and pull representations, allowing both zipping and
$\concatMap$-ing. However, fusion is not guaranteed for all combinations;
zipping two $\concatMap$-s reifies one of the streams in a runtime closure. In
newer versions of the library, fusion is reported to be complete and guaranteed
\cite{DBLP:conf/pepm/0001K24}, however, an exposition of this has not yet been published. Another
notable difference is that $\mathit{strymonas}$ relies on mutable references in
the object language, while we do not.

\emph{Machine fusion} \cite{DBLP:conf/ppdp/RobinsonL17} supports splitting
streams to multiple streams while avoiding duplicated computation; this is not
possible in $\mathit{strymonas}$ or our library. Machine fusion also supports
$\concatMap$ and zipping. However, its state machine representation is
significantly more complicated than ours, and its fusion algorithm is not
guaranteed to succeed on arbitrary stream programs.

Coutts developed pull stream fusion in depth in
\cite{DBLP:phd/ethos/Coutts11}. We borrowed the basic design and the basic
stream combinator definitions from there. This account is also close to existing
stream implementations in Haskell. Ibid.\ characterizes fusibility in a
non-staged setting, in terms of ``good consumers'' and ``good producers''.  This
scheme does not cover $\concatMap$.

\section{Conclusions and Future Work}

We believe that a programming language in the style of this paper would be
highly useful, especially in high-performance functional programming and
domain-specific programming. In particular, the pipeline for using and compiling
monads could look like the following.
\begin{itemize}
\item Users write definitions in a style similar to Haskell, without quotes,
      splices, $\mup$-s and $\mdown$-s, only marking stages in type annotations and
      in let-definitions, with $:=$ and $=$.  Storing monadic actions at runtime
      requires an explicit boxing operation that yields a closure type.
\item Type- and stage-directed elaboration adds the missing operations, and also desugars
      case splitting using $\join$ and the underlying splitting function.
\item In the downstream compiler, fast \& conservative optimization passes are
      sufficient, since abstraction overheads are already eliminated by staging. Eliminating
      dead code and unused arguments would be important, since we have not yet addressed
      this through staging.
\item A systematic way to de-duplicate code would be also needed, probably already
      during staging.
\end{itemize}

Going a bit further, we believe that a 2LTT-based language could be a good design point
for programming in general, buying us plenty of control over code generation for
a modest amount of extra complexity.

Also, rebuilding known abstractions in staged programming is valuable because it
provides a \emph{semantic explanation} of how abstractions can be compiled, and
provides insights that could be reused even in non-staged settings. For
instance, in this paper we demonstrated that compiling monadic code can be done
by using \emph{the same monads} in the metalanguage, extended with code
generation as an effect.

Continuing this line of thought, it could be interesting to also adapt to 2LTT
the style of \emph{binding-time analysis} that is well-known in partial
evaluation. For example, we might do program optimization in the following way.
First, start with monadic code in a non-staged language. Second, try to infer
stages, inventing quotes, splices, $\mup$-s and $\mdown$-s, thereby translating
definitions to a 2LTT. Third, perform staging and proceed from there. This would
be more fragile than unambiguous stage inference, but we would still benefit from
shifting a lot of machinery into staging (which is deterministic and efficient).

We also find it interesting how little impact closure-freedom had on the
developments in this paper. This provides some evidence that in staged
functional programming, closures can be an opt-in feature instead of the
pervasive default.

In this paper we only briefly looked at two applications. Both monad
transformers and streams could be further developed, and many other programming
abstractions could be revisited in the staged setting.
\begin{itemize}
\item
     We did not discuss the issue of \emph{tail calls} in monadic code. For
     example, a tail call in $\Maybe$ should not case split on the result, while
     in our current library, making a call with $\mup$ always does so. We did
     develop an abstraction for tail calls in the Agda and Haskell
     implementations, but omitted it here partly for lack of space, and partly
     because we have not yet explored much of the design space.

\item In a practical stream library, it would be useful to further try to
     minimize the size of the machine state. In the Agda and Haskell versions, we
     additionally track whether streams can $\Skip$, to provide more efficient
     zipping for non-skipping ``synchronous'' streams. However, this could be
     refined in many ways, and we could also try to represent the full transition
     graph in an observable way, thereby enabling merging or deleting states.

\item It seems promising to look for synergies between push streams, pull streams
     and monad transformers. It seems that pull streams could be generalized from
     $\Gen$ to different monads, in the representation of transitions and initial
     states. This would allow putting a $\Pull$ \emph{on the top} of a monad
     transformer stack. On the other hand, push streams form a proper monad, but
     they cannot be used to ``transform'' other monads, so they can be placed
     at the \emph{bottom} of a transformer stack.
\item We could try to lean more heavily on datatype-generic programming and
     generalize abstractions for more object types. For example, push and pull
     streams could be generalized to inductive and coinductive Church-codings of
     arbitrary value types.
\end{itemize}
\newpage

\appendices
\section{Staged Semantics of CFTT}\label{appendix:A}

In this appendix we describe the semantics of CFTT, from which the unstaging
algorithm is obtained, and also describe the soundness and completeness of the
algorithm. We borrow the setup from \cite{staged2ltt}; we use terminology and
definitions from there by default, and focus on differences and extensions.

\begin{notation} In the following, we work in a mostly unspecified extensional
type theory, writing $\Set$ for universes (omitting universe levels), $(x : A) \to B$
for $\Pi$-types, $(x : A) \times B$ for $\Sigma$-types and $\blank\!=\!\blank$ for
extensional identity.

\end{notation}

\subsection{Models and Syntax of CFTT}

We give an algebraic \cite{TODO} specification for the models of CFTT and define
the syntax of CFTT to be the initial model. In the following, all type and term
formers are implicitly assumed to be stable under substitution.

\begin{definition}[] A model of CFTT consists of the following.

\begin{itemize}
\item A category with a terminal object; in the syntax this is the category
  of contexts and parallel substitutions. We use $\Con : \Set$ for contexts,
  $\Sub : \Con \to \Con \to \Set$ for morphisms and $\emptycon : \Con$ for the
  terminal object (which is the empty context in the syntax).
\item A family structure in the sense of category-with-families \cite{TODO},
  specifying metatypes and terms. This structure is additionally indexed by
  universe levels, so we have $\MTy_i : \Con \to \Set$ and $\msf{MetaTm}_i :
  (\Gamma : \Con) \to \MTy_i\,\Gamma \to \Set$ for $i \in \mbb{N}$. We support
  basic type formers, identity types and W-types. We write $\Gamma \ext A$ for
  an extended context. Additionally, we have Coquand-style universes,
  i.e.\ $\U_i : \MTy_{i+1}\,\Gamma$ such that there is an isomorphism
  $\msf{MetaTm}_{i+1}\,\Gamma\,\U_i \simeq \MTy_i\,\Gamma$.  We use $\El$ and
  $\code$ to denote the components of this isomorphism.
\item
  For object-level types, we have $\Ty : \MTy_0\,\Gamma$, $\VTy :
  \MTy_0\,\Gamma$ and $\CTy : \MTy_0\,\Gamma$, together with all object-level
  type formers and inclusion operations $\msf{V} : \MTm\,\Gamma\,\VTy \to \MTm\,\Gamma\,\Ty$
  and $\msf{C} : \MTm\,\Gamma\,\CTy \to \MTm\,\Gamma\,\Ty$.
\item For object-level terms, we have $\Tm : \MTm\,\Gamma\,\Ty \to \Set$,
  supporting context extension, i.e.\ we have $\Gamma,\ext\,A$ for
  $A : \MTm\,\Gamma\,\Ty$ such that $\Sub\,\Gamma\,(\Delta\,\ext\,A)$ is isomorphic
  to $(\sigma : \Sub\,\Gamma\,\Delta) \times \Tm\,\Gamma\,(A[\sigma])$. All
  object-level term formers are supported in $\Tm$.
\item Staging operations: we have $\Lift : \MTm\,\Gamma\,\Ty \to \MTy\,\Gamma$ together
  with isomorphisms $\Tm\,\Gamma\,A \simeq \MTm\,\Gamma\,(\Lift A)$.
  We write $\ql\!\blank\!\qr$ and $\spl\!\blank$ for the components of the isomorphism.
\end{itemize}
\end{definition}
This presentation is a bit more verbose than the surface syntax used in the
paper. First, the meta level uses explicit coding and decoding operations for
universes, while the surface syntax assumes Russell-style universes. Second, the
object level has $\VTy$ and $\CTy$ as Tarski-style universes with explicit
inclusion into $\Ty$, while the surface syntax used implicit coercions.

Also, this is not the only possible presentation. Here, we follow the style of
the informal syntax by having a separate \emph{sort} for object-level terms,
together with explicit quoting and splicing. However, we do not have a sort for
object types and instead specify them as terms of metatypes. This is a mixture
of two styles:
\begin{itemize}
\item The ``classic'' 2LTT style would have separate sorts and staging
  operations for all of $\Ty$, $\CTy$ and $\VTy$. Here the two levels are
  specified independently and level-mixing is only possible through staging
  operations.
\item The ``higher-order abstract syntax'' (HOAS) or ``logical framework'' style
  would only have metatypes and metaterms, no quoting or splicing, and would
  reuse metafunctions to represent object-level binders.Our Agda implementation
  actually follows this style.
\end{itemize}
In the context of CFTT, the choice between HOAS and 2LTT styles is a matter of
taste, and the two are inter-derivable \cite{TODO}.

\subsection{Presheaf Model}

Unstaging is defined as evaluation of the syntax of CFTT in a particular
presheaf model.
\begin{itemize}
  \item The \emph{object theory} is the simple type theory supporting only value
    types and computation types as described in Section
    \ref{sec:the-object-level} and also in the Agda
    formalization. Object-theoretic terms are not quotiented by
    $\beta\eta$-conversion. Substitution is a recursively defined operation on
    terms.
  \item $\mbbo$ is the category whose objects are contexts of the object theory,
    and morphisms are parallel substitutions, as lists of terms. With this,
    object-theoretic types form a constant presheaf which we denote
    as $\Tyo : \mbbo^{\msf{op}} \to \Set$, and terms form a dependent presheaf over $\Tyo$ which we write as
    $\Tmo$.
  \item
  In the presheaf model of CFTT, every typing context (and also every closed
  type) is modeled as a presheaf over $\mbbo$. The interpretation of meta-level
  type formers is standard, following Hofmann \cite{Hofmann97syntaxand} and
  Huber \cite{huber-thesis}.
\item $\Ty : \MTy_0\,\Gamma$ is the displayed presheaf which is constantly
  $\Tyo$. $\VTy$ and $\CTy$ are modeled analogously.
\item $\Lift A : \MTy_0\,\Gamma$ is modeled as follows. We have that $A :
  \MTy\,\Gamma\,\Ty$, but since $\Ty$ is defined as constantly $\Tyo$, it
  follows that $A$ is a natural transformation from $\Gamma$ to $\Tyo$. Hence,
  we define $\Lift A : \MTy_0\,\Gamma$ to be $\Tmo[A]$, where $[\blank]$ is
  substitution in presheaves, i.e.\ the reindexing of a displayed presheaf by a
  natural transformation.
\item
  $\Tm\,\Gamma\,A$ is simply defined to be $\MTy\,\Gamma\,(\Lift A)$ in the model. Hence,
  quotation and splicing are both interpreted as identity maps.
\end{itemize}
Now, we can take a closed term $t : \Tm\,\emptycon\,A$ in CFTT and evaluate it
in the presheaf model to obtain an object-theoretic term whose type is the
result of evaluating $A$.

\subsection{W-types}
So far we have stayed close to \cite{staged2ltt}, with only cosmetic
changes. Supporting W-types, as a way to get inductive families \cite{TODO}, is
the biggest extension here. We extend Section 5 of \cite{staged2ltt} here. In
the syntax of CFTT, $\vW$-types are as follows. The formal specification of
syntactic $\vW{-}\msf{elim}$ is rather gnarly, because of the usage of explicit
substitutions, but we write it out here anways.

\begin{notation} We use $\p^N$ for $N$-fold
weakening and we also use numerals as De Bruijn indices, to abbreviate
$\q[\p^N]$. We write $\msf{app}$ and $\msf{lam}$ for $\Pi$ application and
abstraction. We write $\blank{\Rightarrow}\blank$ for non-dependent function
types. Finally, for the sake of brevity we omit the component maps of the isomorphism
$\re_\ext : \re\,(\Gamma \ext A) \simeq ((\gamma : \re\,\Gamma) \times \re\,A\,\gamma)$.
\end{notation}
\begin{alignat*}{3}
  &\vW   &&: (A : \MTy_i\,\Gamma) \to \MTy_j\,(\Gamma\,\ext\,A) \to \MTy_{\msf{max}(i,j)}\,\Gamma\\
  &\vsup &&: (a : \MTm\,\Gamma\,A) \to \MTm\,\Gamma\,(B[\vid,\,a] \Rightarrow \vW\,A\,B) \to \MTm\,\Gamma\,(\vW\,A\,B)\\
  &\vW{-}\msf{elim}  &&: (P : \MTy\,(\Gamma\,\ext\,\vW\,A\,B))\\
  &                  && \hspace{0.6em}\to \MTm\,(\Gamma\,\ext\,A\,\ext\,B \Rightarrow (\vW\,A\,B)[p]\,\ext\,\Pi\,(B[p])\,(P[\p^3,\,\msf{app}\,1\,0]))\,(P[\p^3,\,\vsup\,2\,1))\\
  &                  && \hspace{0.6em}\to (w : \MTm\,\Gamma\,(\vW\,A\,B)) \to \MTm\,\Gamma\,(P[\id,\,w])\\
  & \vW{\beta}       &&:
    \vW{-}\msf{elim}\,P\,s\,(\vsup\,a\,f) = s[\id,\,a,\,f,\,\msf{lam}\,(\vW{-}\msf{elim}\,(P[\p^2,\,\q])\,(s[\p])\,(\msf{app}\,f\,0)]
\end{alignat*}
We shall omit the $i$ and $j$ universe levels in the following. In $\whset$,
i.e.\ internally to presheaves over $\mbbo$, we also have the standard W-type.
\begin{alignat*}{3}
  & \wh{\vW}   &&: (A : \whset) \to (A \to \whset) \to \whset\\
  & \wh{\vsup} &&: (a : A) \to (B\,a \to \wh{\vW}\,A\,B) \to \wh{\vW}\,A\,B\\
  & \wh{\vW{-}\msf{elim}} &&: (P : \wh{\vW}\,A\,B \to \whset)\\
  & && \hspace{0.6em}\to ((a : A)(f : B\,a \to \wh{\vW}\,A\,B) \to ((b : B\,a) \to P\,(f\,b)) \to P\,(\wh{\vsup}\,a\,f))\\
  & && \hspace{0.6em}\to (w : \wh{\vW}\,A\,B) \to P\,w\\
  & \wh{\vW}{\beta} &&: \wh{\vW{-}\msf{elim}}\,P\,s\,(\wh{\vsup}\,a\,f) = s\,a\,f\,(\lambda\,b.\,\wh{\vW{-}\msf{elim}}\,P\,s\,(f\,b))
\end{alignat*}
In the presheaf model we interpret $\vW$ using $\whW$.
\begin{alignat*}{3}
  & \ev\,(\vW\,A\,B)\,\gamma   &&:= \wh{\vW}\,(\ev\,A\,\gamma)\,(\lambda\,\alpha.\,\ev\,B\,(\gamma,\,\alpha))\\
  & \ev\,(\vsup\,a\,f)\,\gamma &&:= \wh{\vsup}\,(\ev\,a\,\gamma)\,(\lambda\,\beta.\,\ev\,f\,(\gamma, \beta))\\
  & \ev\,(\vW{-}\msf{elim}\,P\,s\,w)\,\gamma &&:= \wh{\vW{-}\msf{elim}}\,(\lambda\,w.\,\ev\,P\,(\gamma,\,w))\,(\lambda\,a\,f\,g.\,\ev\,s\,(\gamma,\,a,\,f,\,g))\,(\ev\,w\,\gamma)
\end{alignat*}

%% W-elim̂ (λ w. E P (γ, w)) (λ a f g. E s (γ, a, f, g)) (E w γ)
Now, assuming $A : \MTy\,\Gamma$ and $B : \MTy\,(\Gamma\,\ext\,A)$, we consider
$\re\,(\vW\,A\,B) : (\gamma : \re\,\Gamma) \to \whset$. Externally, the elements
of $\re\,(\vW\,A\,B)\,\gamma$ are CFTT terms with a $\vW$-type in purely
object-level contexts. We call such terms \emph{restricted}. Now, we internalize
the restriction of the $\vsup$ constructor:
\[ \re_{\vsup} : (\alpha : \re\,A\,\gamma) \to \re\,(B \Rightarrow (W\,A\,B)[\p])\,(\gamma,\,\alpha) \to \re\,(W\,A\,B)\,\gamma \]
Externally, $\re_{\vsup}$ applies $\vsup$ to restricted terms. Note that in
$\re\,(B \Rightarrow (W\,A\,B)[\p])\,(\gamma,\,\alpha)$, the type $B \Rightarrow (W\,A\,B)[\p]$ is in
the context $(\Gamma\,\ext\,A)$, which makes it possible to pass $(\gamma,\,\alpha)$ as
argument; this ``instantiates'' $B$ to $\alpha$. With this, we can also
show that $\re$ preserves $\vsup$:
\[ \re\,(\vsup\,t\,u)\,\gamma = \re_\vsup\,(\re\,t\,\gamma)\,(\re\,u\,\gamma) \]
Note that the type of $\re\,u\,\gamma$ is $\re\,(B[\id,\,t] \Rightarrow
\vW\,A\,B)\,\gamma$, which is equal to $\re\,((B \Rightarrow
(\vW\,A\,B)[\p])[\id,\,t])\,\gamma$, which is in turn equal to $\re\,(B
\Rightarrow (\vW\,A\,B)[\p])\,(\gamma,\,\re\,t\,\gamma)$ by $\re$'s preservation
of substitutions, which makes the above equation well-typed.

Next, we give a logical relation interpretation for $\vW$. We define an
inductive relation in $\whset$ as follows.
\begin{alignat*}{3}
  & \vW\rel &&: \Gamma\rel\,\gamma\,\gamma' \to \ev\,(\vW\,A\,B)\,\gamma \to \re\,(\vW\,A\,B)\,\gamma' \to \whset\\
  & \vsup\rel &&: \hspace{0.8em}\{\alpha : \ev\,A\,\gamma\}\{\alpha' : \re\,A\,\gamma'\}(\alpha\rel : A\rel \gamma\rel\,\alpha\,\alpha')\\
  &           &&\to \{f \hspace{0.5em}: \ev\,B\,(\gamma,\,\alpha) \to \ev\,(\vW\,A\,B)\,\gamma\}\\
  &           &&\to \{f' \hspace{0.2em}: \re\,(B \Rightarrow (\vW\,A\,B)[\p])\,(\gamma',\,\alpha')\}\\
  &           &&\to (f\rel : (B \Rightarrow (\vW\,A\,B)[\p])\rel\,(\gamma\rel,\,\alpha\rel)\,f\,f')\\
  &           &&\to \vW\rel\,\gamma\rel\,(\wh{\vsup}\,\alpha\,f)\,(\re_\vsup\,\alpha'\,f')
\end{alignat*}
Here, $(B \Rightarrow (\vW\,A\,B)[\p])\rel$ is the relational interpretation of the function type, which unfolds to
pointwise preservation of relations. In short, $\vW\rel$ expresses that $\wh{\vsup}\,\alpha\,f$ is related $\re_\vsup\,\alpha'\,f'$
when the arguments (inductively) related. We can immediately interpret $\vW$ and $\vsup$:
\begin{alignat*}{3}
  & (\vW\,A\,B)\rel\,\gamma\rel\,t\,t' &&:= \vW\rel\,\gamma\rel\,t\,t'\\
  & (\vsup\,t\,u)\rel\,\gamma\rel      &&:= \vsup\rel\,(t\rel\,\gamma\rel)\,(u\rel\,\gamma\rel)
\end{alignat*}
It only remains to interpret $\vW$-elimination and its $\beta$-rule. We need to define
\[ (\vW{-}\msf{elim}\,P\,s\,w)\rel\,\gamma\rel : P\rel\,(\gamma\rel,\,w\rel\,\gamma\rel)\,(\ev\,(\welim\,P\,s\,w)\,\gamma)\,(\re\,(\welim\,P\,s\,w)\,\gamma') \]
We do this by induction on $w\rel\,\gamma\rel :
\vW\rel\,\gamma\rel\,(\ev\,w\,\gamma)\,(\re\,w\,\gamma')$. At this point it
becomes rather tedious to compute and unfold the involved types, so we only give
a compact definition here together with some informal explanation:
\begin{alignat*}{3}
  &(\vW{-}\msf{elim}\,P\,s\,w)\rel\,\gamma\rel :=\\
  &\hspace{4em}\vW\rel{-}\msf{elim}\,(\lambda\,w\rel.\,\,P\rel\,(\gamma\rel,\,w\rel)\,(\ev\,(\welim\,P\,s\,w)\,\gamma)\,(\re\,(\welim\,P\,s\,w)\,\gamma'))\\
  &\hspace{8.1em}(\lambda\,a\rel\,f\rel\,\mathit{hyp}.\,\,s\rel\,(\gamma\rel,\,a\rel,\,f\rel,\,\mathit{hyp}))\\
  &\hspace{8.1em}(w\rel\,\gamma\rel)
\end{alignat*}
$\vW\rel{-}\msf{elim}$ is the eliminator for $\vW\rel$. Its first argument is
the induction motive, which matches the goal type. The second argument is the
method for $\vsup\rel$; its $a\rel$ and $f\rel$ inputs are the relatedness
witnesses stored in a $\vsup\rel$, while $\mathit{hyp}$ is induction hypothesis,
witnessing the inductive motive for every output of $f\rel$. In the body here,
we need to witness the motive in the case where $w$ is a $\vsup$, which causes
both $\ev\,(\welim\,P\,s\,w)\,\gamma)$ and $\re\,(\welim\,P\,s\,w)\,\gamma')$ to
compute further to applications of $\ev\,s\,\gamma$ and $\re\,s\,\gamma'$
respectively. Hence, we use $s\rel$ to relate the two sides. The
$\vW{\beta}$-rule also evidently holds with this definition.

\subsection{Identity type}




\interlinepenalty=10000
\bibliography{references}

\end{document}
\endinput
