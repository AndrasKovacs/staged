
\documentclass{easychair}

\usepackage{doc}
\usepackage{scalerel}
\usepackage{amsfonts}

\newcommand{\easychair}{\textsf{easychair}}
\newcommand{\miktex}{MiK{\TeX}}
\newcommand{\texniccenter}{{\TeX}nicCenter}
\newcommand{\makefile}{\texttt{Makefile}}
\newcommand{\latexeditor}{LEd}
\newcommand{\emptycon}{\scaleobj{.75}\bullet}
\newcommand{\ext}{\triangleright}
\newcommand{\arri}{\Rightarrow}

\newcommand{\ms}[1]{\mathsf{#1}}
\newcommand{\ToS}{\mathsf{ToS}}
\newcommand{\U}{\mathsf{U}}
\newcommand{\Code}{\mathsf{Code}}
\newcommand{\Ty}{\mathsf{Ty}}
\newcommand{\V}{\mathsf{V}}
\newcommand{\C}{\mathsf{C}}
\newcommand{\Bool}{\ms{Bool}}

\title{Staged Compilation and Generativity%
  \thanks{The author was supported by the European Union, co-financed
    by the European Social Fund (EFOP-3.6.3-VEKOP-16-2017-00002).}}

\author{
Andr\'as Kov\'acs
}

% Institutes for affiliations are also joined by \and,
\institute{
  E\"otv\"os Lor\'and University,
  Budapest, Hungary \\
  \email{kovacsandras@inf.elte.hu}
}

\authorrunning{Kov\'acs}
\titlerunning{Staged Compilation and Generativity}
\pagenumbering{gobble}
\begin{document}

\maketitle

The purpose of staged compilation is to write code-generating programs in a safe
and ergonomic way. Although it is always possible to write metaprograms by
simply manipulating strings or deeply embedded syntax trees, this is often
error-prone and tedious. Staging is a way to have more guarantees about the
safety and well-typing of metaprograms, and also a way to integrate object-level
and meta-level syntaxes more organically.

Two-level type theory (2LTT) \cite{twolevel} was originally developed for the
purpose of doing synthetic homotopy theory, by adding a metaprogramming layer on
top of homotopy type theory \cite{hottbook}. However, it turns out that 2LTT is
also a great framework for metaprogramming and staging in general; it is
applicable to a wide range of object theories.

There is a natural semantics to 2LTT which justifies the metaprogramming view:
this is the \emph{presheaf model} of 2LTT. Here, meta-level types are presheaves
over the underlying category of the object theory. Hence, every meta-level
construction must be stable under the object-level morphisms.

\paragraph{Generativity}Generativity means that we can only generate code, but not look inside
object-level syntax and make decisions based on that. Generativity simplifies
staging, and it is often enforced in practical implementations
\cite{kiselyov14metaocaml}. However, non-generative staging provides additional
power and flexibility. A simple example for a non-generative feature is
conversion checking. Assume that given an object-level type $A$, $\Code\,A$ is
the type of meta-level programs which compute $A$-expressions. Conversion
checking may be postulated as $\ms{conversion}_A : (t\,u : \Code\,A) \to (t = u)
+ (t \neq u)$. We aim to investigate generativity in the presheaf model. We
observe that it depend on the choice of morphisms in the object theory.

If base morphisms are \emph{substitutions}, then $\ms{conversion}$ does not hold
in the model, because definitional inequality is not stable under substitution. For
example, inequal variables may become equal after substitution.

If base morphisms are \emph{weakenings}, then $\ms{conversion}$
holds. However, in this case the object theory can only support features which
can be specified with weakenings. For instance, this allows simple types at at
the object level, without any $\beta$-rules, but it does not allow polymorphism
or dependent types. This is still sufficient for many staging applications.

\paragraph{Yoneda lemma} It is worth to note that we get generativity statements
in the presheaf model from the Yoneda lemma. For example, the natural
transformation $\ms{y}\,\Bool_0 \Rightarrow \Bool_1$ must be a constant map in the
model. How is it possible then to have non-generativity? The answer is that
the Yoneda lemma can only meaningfully apply to \emph{types} (as opposed to
contexts) in the presheaf model, if context extension is preserved by Yoneda
embedding. If base morphisms are weakenings, the base category does not have all
binary products, and context extension is not preserved by $\ms{y}$. Hence, the
singleton context $\emptycon,\ms{y}\,\Bool_0$ is not representable, and the
Yoneda lemma does not say anything about dependency on $\ms{y}\,\Bool_0$.


\bibliographystyle{plain}
\bibliography{references}

\end{document}
