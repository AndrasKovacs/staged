%% build: latexmk -pdf -pvc prez.tex

\documentclass[dvipsnames,aspectratio=169]{beamer}
\usetheme{Madrid}

%% kill footline
\setbeamertemplate{footline}[frame number]{}
\setbeamertemplate{navigation symbols}{}
\setbeamertemplate{footline}{}

%% bibliography
\bibliographystyle{alpha}
\setbeamerfont{bibliography item}{size=\footnotesize}
\setbeamerfont{bibliography entry author}{size=\footnotesize}
\setbeamerfont{bibliography entry title}{size=\footnotesize}
\setbeamerfont{bibliography entry location}{size=\footnotesize}
\setbeamerfont{bibliography entry note}{size=\footnotesize}
\setbeamertemplate{bibliography item}{}

%% kill ball enumeration
\setbeamertemplate{enumerate items}[circle]
\setbeamertemplate{section in toc}[circle]

%% kill block shadows
\setbeamertemplate{blocks}[rounded][shadow=false]
\setbeamertemplate{title page}[default][colsep=-4bp,rounded=true]

%% kill ball itemize
\setbeamertemplate{itemize items}[circle]

%% --------------------------------------------------------------------------------

\usepackage[utf8]{inputenc}
%% \usepackage[hidelinks]{hyperref}
\usepackage{amsmath}
\usepackage{cite}
\usepackage{amsthm}
\usepackage{amssymb}
\usepackage{amsfonts}
\usepackage{mathpartir}
\usepackage{scalerel}
\usepackage{stmaryrd}
\usepackage{bm}
\usepackage{graphicx}
\usepackage{fontspec}
\usepackage{textcomp}

\setmonofont[Scale=0.8]{DejaVu Sans Mono}

%% --------------------------------------------------------------------------------

%% HoTT style composition
\makeatletter
\DeclareRobustCommand{\sqcdot}{\mathbin{\mathpalette\morphic@sqcdot\relax}}
\newcommand{\morphic@sqcdot}[2]{%
  \sbox\z@{$\m@th#1\centerdot$}%
  \ht\z@=.33333\ht\z@
  \vcenter{\box\z@}%
}
\makeatother

\renewcommand{\mit}[1]{{\mathit{#1}}}
\newcommand{\ttt}[1]{{\texttt{#1}}}
\newcommand{\msf}[1]{{\mathsf{#1}}}
\newcommand{\mbf}[1]{{\mathbf{#1}}}
\newcommand{\mbb}[1]{\mathbb{#1}}
\newcommand{\U}{\mathsf{U}}
\newcommand{\code}{\mathsf{code}}
\newcommand{\bs}[1]{\boldsymbol{#1}}
\newcommand{\wh}[1]{\widehat{#1}}
\newcommand{\mdo}{\mbf{do}\,}
\newcommand{\ind}{\hspace{1em}}
\newcommand{\bif}{\mbf{if}\,}
\newcommand{\bthen}{\mbf{then}\,}
\newcommand{\belse}{\mbf{else}\,}
\newcommand{\return}{\mbf{return}\,}
\newcommand{\pure}{\mbf{pure}\,}
\newcommand{\lam}{\lambda\,}
\newcommand{\data}{\mbf{data}\,}
\newcommand{\where}{\mbf{where}}
\newcommand{\M}{\msf{M}}
\newcommand{\letrec}{\mbf{letrec}\,}
\newcommand{\of}{\mbf{of}\,}
\newcommand{\go}{\mit{go}}
\newcommand{\add}{\mit{add}}
\newcommand{\letdef}{\mbf{let\,}}
\newcommand{\map}{\mit{map}}
\newcommand{\emptycon}{\scaleobj{.75}\bullet}
\newcommand{\Tyo}{\msf{Ty}_{\mbbo}}
\newcommand{\Tmo}{\msf{Tm}_{\mbbo}}
\newcommand{\whset}{\wh{\Set}}
\newcommand{\ev}{\mbb{E}}
\newcommand{\re}{\mbb{R}}


\newcommand{\mbbc}{\mbb{C}}
\newcommand{\mbbo}{\mbb{O}}

\newcommand{\vas}{\mathsf{as}}
\newcommand{\vbs}{\mathsf{bs}}
\newcommand{\vcs}{\mathsf{cs}}
\newcommand{\vxs}{\mathsf{xs}}
\newcommand{\vys}{\mathsf{ys}}
\newcommand{\vsp}{\mathsf{sp}}
\newcommand{\vma}{\mathsf{ma}}
\newcommand{\vga}{\mathsf{ga}}
\newcommand{\vm}{\mathsf{m}}
\newcommand{\vn}{\mathsf{n}}
\newcommand{\vk}{\mathsf{k}}
\newcommand{\vA}{\mathsf{A}}
\newcommand{\vB}{\mathsf{B}}
\newcommand{\vC}{\mathsf{C}}
\newcommand{\vS}{\mathsf{S}}
\newcommand{\vF}{\mathsf{F}}
\newcommand{\vR}{\mathsf{R}}
\newcommand{\vM}{\mathsf{M}}
\newcommand{\vmb}{\mathsf{mb}}
\newcommand{\mAs}{\mathsf{As}}
\newcommand{\va}{\mathsf{a}}
\newcommand{\vb}{\mathsf{b}}
\newcommand{\vc}{\mathsf{c}}
\newcommand{\vd}{\mathsf{d}}
\newcommand{\vx}{\mathsf{x}}
\newcommand{\vy}{\mathsf{y}}
\newcommand{\vz}{\mathsf{z}}
\newcommand{\vf}{\mathsf{f}}
\newcommand{\vfs}{\mathsf{fs}}
\newcommand{\vg}{\mathsf{g}}
\newcommand{\vh}{\mathsf{h}}
\newcommand{\vt}{\mathsf{t}}
\newcommand{\vs}{\mathsf{s}}
\newcommand{\vr}{\mathsf{r}}
\newcommand{\vu}{\mathsf{u}}
\newcommand{\vl}{\mathsf{l}}
\newcommand{\vns}{\mathsf{ns}}
\newcommand{\vW}{\mathsf{W}}
\newcommand{\vsup}{\mathsf{sup}}
\newcommand{\vid}{\mathsf{id}}
\newcommand{\whW}{\wh{\vW}}


\newcommand{\SOP}{\msf{SOP}}
\newcommand{\El}{\msf{El}}
\newcommand{\USOP}{\msf{U}_{\msf{SOP}}}
\newcommand{\Uprod}{\msf{U_P}}
\newcommand{\Elprod}{\msf{El_{P}}}
\newcommand{\IsSOP}{\msf{IsSOP}}
\newcommand{\forEach}{\msf{forEach}}
\newcommand{\single}{\msf{single}}
\newcommand{\msplit}{\msf{split}}
\newcommand{\mapGen}{\msf{mapGen}}
\newcommand{\genPull}{\msf{gen_{Pull}}}
\newcommand{\Set}{\msf{Set}}
\newcommand{\casePull}{\msf{case_{Pull}}}
\newcommand{\appull}{\ap_{\Pull}}

\newcommand{\Con}{\msf{Con}}
\newcommand{\Sub}{\msf{Sub}}
\newcommand{\Tm}{\msf{Tm}}
\newcommand{\Mor}{\msf{Tm}}

\newcommand{\ext}{\triangleright}

\newcommand{\Int}{\msf{Int}}
\newcommand{\List}{\msf{List}}
\newcommand{\Tree}{\msf{Tree}}
\newcommand{\Node}{\msf{Node}}
\newcommand{\Leaf}{\msf{Leaf}}
\newcommand{\Nil}{\msf{Nil}}
\newcommand{\Cons}{\msf{Cons}}
\newcommand{\Reader}{\msf{Reader}}
\newcommand{\ReaderT}{\msf{ReaderT}}
\newcommand{\Monad}{\msf{Monad}}
\newcommand{\Applicative}{\msf{Applicative}}
\newcommand{\class}{\msf{class}}
\newcommand{\Functor}{\msf{Functor}}
\newcommand{\Bool}{\msf{Bool}}
\newcommand{\Statel}{\msf{State}}
\newcommand{\fro}{\leftarrow}
\newcommand{\case}{\mbf{case\,}}
\newcommand{\foldr}{\msf{foldr}}
\newcommand{\foldl}{\msf{foldl}}
\newcommand{\rep}{\msf{rep}}
\newcommand{\concatMap}{\msf{concatMap}}

\newcommand{\Lift}{{\Uparrow}}
\newcommand{\Up}{{\Uparrow}}
\newcommand{\spl}{{\bs{\sim}}}
\newcommand{\ql}{{\bs{\langle}}}
\newcommand{\qr}{{\bs{\rangle}}}
\newcommand{\bind}{\mathbin{>\!\!>\mkern-6.7mu=}}

\newcommand{\MTy}{\msf{MetaTy}}
\newcommand{\MTm}{\msf{MetaTm}}
\newcommand{\VTy}{\msf{ValTy}}
\newcommand{\Ty}{\msf{Ty}}
\newcommand{\CTy}{\msf{CompTy}}
\newcommand{\True}{\msf{True}}
\newcommand{\False}{\msf{False}}
\newcommand{\fst}{\msf{fst}}
\newcommand{\snd}{\msf{snd}}

\newcommand{\blank}{{\mathord{\hspace{1pt}\text{--}\hspace{1pt}}}}

\newcommand{\Nat}{\msf{Nat}}
\newcommand{\Zero}{\msf{Zero}}
\newcommand{\Suc}{\msf{Suc}}
\newcommand{\Maybe}{\msf{Maybe}}
\newcommand{\MaybeT}{\msf{MaybeT}}
\newcommand{\Nothing}{\msf{Nothing}}
\newcommand{\Just}{\msf{Just}}

\theoremstyle{remark}
\newtheorem{notation}{Notation}
\newtheorem*{axiom}{Axiom}

\newcommand{\id}{\mit{id}}
\newcommand{\mup}{\mbf{up}}
\newcommand{\mdown}{\mbf{down}}
\newcommand{\tyclass}{\mbf{class}}
\newcommand{\instance}{\mbf{instance}\,}
\newcommand{\Improve}{\msf{Improve}}
\newcommand{\Gen}{\msf{Gen}}
\newcommand{\unGen}{\mit{unGen}}
\renewcommand{\Vec}{\msf{Vec}}
\newcommand{\gen}{\mit{gen}}
\newcommand{\genRec}{\mit{genRec}}
\newcommand{\fmap}{<\!\!\$\!\!>}
\newcommand{\ap}
\newcommand{\runGen}{\mit{runGen}}
\newcommand{\qt}[1]{\ql#1\qr}
\newcommand{\lift}{\mit{lift}}
\newcommand{\liftGen}{\mit{liftGen}}
\newcommand{\MonadGen}{\msf{MonadGen}}
\newcommand{\MonadState}{\msf{MonadState}}
\newcommand{\MonadReader}{\msf{MonadReader}}
\newcommand{\RA}{\Rightarrow}
\newcommand{\EitherT}{\msf{EitherT}}
\newcommand{\Either}{\msf{Either}}
\newcommand{\Left}{\msf{Left}}
\newcommand{\Right}{\msf{Right}}
\newcommand{\StateT}{\msf{StateT}}
\newcommand{\Identity}{\msf{Identity}}

\newcommand{\Stop}{\msf{Stop}}
\newcommand{\Skip}{\msf{Skip}}
\newcommand{\Yield}{\msf{Yield}}

\newcommand{\runIdentity}{\mit{runIdentity}}
\newcommand{\runReaderT}{\mit{runReaderT}}
\newcommand{\newtype}{\mbf{newtype}\,}
\newcommand{\runMaybeT}{\mit{runMaybeT}}
\newcommand{\runStateT}{\mit{runStateT}}
\newcommand{\runState}{\mit{runState}}
\newcommand{\dlr}{\,\$\,}
\newcommand{\ImproveF}{\msf{ImproveF}}
\newcommand{\ExceptT}{\msf{ExceptT}}
\newcommand{\State}{\msf{State}}
\newcommand{\SumVS}{\msf{SumVS}}
\newcommand{\ProdCS}{\msf{ProdCS}}
\newcommand{\Here}{\msf{Here}}
\newcommand{\There}{\msf{There}}
\newcommand{\IsSumVS}{\msf{IsSumVS}}
\newcommand{\MonadJoin}{\msf{MonadJoin}}
\newcommand{\Stream}{\msf{Stream}}
\newcommand{\join}{\mit{join}}
\newcommand{\modify}{\mit{modify}}
\newcommand{\get}{\mit{get}}
\newcommand{\mput}{\mit{put}}
\newcommand{\Rep}{\mit{Rep}}
\newcommand{\encode}{\mit{encode}}
\newcommand{\decode}{\mit{decode}}
\newcommand{\mindex}{\mit{index}}
\newcommand{\mtabulate}{\mit{tabulate}}
\newcommand{\States}{\mit{States}}
\newcommand{\seed}{\mit{seed}}
\newcommand{\step}{\mit{step}}
\newcommand{\Step}{\msf{Step}}
\newcommand{\Pull}{\msf{Pull}}
\newcommand{\MkPull}{\msf{MkPull}}

%% --------------------------------------------------------------------------------

\title{Polarized Lambda-Calculus at Runtime, Dependent Types at Compile Time}
\author{András Kovács}
%% \institute{University of Gothenburg}
\date{4 June 2024, CS retreat}
\begin{document}

\frame{\titlepage}

\begin{frame}

\large{\textbf{András Kovács}}
\vspace{0.3em}
\hrule
\vspace{1em}

Postdoc in Logic and Types under Thierry Coquand since 2023 September.
\vspace{1em}

Started in economics \& finance in Budapest, switched to CS,
did PhD in type theory.
\vspace{1em}

Some current interests:
\begin{itemize}
\item Fitness, nutrition.
\item ``Harsh vocals'' (screaming in metal music).
\end{itemize}
\vspace{1em}

Research:
\begin{itemize}
  \item Type theory: theory of inductive types, universes.
  \item Making proof assistants run fast (annoyed by Agda \& Coq).
  \item High-level high-performance programming (annoyed by Haskell).
\end{itemize}


\end{frame}

\begin{frame}[fragile]{Compiling monads today in Haskell}

\begin{columns}
\begin{column}{0.3\textwidth}
\textbf{GHC's input:}
\begin{verbatim}
    f :: Reader Bool Int
    f = do
      b ← ask
      if b then return 10
           else return 20



\end{verbatim}
\end{column}
\begin{column}{0.5\textwidth}
\textbf{GHC's \texttt{-O0} output:}
\begin{verbatim}
    dict :: Monad (Reader Int)
    dict = MkDict bindReader returnReader

    f :: Reader Bool Int
    f x = (>>=) dict (ask dict) (\b →
      case b of
        True  → return dict 10
        False → return dict 20)
\end{verbatim}
\end{column}
\end{columns}

\end{frame}


\begin{frame}[fragile]{Compiling monads today in Haskell}

\begin{columns}
\begin{column}{0.3\textwidth}
\textbf{GHC's \texttt{-O1} output:}
\begin{verbatim}
    f :: Bool → Int
    f b = case b of
      True  → 10
      False → 20
\end{verbatim}

\end{column}
\begin{column}{0.6\textwidth}
\begin{itemize}
  \item Elaboration to \texttt{-O0} is deterministic and relatively cheap.
  \item Going from \texttt{-O0} to \texttt{-O1} is \alert{hard} and needs
        a lot of machinery.
\end{itemize}
\end{column}
\end{columns}
\vspace{2em}

Example: \ttt{mapM} is third-order, rank-2 polymorphic, but almost all usages should
compile to first-order monomorphic code.
\begin{verbatim}
    mapM :: Monad m => (a → m b) → [a] → m [b]
\end{verbatim}

GHC has to guess the programmer's intent.
\vspace{1em}

\end{frame}

\begin{frame}[fragile]{Doing it differently}

\begin{columns}
\begin{column}{0.3\textwidth}
\textbf{Input in WIP language:}
\begin{verbatim}
    f : Reader Bool Int
    f := do
      b ← ask
      if b then return 10
           else return 20
\end{verbatim}

\end{column}
\begin{column}{0.6\textwidth}
\begin{itemize}
\item Looks similar to Haskell.
\item Desugaring \& elaboration does slightly more work.
\item Compiles to efficient code \emph{deterministically, without
      general-purpose optimization}.
\end{itemize}
\end{column}
\end{columns}
\vspace{1em}
\pause

\begin{block}{}
\textbf{Main idea}
\begin{itemize}
\item We use a \emph{two-level type theory (2LTT)}:
  \begin{itemize}
    \item Metalanguage (compile time): dependently typed, fancy features.
    \item Object language (runtime): simpler \& lower-level.
    \item The two are smoothly integrated.
  \end{itemize}
\item Monadic programs are \emph{metaprograms} which generate efficient
  runtime code.
\item Most optimizations are implemented in libraries instead of compiler internals.
\end{itemize}
\end{block}

\end{frame}

\begin{frame}[fragile]{The 2LTT}

Two type universes for the two levels.
\vspace{1em}
\begin{enumerate}
\item \textbf{\ttt{MetaTy}}: universe of meta-level types. Supports
      $\Pi$, $\Sigma$, inductive families.
\item \textbf{\ttt{Ty}}: universe of object-level types.
  \begin{itemize}
  \item \ttt{Ty} is itself an element of \ttt{MetaTy}.
  \item No polymorphism or type dependency in \ttt{Ty}.
  \item Two sub-universes:
  \begin{itemize}
  \item \ttt{CompTy} contains \emph{computation types}: functions, computational products.
  \item \ttt{ValTy}  contains \emph{value types}: ADT-s and closure types.
  \item ADT constructors only store values, functions only take value inputs.
  \end{itemize}
  \end{itemize}
\end{enumerate}
\vspace{0.5em}

\end{frame}

\begin{frame}[fragile]{The 2LTT}

\begin{columns}
\begin{column}{0.4\textwidth}
\textbf{A metaprogram}:
\begin{verbatim}
  id : {A : MetaTy} → A → A
  id x = x




\end{verbatim}
\end{column}

\begin{column}{0.6\textwidth}
\textbf{An object program:}
\begin{verbatim}
  data List (A : ValTy) := Nil | Cons A List

  myMap : List Int → List Int
  myMap ns := case xs of
    Nil       → Nil
    Cons n ns → Cons (n + 10) (myMap ns)
\end{verbatim}
\end{column}
\end{columns}


\end{frame}

\begin{frame}[fragile]{The 2LTT - interaction between stages}

\begin{itemize}
\item \textbf{Lifting}: for \ttt{A : Ty}, we have \ttt{⇑A : MetaTy}, as the type of
      metaprograms that produce \ttt{A}-typed object programs.
\item \textbf{Quoting}: for \ttt{t : A} and \ttt{A : Ty}, we have \ttt{<t>} as the metaprogram
      which immediately returns \ttt{t}.
\item \textbf{Splicing}: for \ttt{t : ⇑A}, we have \ttt{\char`~t : A} which runs the
       metaprogram \ttt{t} and inserts its output in some object-level code.
\item Definitional equalities: \ttt{\char`~<t> ≡ t} and \ttt{<\char`~t> ≡ t}.
\end{itemize}

\end{frame}

\begin{frame}[fragile]{Staged example}

\begin{verbatim}
    map : {A B : ValTy} → (⇑A → ⇑B) → ⇑(List A) → ⇑(List B)
    map f as = <letrec go as := case as of
                  Nil       → Nil
                  Cons a as → Cons ~(f <a>) (go as)
                in go ~as>

    myMap : List Int → List Int
    myMap ns := ~(map (λ x. <~x + 10>) <ns>)
\end{verbatim}

\end{frame}

\begin{frame}[fragile]{Staged example - with stage inference}

\begin{verbatim}
    map : {A B : ValTy} → (A → B) → List A → List B
    map f = letrec go as := case as of
                  Nil       → Nil
                  Cons a as → Cons (f a) (go as)
            in go

    myMap : List Int → List Int
    myMap := map (λ x. x + 10)
\end{verbatim}
\end{frame}

\begin{frame}[fragile]{A monad for code generation}

Type classes only exist in the metalanguage.
\begin{verbatim}
    class Monad (m : MetaTy → MetaTy) where
      return : a → m a
      (>>=)  : m a → (a → m b) → m b
\end{verbatim}
\ttt{Gen} is a Monad whose effect is \textbf{generating object code}:
\begin{verbatim}
    newtype Gen A = Gen {unGen : {R : Ty} → (A → ⇑R) → ⇑R}
    instance Monad Gen where ...

    runGen : Gen (⇑A) → ⇑A
    runGen (Gen f) = f id
\end{verbatim}

Generating an object-level \ttt{let}-definition:
\begin{verbatim}
    gen : {A : Ty} → ⇑A → Gen ⇑A
    gen {A} a = Gen $ λ k. <let x : A := ~a in ~(k <x>)>
\end{verbatim}

\end{frame}

\begin{frame}[fragile]{A monad for code generation}

\begin{columns}
\begin{column}{0.4\textwidth}
\textbf{Staged input:}
\begin{verbatim}
    myAction : ⇑Int → Gen ⇑Int
    myAction x = do
      y ← gen <~x + ~x>
      z ← gen <~y * ~y>
      pure <~y * ~z>

    foo : Int
    foo := ~(runGen $ myAction <10>)
\end{verbatim}
\end{column}
\begin{column}{0.5\textwidth}
\textbf{Output:}
\begin{verbatim}
    foo : Int
    foo := let y := 10 + 10 in
           let z := y * y in
           y * z




\end{verbatim}
\end{column}
\end{columns}

\end{frame}

\begin{frame}[fragile]{Staging monads}

Example for \ttt{Reader}:
\vspace{0.7em}

\begin{verbatim}
    newtype Identity (A : ValTy) := Identity {runIdentity : A}

    newtype ReaderT (R : ValTy) (M : ValTy → Ty)(A : ValTy) :=
      Reader {runReader : R → A}

    newtype ReaderTᴹ(R : MetaTy) (M : MetaTy → MetaTy)(A : MetaTy) =
      Readerᴹ {runReaderᴹ : R → A}
\end{verbatim}




\end{frame}


\begin{frame}[fragile]{Staging monads}

Instead of programming at type \ttt{ReaderTₒ R Identityₒ} (which is not a
monad!), we program at \ttt{ReaderT (⇑R) Gen}, and define back-and-forth
conversions:

\begin{verbatim}
    up : ⇑(ReaderT R Identity A) → ReaderTᴹ (⇑R) Gen (⇑A)
    up f = ReaderTᴹ $ λ r. pure <runIdentity (runReaderT ~f ~r)>

    down : ReaderTᴹ (⇑R) Gen (⇑A) → ⇑(ReaderTₒ R Identity A)
    down (ReaderTᴹ f) = <ReaderTₒ (λ r. Identity (~runGen (f <r>)))>
\end{verbatim}

\begin{block}{}
\textbf{In general:} up/down is defined by recursion on a transformer stack. The
bottom \ttt{Identity} is swapped to \ttt{Gen} at the meta-level.
\end{block}
\end{frame}


\begin{frame}[fragile]{Staging monads}

\begin{columns}
\begin{column}{0.5\textwidth}
Somewhat explicit source code:
\begin{verbatim}
    f : ReaderT Int Int
    f := ~(down $ do
       x <- ask
       pure <~x + 100>)
\end{verbatim}
\end{column}
\begin{column}{0.5\textwidth}
With more inference:
\begin{verbatim}
    f : ReaderT Int Int
    f := do
       x <- ask
       pure (x + 100)
\end{verbatim}
\end{column}
\end{columns}
\vspace{2em}

Generated output:
\begin{verbatim}
    f : ReaderT Int Identity Int
    f := ReaderT (λ n. Identity (n + 100))
\end{verbatim}

\end{frame}

\begin{frame}[fragile]{Polarization \& Closure-Freedom}

\emph{Computation} and \emph{value} types are tracked
in the object language.
\vspace{1em}

There's a value type former for \emph{closures},
that we \textbf{have not yet used} in this talk.
\vspace{1em}

The computational function type guarantees compilation
without closures, with only statically known calls!
\vspace{1em}

Essentially usage of closures is surprisingly rare
in programming.

\end{frame}


\begin{frame}

\begin{itemize}
  \item Conditionally accepted at ICFP 24: \emph{Closure-Free Functional Programming in a Two-Level Type Theory}.
  \item More things in paper: case splitting on object-level data, join points,
    stream fusion, more about polarized types.
  \item Implementations:
    \begin{itemize}
      \item In Agda and typed Template Haskell with some limitations.
      \item Standalone implementation planned, help from Ondrej Kubánek (MSc project).
    \end{itemize}
\end{itemize}
\vspace{2em}

\begin{center}

  \Large{\textbf{Thank you!}}

\end{center}



\end{frame}




\end{document}
