%% build: latexmk -pdf -pvc paper.tex

\documentclass[acmsmall,review,screen]{acmart}

%% \setcopyright{acmcopyright}
%% \copyrightyear{2018}
%% \acmYear{2018}
%% \acmDOI{XXXXXXX.XXXXXXX}

\bibliographystyle{ACM-Reference-Format}
\citestyle{acmauthoryear}

%% --------------------------------------------------------------------------------

\usepackage{xcolor}
\usepackage{mathpartir}
\usepackage{todonotes}
\presetkeys{todonotes}{inline}{}
\usepackage{scalerel}
\usepackage{bm}

\newcommand{\mit}[1]{\mathit{#1}}
\newcommand{\msf}[1]{\mathsf{#1}}
\newcommand{\mbb}[1]{\mathbb{#1}}
\newcommand{\mbf}[1]{\mathbf{#1}}
\newcommand{\bs}[1]{\boldsymbol{#1}}
\newcommand{\wh}[1]{\widehat{#1}}
\newcommand{\ext}{\triangleright}
\newcommand{\Code}{\msf{Code}}
\newcommand{\El}{\msf{El}}
\newcommand{\lam}{\msf{lam}}
\newcommand{\app}{\msf{app}}
\newcommand{\NatElim}{\msf{NatElim}}
\newcommand{\y}{\msf{y}}

\newcommand{\Lift}{{\Uparrow}}
\newcommand{\spl}{{\sim}}
\newcommand{\qut}[1]{\langle #1\rangle}

\newcommand{\mbbc}{\mbb{C}}
\newcommand{\mbbo}{\mbb{O}}
\newcommand{\ob}{_\mbbo}

\newcommand{\U}{\msf{U}}
\newcommand{\Con}{\msf{Con}}
\newcommand{\Sub}{\msf{Sub}}
\newcommand{\Ty}{\msf{Ty}}
\newcommand{\Tm}{\msf{Tm}}
\newcommand{\Cono}{\msf{Con}_{\mbbo}}
\newcommand{\Subo}{\msf{Sub}_{\mbbo}}
\newcommand{\Tyo}{\msf{Ty}_{\mbbo}}
\newcommand{\Tmo}{\msf{Tm}_{\mbbo}}
\newcommand{\hCon}{\wh{\msf{Con}}}
\newcommand{\hSub}{\wh{\msf{Sub}}}
\newcommand{\hTy}{\wh{\msf{Ty}}}
\newcommand{\hTm}{\wh{\msf{Tm}}}

\newcommand{\p}{\mathsf{p}}
\newcommand{\q}{\mathsf{q}}

\newcommand{\refl}{\msf{refl}}
\newcommand{\Bool}{\msf{Bool}}
\newcommand{\true}{\msf{true}}
\newcommand{\false}{\msf{false}}
\newcommand{\True}{\msf{True}}
\newcommand{\False}{\msf{False}}
\newcommand{\List}{\msf{List}}
\newcommand{\nil}{\msf{nil}}
\newcommand{\cons}{\msf{cons}}
\newcommand{\Nat}{\msf{Nat}}
\newcommand{\zero}{\msf{zero}}
\newcommand{\suc}{\msf{suc}}
\renewcommand{\tt}{\msf{tt}}
\newcommand{\fst}{\msf{fst}}
\newcommand{\snd}{\msf{snd}}
\newcommand{\mylet}{\msf{let}}
\newcommand{\emptycon}{\scaleobj{.75}\bullet}
\newcommand{\id}{\msf{id}}

\newcommand{\Set}{\mathsf{Set}}
\newcommand{\Prop}{\mathsf{Prop}}
\newcommand{\Rep}{\msf{Rep}}
\newcommand{\blank}{{\mathord{\hspace{1pt}\text{--}\hspace{1pt}}}}
\newcommand{\emb}[1]{\ulcorner#1\urcorner}

\newcommand{\Stage}{\msf{Stage}}
\newcommand{\hato}{\bm\hat{\mbbo}}
\newcommand{\ev}{\mbb{E}}
\newcommand{\re}{\mbb{R}}

\theoremstyle{remark}
\newtheorem{notation}{Notation}

\newcommand{\whset}{\wh{\Set}}
\newcommand{\rexti}{\re_{\ext_1}^{-1}}
\newcommand{\rextizero}{\re_{\ext_0}^{-1}}

\newcommand{\rel}{^{\approx}}
\newcommand{\yon}{\msf{y}}


%% --------------------------------------------------------------------------------

\begin{document}

%%
%% The "title" command has an optional parameter,
%% allowing the author to define a "short title" to be used in page headers.
\title{Partially Static Types and Staged Partial Evaluation in Two-Level Type Theory}

%%
%% The "author" command and its associated commands are used to define
%% the authors and their affiliations.
%% Of note is the shared affiliation of the first two authors, and the
%% "authornote" and "authornotemark" commands
%% used to denote shared contribution to the research.
\author{András Kovács}
\email{kovacsandras@inf.elte.hu}
\orcid{0000-0002-6375-9781}
\affiliation{%
  \institution{Eötvös Loránd University}
  \country{Hungary}
  \city{Budapest}
}

%% The abstract is a short summary of the work to be presented in the
%% article.
\begin{abstract}
  Two-level type theory (2LTT) is a system that can be used for two-stage
  compilation, which allows dependent types both in metaprograms and in
  generated code output. Partially static data types are used in staged
  compilation in situations where only some parts of structures are known at
  compile time. We investigate partially static types and partial evaluation in
  the context of 2LTT. A key question is whether we can mechanically obtain
  partially static types from non-staged data types, and likewise lift
  non-staged operations to operations on partially static data. We show that all
  types and constructions in a Martin-Löf type theory (MLTT) can be given a
  partially static interpretation. Moreover, this interpretation yields a staged
  partial evaluator for MLTT, defined within 2LTT, which can be used to compile
  embedded MLTT syntax to normalized object code. We show that the
  interpretation can be used to partially mechanize the construction of staged
  normalizers for certain algebras. Finally, we compare and investigate free
  extensions of algebras (``frex'') in 2LTT as an alternative generic approach to
  partially static types.
\end{abstract}

\begin{CCSXML}
<ccs2012>
   <concept>
       <concept_id>10003752.10003790.10011740</concept_id>
       <concept_desc>Theory of computation~Type theory</concept_desc>
       <concept_significance>500</concept_significance>
       </concept>
   <concept>
       <concept_id>10011007.10011006.10011041.10011047</concept_id>
       <concept_desc>Software and its engineering~Source code generation</concept_desc>
       <concept_significance>500</concept_significance>
       </concept>
 </ccs2012>
\end{CCSXML}

\ccsdesc[500]{Theory of computation~Type theory}
\ccsdesc[500]{Software and its engineering~Source code generation}
\keywords{type theory, two-level type theory, staged compilation, partial evaluation}
\maketitle

\section{Introduction}\label{sec:introduction}

In most practical programming languages it is possible to write code-generating
code by simply manipulating strings or syntax trees, but this tends to be
tedious, unsafe and non-composable. The purpose of \emph{staged compilation} is
to support metaprogramming with better ergonomics and more safety guarantees. In
two-stage systems, user-written metaprograms are executed at compile time to
produce code that is processed in further compilation.

\emph{Two-level type theory} (2LTT) is a framework which supports two-stage
compilation with strong safety and correctness properties. It is also compatible
with a wide range of object-level and meta-level language features; in
particular it allows dependent types at both levels. 2LTT was originally used in
a purely mathematical context in synthetic homotopy theory \cite{twolevel}, but
it can be also applied in staged compilation \cite{2ltt-staged}.

A common application of staging is to use embedded languages without
interpretative overhead, using \emph{staged interpretation}, or going further,
to optimize embedded programs or partially evaluate them at compile
time. \emph{Partially static data} is commonly used in staged interpreters and
partial evaluators \cite{TODO}. Such data contains a mixture of expressions of
the object language and actual structured data that is computed at compile time.
A simple example is a compile-time list whose elements are object-level expressions.

In this paper, we investigate partially static types and partial evaluation in
the context of 2LTT. In this context, we have a full-powered mathematical
language at our disposal on the meta level. Our object language is a bit simpler
but still highly expressive. Hence, we can internalize much of the formal
reasoning about staging in 2LTT itself, and internalize constructions which were
not feasible in systems with weaker type systems.

\subsection{Overview \& Contributions}\label{sec:overview}
\begin{itemize}
\item In \textbf{Section \ref{sec:2ltt}} we describe the specific variant of 2LTT used
      in this paper and give a short overview of its staging features.
\item In \textbf{Section \ref{sec:ps-interp}} we consider deep embeddings of a
      Martin-Löf type theory (MLTT) into 2LTT, and describe its staged
      interpretation and partial evaluation. The former maps embedded MLTT syntax to
      object-level code. The latter is similar, but it can also perform
      $\beta\eta$-normalization and could serve as a basis for more sophisticated
      compile-time optimization. We define a model of MLTT where each type is
      interpreted as an object-level type (``dynamic'' type) together with a
      partially static type, and there is an embedding of the former into the
      latter.  The interpretation of MLTT syntax in this model corresponds to staged
      partial evaluation.
      \begin{itemize}
        \item  From another point of view, this model explains how to extend all MLTT types
               with neutral values representing object-level expressions, obtaining partially
               static types.
        \item Since MLTT terms are also modeled, any function in MLTT can be
              interpreted as a function operating on partially static types in 2LTT.
      \end{itemize}
\item In \textbf{Section \ref{sec:ps-alg}} we describe an application of the
  partially static interpretation in normalization for algebraic structures.  In
  some cases it is possible to replace an algebra with an equivalent algebra,
  but in which some or all equations hold definitionally. Then, we can take the
  partially static interpretation of such strict algebras. This yields a staged
  normalizer with respect to the definitional equations.

\item In \textbf{Section \ref{sec:frex}} we look at free extensions of algebras,
  called \emph{frex} in prior literature
  \cite{DBLP:journals/pacmpl/YallopGK18}. A key advantage here is that algebras
  can be extended with decidable or even finite sets of variables, which makes
  it possible to compute more normal forms. For example, normalization for free
  groups require decidable equality of variables. Some subtleties arise here
  from the 2LTT staging setup and the fact that we internally verify algebraic
  laws. Free extensions always exist, but only those can be actually interpreted in
  object-level algebras which have a ``fully normalized'' presentation that does not
  use quotients.
\end{itemize}

\section{Overview of Two-Level Type Theory}\label{sec:2ltt}

\section{Staged Interpretation and Normalization for Martin-Löf Type Theory}\label{sec:ps-interp}

\section{Partially Static Types in Normalization of Algebras}\label{sec:ps-alg}

\section{Free Extensions of Algebras in Two-Level Type Theory}\label{sec:frex}

\section{Related Work \& Conclusions}\label{sec:conclusion}








\interlinepenalty=10000
\bibliography{references}

\end{document}
\endinput
