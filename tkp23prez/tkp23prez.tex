

\documentclass[dvipsnames,aspectratio=169]{beamer}
\usetheme{Madrid}

%% kill footline
\setbeamertemplate{footline}[frame number]{}
\setbeamertemplate{navigation symbols}{}
\setbeamertemplate{footline}{}

%% bibliography
\bibliographystyle{alpha}
\setbeamerfont{bibliography item}{size=\footnotesize}
\setbeamerfont{bibliography entry author}{size=\footnotesize}
\setbeamerfont{bibliography entry title}{size=\footnotesize}
\setbeamerfont{bibliography entry location}{size=\footnotesize}
\setbeamerfont{bibliography entry note}{size=\footnotesize}
\setbeamertemplate{bibliography item}{}

%% kill ball enumeration
\setbeamertemplate{enumerate items}[circle]
\setbeamertemplate{section in toc}[circle]

%% kill block shadows
\setbeamertemplate{blocks}[rounded][shadow=false]
\setbeamertemplate{title page}[default][colsep=-4bp,rounded=true]

%% kill ball itemize
\setbeamertemplate{itemize items}[circle]

%% --------------------------------------------------------------------------------

\usepackage[utf8]{inputenc}
\usepackage{hyperref}
\usepackage{amsmath}
\usepackage{cite}
\usepackage{amsthm}
\usepackage{amssymb}
\usepackage{amsfonts}
\usepackage{mathpartir}
\usepackage{scalerel}
\usepackage{stmaryrd}
\usepackage{bm}
\usepackage{graphicx}

\hypersetup{
  colorlinks=true,
  linkcolor=blue,
  urlcolor=blue}

%% --------------------------------------------------------------------------------

\renewcommand{\mit}[1]{\mathit{#1}}
\newcommand{\msf}[1]{\mathsf{#1}}
\newcommand{\mbb}[1]{\mathbb{#1}}
\newcommand{\mbf}[1]{\mathbf{#1}}
\newcommand{\bs}[1]{\boldsymbol{#1}}
\newcommand{\wh}[1]{\widehat{#1}}
\newcommand{\ext}{\triangleright}
\newcommand{\Code}{\msf{Code}}
\newcommand{\El}{\msf{El}}
\newcommand{\lam}{\msf{lam}}
\newcommand{\app}{\msf{app}}
\newcommand{\NatElim}{\msf{NatElim}}
\newcommand{\y}{\msf{y}}

\newcommand{\Lift}{{\Uparrow}}
\newcommand{\spl}{{\sim}}
\newcommand{\qut}[1]{{<}#1{>}}

\newcommand{\mbbc}{\mbb{C}}
\newcommand{\mbbo}{\mbb{O}}
\newcommand{\ob}{_\mbbo}

\newcommand{\U}{\msf{U}}
\newcommand{\Con}{\msf{Con}}
\newcommand{\Sub}{\msf{Sub}}
\newcommand{\Ty}{\msf{Ty}}
\newcommand{\Tm}{\msf{Tm}}
\newcommand{\Cono}{\msf{Con}_{\mbbo}}
\newcommand{\Subo}{\msf{Sub}_{\mbbo}}
\newcommand{\Tyo}{\msf{Ty}_{\mbbo}}
\newcommand{\Tmo}{\msf{Tm}_{\mbbo}}
\newcommand{\hCon}{\wh{\msf{Con}}}
\newcommand{\hSub}{\wh{\msf{Sub}}}
\newcommand{\hTy}{\wh{\msf{Ty}}}
\newcommand{\hTm}{\wh{\msf{Tm}}}

\newcommand{\p}{\mathsf{p}}
\newcommand{\q}{\mathsf{q}}

\newcommand{\refl}{\msf{refl}}
\newcommand{\Bool}{\msf{Bool}}
\newcommand{\true}{\msf{true}}
\newcommand{\false}{\msf{false}}
\newcommand{\True}{\msf{True}}
\newcommand{\False}{\msf{False}}
\newcommand{\List}{\msf{List}}
\newcommand{\nil}{\msf{nil}}
\newcommand{\cons}{\msf{cons}}
\newcommand{\Nat}{\msf{Nat}}
\newcommand{\zero}{\msf{zero}}
\newcommand{\suc}{\msf{suc}}
\renewcommand{\tt}{\msf{tt}}
\newcommand{\fst}{\msf{fst}}
\newcommand{\snd}{\msf{snd}}
\newcommand{\mylet}{\msf{let}}
\newcommand{\emptycon}{\scaleobj{.75}\bullet}
\newcommand{\id}{\msf{id}}

\newcommand{\Set}{\mathsf{Set}}
\newcommand{\Prop}{\mathsf{Prop}}
\newcommand{\Rep}{\msf{Rep}}
\newcommand{\blank}{{\mathord{\hspace{1pt}\text{--}\hspace{1pt}}}}
\newcommand{\emb}[1]{\ulcorner#1\urcorner}

\newcommand{\Stage}{\msf{Stage}}
\newcommand{\hato}{\bm\hat{\mbbo}}
\newcommand{\ev}{\mbb{E}}
\newcommand{\re}{\mbb{R}}

\theoremstyle{remark}
\newtheorem{notation}{Notation}

\newcommand{\whset}{\wh{\Set}}
\newcommand{\rexti}{\re_{\ext_1}^{-1}}
\newcommand{\rextizero}{\re_{\ext_0}^{-1}}

\newcommand{\rel}{^{\approx}}
\newcommand{\yon}{\msf{y}}

\newcommand{\Vect}{\msf{Vec}}
\newcommand{\msA}{\msf{A}}
\newcommand{\msB}{\msf{B}}
\newcommand{\mst}{\msf{t}}
\newcommand{\msu}{\msf{u}}
\newcommand{\msv}{\msf{v}}
\newcommand{\msx}{\msf{x}}
\newcommand{\msy}{\msf{y}}
\newcommand{\msz}{\msf{z}}
\newcommand{\msas}{\msf{as}}
\newcommand{\msbs}{\msf{bs}}
\newcommand{\msxs}{\msf{xs}}
\newcommand{\msys}{\msf{ys}}
\newcommand{\msff}{\msf{f}}
\newcommand{\msg}{\msf{g}}
\newcommand{\msh}{\msf{h}}
\newcommand{\msa}{\msf{a}}
\newcommand{\msn}{\msf{n}}

%% --------------------------------------------------------------------------------


\title{Staged Compilation with Two-Level Type Theory}
\author{András Kovács}
\institute{
  {Eötvös Loránd University}
}
\date{18 January 2022, TKP Workshop}
\begin{document}

\frame{\titlepage}

\begin{frame}{Staged Compilation}

\begin{block}{}
\textbf{Staged compilation} is about writing code-generating code with good ergonomics
and safety guarantees.
\end{block}
\vspace{1em}
\pause

Examples:
\begin{itemize}
  \item (Typed) Template Haskell.
  \item C++ templates.
  \item Rust traits, macros \& generics.
\end{itemize}
\vspace{1em}
\pause

Motivations:
\begin{itemize}
  \item Low-cost abstraction.
  \item DSLs.
  \item Inlining \& fusion with strong guarantees.
\end{itemize}

\end{frame}

\begin{frame}{Two-Level Type Theory (2LTT)}

Comes from \textbf{homotopy type theory}:
\begin{itemize}
  \item \emph{Voevodsky: A simple type system with two identity types}.
  \item \emph{Annekov, Capriotti, Kraus, Sattler: Two-Level Type Theory and Applications}.
  \item Motivation: meta-programming and modular axioms for HoTT.
\end{itemize}
\vspace{1em}
\pause

2LTT is directly applicable to two-stage compilation.
\vspace{1em}
\pause

Features:
\begin{enumerate}
\item Integrates a compile-time (``meta'') language and a runtime (``object'') language.
\pause
\item Guaranteed well-typing of code output, guaranteed well-staging.
\pause
\item Supports a wide range of runtime and meta-languages.
  \begin{itemize}
  \pause
  \item \alert{Including dependent types}.
  \end{itemize}
\pause
\item Supports efficient \emph{staging-by-evaluation}.
\end{enumerate}
\vspace{1em}
\end{frame}

\begin{frame}{This talk}

This talk mostly contains \textbf{small programming examples}.
\vspace{1em}
\pause

There is a \textbf{paper} and an \textbf{implementation}:
\begin{itemize}
  \item AK: \emph{Staged Compilation with Two-Level Type Theory}, in the proceedings of ICFP 2022.
\end{itemize}
\vspace{1em}
\pause

For a \textbf{tutorial} and \textbf{larger programming examples}, see the implementation.
\vspace{1em}
\pause

For \textbf{formal details}, see the paper.

\end{frame}

\begin{frame}{Rules of 2LTT}

  \begin{enumerate}
    \item Two universes $\U_0$, $\U_1$, closed under arbitrary type formers.
      \begin{itemize}
        \item $\U_0$ is the universe of runtime (object-level) types.
        \item $\U_1$ is the universe of compile-time (meta-level) types.
      \end{itemize}
    \pause
    \item All type/term formers and eliminators stay within the same universe.
    \pause
    \item \emph{Lifting:} for $\msA : \U_0$, we have $\Lift \msA : \U_1$.
    \pause
    \item \emph{Quoting:} for $\msA : \U_0$ and $\mst : \msA$, we have $\qut{\mst} : \Lift \msA$.
    \pause
    \item \emph{Splicing:} for $\mst : \Lift \msA$, we have $\spl{\mst} : \msA$.
    \pause
    \item $\qut{\spl{\mst}} \equiv \mst$ and $\spl{\qut{\mst}} \equiv \mst$.
  \end{enumerate}
\vspace{1em}
\pause

\begin{block}{}
\textbf{Staging} runs all metaprograms in splices and inserts their result
in the code output.
\end{block}

\end{frame}

\begin{frame}{Inlined definitions}

Staging input:
\begin{alignat*}{4}
  &\msf{two} : \Lift \Nat_0 \\
  &\msf{two} = \qut{\suc_0\,(\suc_0\,\zero_0)} \\
  & \\
  & \msff : \Nat_0 \to \Nat_0 \\
  & \msff = \lambda\,\msx.\, \msx + \spl{\msf{two}}
\end{alignat*}
\pause
Output:
\begin{alignat*}{4}
  & \msff : \Nat_0 \to \Nat_0 \\
  & \msff = \lambda\,\msx.\, \msx + \suc_0\,(\suc_0\,\zero_0)
\end{alignat*}


\end{frame}

\begin{frame}{Compile-time identity function}

Input:
\begin{alignat*}{4}
  & \id : (\msA : \U_1) \to \msA \to \msA\\
  & \id = \lambda\,\msA\,\msx.\,\msx     \\
  & \\
  & \msf{idBool_0} : \Bool_0 \to \Bool_0\\
  & \msf{idBool_0} = \lambda\,\msx.\,\spl(\id\,(\Lift \Bool_0)\,\qut{\msx})
\end{alignat*}
\pause
Output:
\begin{alignat*}{4}
  & \msf{idBool_0} : \Bool_0 \to \Bool_0\\
  & \msf{idBool_0} = \lambda\,\msx.\,\msx
\end{alignat*}

\end{frame}

\begin{frame}{An alternative identity function}

%%   \begin{columns}
%% \begin{column}{0.5\textwidth}
%%    some text here some text here some text here some text here some text here
%% \end{column}
%% \begin{column}{0.5\textwidth}  %%<--- here
%%     \begin{center}
%%      \includegraphics[width=0.5\textwidth]{image1.jpg}
%%      \end{center}
%% \end{column}
%% \end{columns}

Input:
\begin{columns}
\begin{column}{0.5\textwidth}
\begin{alignat*}{4}
  & \id_\Lift : (\msA : \Lift\U_0) \to \Lift\,\spl{\msA} \to \Lift\,\spl{\msA}\\
  & \id_\Lift = \lambda\,\msA\,\msx.\,\msx     \\
  & \\
  & \msf{idBool_0} : \Bool_0 \to \Bool_0\\
  & \msf{idBool_0} = \lambda\,\msx.\,\spl(\id_\Lift\,\qut{\Bool_0}\,\qut{\msx})
\end{alignat*}
\end{column}
\pause
\begin{column}{0.5\textwidth}
\emph{Note that}
\vspace{-1em}
\begin{alignat*}{4}
  & \msA &&: \Lift\U_0 \\
  & \spl\msA &&: \U_0 \\
  & \Lift\,\spl\msA &&: \U_1\\
  & \qut{\msx} &&: \Lift\Bool_0\\
  & \qut{\msx} &&: \Lift\,\spl\qut{\Bool_0}
\end{alignat*}

\end{column}
\end{columns}
\vspace{1em}
\pause
Output:
\begin{alignat*}{4}
  & \msf{idBool_0} : \Bool_0 \to \Bool_0\\
  & \msf{idBool_0} = \lambda\,\msx.\,\msx
\end{alignat*}

%% Input:
%% \begin{alignat*}{4}
%%   & \id_\Lift : (\msA : \Lift\U_0) \to \Lift\,\spl{\msA} \to \Lift\,\spl{\msA}\\
%%   & \id_\Lift = \lambda\,\msA\,\msx.\,\msx     \\
%%   & \\
%%   & \msf{idBool_0} : \Bool_0 \to \Bool_0\\
%%   & \msf{idBool_0} = \lambda\,\msx.\,\spl(\id_\Lift\,\qut{\Bool_0}\,\qut{\msx})
%% \end{alignat*}
%% \pause
%% Output:
%% \begin{alignat*}{4}
%%   & \msf{idBool_0} : \Bool_0 \to \Bool_0\\
%%   & \msf{idBool_0} = \lambda\,\msx.\,\msx
%% \end{alignat*}


\end{frame}


\begin{frame}{$\msf{map}$ with inlining}

Input:
\begin{alignat*}{4}
  & \msf{inlMap} : \{\msA\,\msB : \Lift \U_0\} \to (\Lift\,\spl\msA \to \Lift\,\spl\msB) \to \Lift(\msf{List_0}\,\spl\msA) \to \Lift(\msf{List_0}\,\spl\msB) \\
  & \msf{inlMap} = \lambda\,\msff\,\msas.\,\qut{\msf{foldr_0}\,
    (\lambda\,\msa\,\msbs.\,\cons_0\,\spl(\msff\,\qut{\msa})\,\msbs)\,
    \nil_0\,
    \spl{\msas}
    }\\
  & \\
  & \msf{f} : \List_0\,\Nat_0 \to \List_0\,\Nat_0\\
  & \msf{f} = \lambda\,\msxs.\,\,\spl(\msf{inlMap}\,(\lambda\,\msf{n}.\,\qut{\spl\msf{n} + 2})
     \,\qut{\msxs})
\end{alignat*}
\pause
Output:
\begin{alignat*}{4}
  & \msf{f} : \List_0\,\Nat_0 \to \List_0\,\Nat_0\\
  & \msf{f} = \lambda\,\msxs.\, \msf{foldr}_0\,(\lambda\,\msa\,\msbs.\,\cons_0\,(\msa + 2)\,\msbs)\,\nil_0\,\msxs
\end{alignat*}

\end{frame}

\begin{frame}{Inference for staging operations}

Lifting preserves negative types up to definitional isomorphism:
\begin{alignat*}{4}
  \Lift\,\top_0 &\simeq \top_1 \\
  \Lift\,((\msa : \msA) \to \msB\,\msa) &\simeq ((\msa : \Lift\msA) \to \Lift(\msB\,\spl\msa)) \\
  \Lift\,((\msa : \msA) \times \msB\,\msa) &\simeq ((\msa : \Lift\msA) \times \Lift(\msB\,\spl\msa))
\end{alignat*}
\pause
We can use \textbf{bidirectional elaboration} \& \textbf{coercive subtyping along isos} to
infer most quotes and splices.
\begin{alignat*}{4}
  & \msf{inlMap} : \{\msA\,\msB : \Lift \U_0\} \to (\Lift\,\msA \to \Lift\,\msB) \to \Lift(\msf{List_0}\,\msA) \to \Lift(\msf{List_0}\,\msB) \\
  & \msf{inlMap} = \lambda\,\msff.\,\msf{foldr_0}\,
    (\lambda\,\msa\,\msbs.\,\cons_0\,(\msff\,\msa)\,\msbs)\,
    \nil_0
    \\
  & \\
  & \msf{f} : \List_0\,\Nat_0 \to \List_0\,\Nat_0\\
  & \msf{f} = \msf{inlMap}\,(\lambda\,\msf{n}.\,\msf{n} + 2)
\end{alignat*}



\end{frame}

\begin{frame}{Staging types}

Input:
\begin{alignat*}{4}
  & \rlap{$\Vect : \Nat_1 \to \Lift \U_0 \to \Lift \U_0$}\\
  & \Vect\,&&\zero_1       \,&&\msA = \qut{\top_0}\\
  & \Vect\,&&(\suc_1\,\msn)\,&&\msA = \qut{\spl \msA \times \spl(\Vect\,\msn\,\msA)}\\
  & \\
  & \rlap{$\msf{Tuple3} : \U_0 \to \U_0$} \\
  & \rlap{$\msf{Tuple3}\,\msA = \spl(\Vect\,3\,\qut{\msA})$}
\end{alignat*}
\pause
Output:
\begin{alignat*}{4}
  &\msf{Tuple3} : \U_0 \to \U_0 \\
  &\msf{Tuple3}\,\msA = \msA \times (\msA \times (\msA \times \top_0))
\end{alignat*}

\end{frame}

\begin{frame}{$\msf{map}$ for $\Vect$}

Input:
\begin{alignat*}{3}
  &\msf{map} : \{\msA\,\msB : \Lift \U_0\} \to
  (\msn : \Nat_1) \to (\Lift\,\spl\msA \to \Lift\,\spl\msB) \\
  & \hspace{7.7em} \to \Lift(\Vect\,\msn\,\msA) \to \Lift(\Vect\,\msn\,\msB)\\
  &\msf{map}\,\zero_1\hspace{1.25em}\msff\,\msas = \qut{\tt_0} \\
  &\msf{map}\,(\suc_1\,\msn)\,\msff\,\msas =
    \qut{(\spl(\msff\,\qut{\fst_0\,\spl \msas}),\,\spl(\msf{map}\,\msn\,\msff\,\qut{\snd_0\,\spl \msas}))} \\
  & \\
  & \msf{f} : \spl(\Vect\,2\,\qut{\Nat_0}) \to \spl(\Vect\,2\,\qut{\Nat_0}) \\
  & \msf{f}\,\msxs = \spl(\msf{map}\,2\,(\lambda\,\msx.\,\qut{\spl\msx + 2})\,\qut{\msxs})
\end{alignat*}
\pause
Output:
\begin{alignat*}{3}
  & \msf{f} : \Nat_0 \times (\Nat_0 \times \top_0) \to \Nat_0 \times (\Nat_0 \times \top_0) \\
  & \msf{f}\,\msxs = (\fst_0\,\msxs + 2,\,(\fst_0\,(\snd_0\,\msxs) + 2,\,\tt_0))
\end{alignat*}

\end{frame}

\begin{frame}{}

More in the paper \& implementation:
\begin{itemize}
  \item Correctness of staging.
  \item Staged foldr/build fusion.
  \item Well-typed staged STLC interpreter.
  \item Monadic let-insertion.
\end{itemize}
\vspace{1em}
\pause

Possible future research:
\begin{itemize}
  \item Staging to low-level (e.g.\ first-order) languages.
  \item Staged fusion.
  \item Partially static data types.
\end{itemize}

\end{frame}

\begin{frame}{}

\begin{center}
  \Large {Thank you!}
\end{center}

\end{frame}



\end{document}
