%% build: latexmk -pdf -pvc -xelatex prez.tex

\documentclass[dvipsnames,10pt,aspectratio=169]{beamer}
\usetheme{metropolis}           % Use metropolis theme

\usepackage{xcolor}
\usepackage{amssymb}
\usepackage{amsmath}
\usepackage{amsfonts}
\usepackage{fontspec}
\usepackage{proof}
\usepackage{tikz-cd}
\usepackage{mathpartir}
\usepackage{stmaryrd}
\usepackage{hyperref}
\hypersetup{colorlinks,linkcolor=blue,urlcolor=blue}
\setmonofont[Scale=0.7]{DejaVu Sans Mono}


\bibliographystyle{alpha}
\setbeamerfont{bibliography item}{size=\footnotesize}
\setbeamerfont{bibliography entry author}{size=\footnotesize}
\setbeamerfont{bibliography entry title}{size=\footnotesize}
\setbeamerfont{bibliography entry location}{size=\footnotesize}
\setbeamerfont{bibliography entry note}{size=\footnotesize}
\setbeamertemplate{bibliography item}{}

\newcommand{\Set}[1]{\mathsf{Set_{#1}}}
\newcommand{\Seti}{\mathsf{Set}}
\newcommand{\refl}{\mathsf{refl}}
\newcommand{\Con}{\mathsf{Con}}
\newcommand{\Ty}{\mathsf{Ty}}
\newcommand{\Tm}{\mathsf{Tm}}
\newcommand{\Sub}{\mathsf{Sub}}
\newcommand{\emptycon}{\scaleobj{.75}\bullet}
\renewcommand{\U}{\mathsf{U}}
\newcommand{\El}{\mathsf{El}}
\newcommand{\id}{\mathsf{id}}
\newcommand{\ext}{\triangleright}
\newcommand{\blank}{\mathord{\hspace{1pt}\text{--}\hspace{1pt}}}
\newcommand{\mi}[1]{\mathit{#1}}
\newcommand{\p}{\mathsf{p}}
\newcommand{\q}{\mathsf{q}}
\newcommand{\Id}{\mathsf{Id}}
\newcommand{\Nat}{\mathsf{Nat}}
\newcommand{\Bool}{\mathsf{Bool}}
\newcommand{\true}{\mathsf{true}}
\newcommand{\false}{\mathsf{false}}
\newcommand{\up}{\uparrow}
\newcommand{\down}{\downarrow}
\newcommand{\Lift}{\mathsf{Lift}}
\renewcommand{\tt}{\mathsf{tt}}
\newcommand{\Acc}{\mathsf{Acc}}
\newcommand{\acc}{\mathsf{acc}}
\newcommand{\Lvl}{\mathsf{Lvl}}
\newcommand{\Code}{\mathsf{Code}}
\newcommand{\msf}[1]{\mathsf{#1}}
\newcommand{\uir}{\msf{U^{IR}}}
\newcommand{\elir}{\msf{El^{IR}}}
\newcommand{\ult}{\U_{<}}
\newcommand{\mkMor}{\msf{mk}\!_<}
\newcommand{\unMor}{\msf{un}\!_<}
\newcommand{\mkLvl}{\msf{mk}_{\Lvl}}
\newcommand{\unLvl}{\msf{un}_{\Lvl}}

\newcommand{\qtm}[1]{\langle #1\rangle}


%% --------------------------------------------------------------------------------

\title{Using Two-Level Type Theory for Staged Compilation}
\date{18 January 2022, Programming Languages and Compilers Department Workshop}
\author{\normalsize{\vspace{-1em}\textbf{András Kovács}\footnote{The author was supported by the European Union,
co-financed by the European Social Fund (EFOP-3.6.3-VEKOP-16-2017-00002).\vspace{0.5em}}}}
%% \institute{ELTE, Programozási Nyelvek és Fordítóprogramok tanszék\\Informatikai logika kutatócsoport\\ Témavezető: Kaposi Ambrus}
\begin{document}
\maketitle

%% ------------------------------------------------------------

\begin{frame}{Staged Compilation}

Programs which generate programs.
\vspace{1em}
\pause

With extra features:
\begin{itemize}
  \item Guaranteed well-typed output.
  \item Lightweight syntax.
  \item Seamless integration of object-level and meta-level.
\end{itemize}
\pause

Price to pay: some metaprograms are not expressible.

\end{frame}

%% ------------------------------------------------------------

\begin{frame}{Two-Level Type Theory}

Voevodsky's Homotopy Type System \cite{voevodsky2013simple}, 2LTT \cite{twolevel}.
\vspace{1em}

Original motivation: metaprogramming, with HoTT as object theory
\vspace{1em}
\pause

Turns out to implement two-stage programming:
\begin{itemize}
  \item Works for wide range of theories
  \pause
  \item Simple rules
  \pause
  \item Fast staging with NbE
  \pause
  \item Nice model theory and standard semantics
\end{itemize}

\end{frame}


\begin{frame}[fragile]{Rules}

\begin{itemize}
    \item $\U_0$ (object-level) and $\U_1$ (meta-level) universes, both closed
      under arbitrary type formers.
    \item Constructors and eliminators \alert{stay within} universes!
    \pause
    \item For $A : \U_0$, we have $\Code\,A : \U_1$
    \pause
    \item Quoting: for $A : \U_0$ and $t : A$ we have $\qtm{t} : \Code\,A$
    \pause
    \item Splicing: for $t : \Code\,A$, we have $\sim\!t : A$
    \pause
    \item $\qtm{\sim\!t} = t$
    \item $\sim\!\qtm{t} = t$
\end{itemize}
\vspace{1em}
\pause

Staging: computing away every meta-level subterm in an object-level term.
\end{frame}

\begin{frame}{Identity functions}

\begin{alignat*}{3}
  &\msf{id_0} &&: (A : \U_0) \to A \to A\\
  &\msf{id_0}\,\Bool_0\,\true_0 &&: \Bool_0
\end{alignat*}
\end{frame}

\begin{frame}{Identity functions}

\begin{alignat*}{3}
  &\msf{id_1} &&: (A : \U_1) \to A \to A\\
  &\msf{id_1}\,\Bool_1\,\true_1 &&: \Bool_1\\
  &\msf{id_1}\,(\Code\,\Bool_0)\,\qtm{\true_0} &&: \Code\,\Bool_0
\end{alignat*}
\end{frame}

\begin{frame}{Quantifying over $\Code\,\U_0$}

Inlined object-level $\msf{map}$:
\begin{alignat*}{3}
  &\msf{map} : (A\,B : \Code\,\U_0) \to (\Code(\sim\!A) \to \Code(\sim\!B)) \to \Code(\msf{List_0}\,(\sim\!A)) \to \Code(\msf{List_0}\,(\sim\!B)))\\
  & \msf{map}\,\qtm{\Nat_0}\,\qtm{\Nat_0}\,(\lambda\,x. \qtm{(\sim\!x) + 10}) : \Code(\msf{List_0}\,\Nat_0) \to \Code(\msf{List_0}\,\Nat_0))
\end{alignat*}
\end{frame}

\begin{frame}{Staging Types, Inference}

\begin{alignat*}{5}
  & \rlap{$\msf{Vec} : \Nat_1 \to \Code\,\U_0 \to \Code\,\U_0$}\\
  & \msf{Vec}\,&&\msf{zero}_1\,&&A &&= \qtm{\top_0}\\
  & \msf{Vec}\,&&(\msf{suc}_1\,n)\,&&A &&= \qtm{(\sim\!A)\,\,\times_0 \sim\!(\msf{Vec}\,n\,A)}
\end{alignat*}
\pause
With annotation inference:
\begin{alignat*}{5}
  & \rlap{$\msf{Vec} : \Nat_1 \to \U_0 \to \U_0$}\\
  & \msf{Vec}\,&&\msf{zero}_1\,&&A &&= \top_0\\
  & \msf{Vec}\,&&(\msf{suc}_1\,n)\,&&A &&= A \times_0 \msf{Vec}\,n\,A
\end{alignat*}

Demo: well-typed staged STLC interpreter, all annotations inferred.
\end{frame}

\begin{frame}{Weak Object Language + Strong Metalanguage}

Simpler object theory $\to$ better performance
\vspace{1em}

2LTT recovers features for free: universe, $\Pi$, $\Sigma$
\vspace{1em}

\end{frame}

\begin{frame}{Simple TT at Object-Level}

A system for monomorphization.
\vspace{1em}
\pause

Universe of object types: $\Ty : \U_1$, $\Code : \Ty \to \U_1$.
\begin{alignat*}{3}
  &\msf{id}  &&: (A : \Ty) \to \Code(A \to A)
\end{alignat*}
\pause

  %% &\msf{id'} &&: (A : \Ty) \to \Code\,A \to \Code\,A

Higher-rank polymorphism via inlining:
\[ \msf{poly} : ((A : \Ty) \to \Code\,A \to \Code\,A) \to (\Code\,\Bool,\,\Code\,\msf{Int}) \]

What we can't do: store polymorphic functions in object-level data.

\end{frame}


\begin{frame}{First-Order Functions at Object-Level}

A system for closure-free compilation.
\vspace{1em}
\pause

%% Splitting $\Ty$ to value and computation types.
%% \begin{itemize}
%% \item Value types: $\Ty_V$
%% \item Computation types: $\Ty_C$
%% \item $\Ty_V$ closed under inductive types.
%% \item $\Ty_C$ closed under functions with $\Ty_v$ arguments.
%% \end{itemize}
%% \pause

2LTT gives us higher-order functions for free.
\vspace{1em}
\pause

What we can't do: store functions in object-level data.
\vspace{1em}
\pause

Surprisingly expressive.

\end{frame}

\begin{frame}{Memory Layout-Indexed Types at Object-Level}

A system for layout polymorphism (levity polymorphism).

\[ \msf{id} : (L : \msf{Layout}) \to (A : \U_0\,L) \to \Code(A \to A) \]

Staging computes layouts to closed canonical values.

\end{frame}

\begin{frame}{Standard Semantics}

Presheaves over the syntactic category of object theory.
\vspace{1em}
\pause

$\Code$ is ``dependent'' Yoneda-embedding.
\vspace{1em}
\pause

Choice of morphisms in the base category:
\begin{itemize}
  \pause
  \item Substitutions: only generative staging
  \pause
  \item Weakenings: allows $\Code$ analysis, but fewer object theories
\end{itemize}
\end{frame}

\begin{frame}{Demos}

\url{https://github.com/AndrasKovacs/implicit-fun-elaboration/tree/staging}
\vspace{1em}

WIP: \url{https://github.com/AndrasKovacs/staged}

\end{frame}

\begin{frame}[allowframebreaks]{References}
  \bibliography{references}
\end{frame}

\end{document}
