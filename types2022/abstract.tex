
\documentclass{easychair}
\raggedbottom

\usepackage{doc}
\usepackage{xcolor}
\usepackage{mathpartir}
\usepackage{todonotes}
\presetkeys{todonotes}{inline}{}
\usepackage{scalerel}
\usepackage{bm}
\usepackage{amssymb}

\renewcommand{\mit}[1]{\mathit{#1}}
\newcommand{\msf}[1]{\mathsf{#1}}
\newcommand{\mbb}[1]{\mathbb{#1}}
\newcommand{\mbf}[1]{\mathbf{#1}}
\newcommand{\bs}[1]{\boldsymbol{#1}}
\newcommand{\wh}[1]{\widehat{#1}}
\newcommand{\ext}{\triangleright}
\newcommand{\Code}{\msf{Code}}
\newcommand{\El}{\msf{El}}
\newcommand{\lam}{\msf{lam}}
\newcommand{\app}{\msf{app}}
\newcommand{\NatElim}{\msf{NatElim}}
\newcommand{\y}{\msf{y}}
\newcommand\whobj{\wh{\msf{Obj}}}

\newcommand{\Lift}{{\Uparrow}}
\newcommand{\spl}{{\sim}}
\newcommand{\qut}[1]{\langle #1\rangle}

\newcommand{\mbbc}{\mbb{C}}
\newcommand{\mbbo}{\mbb{O}}
\newcommand{\ob}{_\mbbo}

\newcommand{\U}{\msf{U}}
\newcommand{\Con}{\msf{Con}}
\newcommand{\Sub}{\msf{Sub}}
\newcommand{\Ty}{\msf{Ty}}
\newcommand{\Tm}{\msf{Tm}}
\newcommand{\Cono}{\msf{Con}_{\mbbo}}
\newcommand{\Subo}{\msf{Sub}_{\mbbo}}
\newcommand{\Tyo}{\msf{Ty}_{\mbbo}}
\newcommand{\Tmo}{\msf{Tm}_{\mbbo}}
\newcommand{\hCon}{\wh{\msf{Con}}}
\newcommand{\hSub}{\wh{\msf{Sub}}}
\newcommand{\hTy}{\wh{\msf{Ty}}}
\newcommand{\hTm}{\wh{\msf{Tm}}}

\newcommand{\p}{\msf{p}}
\newcommand{\q}{\msf{q}}

\newcommand{\Bool}{\msf{Bool}}
\newcommand{\true}{\msf{true}}
\newcommand{\false}{\msf{false}}
\newcommand{\True}{\msf{True}}
\newcommand{\False}{\msf{False}}
\newcommand{\List}{\msf{List}}
\newcommand{\nil}{\msf{nil}}
\newcommand{\cons}{\msf{cons}}
\newcommand{\Nat}{\msf{Nat}}
\newcommand{\zero}{\msf{zero}}
\newcommand{\suc}{\msf{suc}}
\newcommand{\ttt}{\msf{tt}}
\newcommand{\fst}{\msf{fst}}
\newcommand{\snd}{\msf{snd}}
\newcommand{\mylet}{\msf{let}}
\newcommand{\emptycon}{\scaleobj{.75}\bullet}
\newcommand{\id}{\msf{id}}

\newcommand{\Set}{\msf{Set}}
\newcommand{\Rep}{\msf{Rep}}
\newcommand{\blank}{{\mathord{\hspace{1pt}\text{--}\hspace{1pt}}}}
\newcommand{\emb}[1]{\ulcorner#1\urcorner}

\newcommand{\Stage}{\msf{Stage}}
\newcommand{\hato}{\bm\hat{\mbbo}}
\newcommand{\ev}{\mbb{E}}
\newcommand{\re}{\mbb{R}}

\theoremstyle{remark}
\newtheorem{notation}{Notation}

\newcommand{\whset}{\wh{\Set}}
\newcommand{\rexti}{\re_{\ext_1}^{-1}}
\newcommand{\rextizero}{\re_{\ext_0}^{-1}}
\newcommand{\rel}{^{\approx}}
\newcommand{\yon}{\msf{y}}

\title{Conservativity of Two-Level Type Theory Corresponds to Staged Compilation}

\author{Andr\'as Kov\'acs}

% Institutes for affiliations are also joined by \and,
\institute{
  E\"otv\"os Lor\'and University,
  Budapest, Hungary \\
  \email{kovacsandras@inf.elte.hu}
}

\authorrunning{Kov\'acs}
\titlerunning{Using Two-Level Type Theory for Staged Compilation}
\pagenumbering{gobble}
\begin{document}

\maketitle

%% %% abstract

%% Two-level type theory (2LTT) was originally intended as a tool to make
%% metatheoretical reasoning about constructions in homotopy type theory more
%% convenient, by internalizing an extensional metatheory in a syntactic layer. We
%% observe that variants of 2LTT formalize setups with two-stage compilation.
%% Additionally, the conservativity property of a 2LTT over the object-level theory
%% is equivalent to having a sound and stable staging algorithm, which compiles the
%% object-level fragment of 2LTT to the syntax of the object theory. In previous
%% work on 2LTT by Annekov et al. only a weak form of conservativity is shown,
%% which corresponds to staging without the soundness property. We define staging
%% as evaluation of 2LTT syntax in the presheaf model over the object-level
%% syntactic category. Operationally, this is in similar to
%% normalization-by-evaluation, except that evaluation in the meta-level fragment
%% is closed (i.e. no free variables occur) and thus more efficient. We show
%% stability and soundness of staging; the former is straightforward, and the
%% latter is given by a proof-relevant logical relation internally to the presheaf
%% model. Equivalently, staging and its soundness can be established in a single
%% step, by gluing along the functor which restricts 2LTT syntax to object-level
%% typing contexts.

Two-level type theory (2LTT) \cite{twolevel} was originally intended as a tool
to make metatheoretical reasoning about constructions in homotopy type theory
more convenient, by internalizing an extensional metatheory in a syntactic
layer.
\\\\
\textbf{2LTT as a two-stage language.} First, we observe that basic variants of
2LTT formalize two-stage compilation, where the meta-level syntactic fragment contains
static (compile-time) computations, and \emph{staging} is the algorithm which
performs all static computation, producing object-theoretic syntax as output. The
basic rules are the following. There are two universes, $\U_0$ and $\U_1$, where
$\U_0$ classifies object-level (runtime) types and $\U_1$ classifies meta-level
(compile time) types. There are three \emph{staging operations}.

\emph{Lifting:} for $A : \U_0$, we have $\Lift A : \U_1$. From the staging point
of view, $\Lift A$ is the type of metaprograms which compute runtime expressions
of type $A$.

\emph{Quoting:} for $A : \U_0$ and $t : A$, we have $\qut{t} :\,\Lift A$.  A
quoted term $\qut{t}$ represents the metaprogram which immediately yields $t$.

\emph{Splicing:} for $A : \U_0$ and $t :\,\Lift A$, we have $\spl t : A$.
During staging, the metaprogram in the splice is executed, and the resulting
expression is inserted into the output.

Quoting and splicing are definitional inverses. Also, the above
operations are the \emph{only} way of crossing between stages; all type formers
stay within a single stage, and in particular we cannot eliminate from one stage
to a different one. The staging interpretation of 2LTT remains valid with
arbitrary assumed type formers. Note that all three operations correspond to
features in existing staged systems such as MetaOCaml \cite{kiselyov14metaocaml}
or typed Template Haskell \cite{typed-th}, although none of the existing systems
support staging with dependent types.
\\\\
\textbf{Conservativity as staging.} By conservativity we mean the following.
There is an embedding morphism $\emb{\blank}$ which maps from the object theory
to the object-level syntactic fragment of 2LTT. 2LTT is conservative if
$\emb{\blank}$ is bijective on types and terms,
i.e.\ $\Ty_{\msf{Obj}}\,\Gamma\ \simeq \Tm_{\msf{2LTT}}\,\emb{\Gamma}\,\U_0$ and
$\Tm_{\msf{Obj}}\,\Gamma\,A \simeq \Tm_{\msf{2LTT}}\,\emb{\Gamma}\,\emb{A}$.

A staging algorithm consists of functions $\Stage :
\Tm_{\msf{2LTT}}\,\emb{\Gamma}\,\U_0 \to \Ty_{\msf{Obj}}\,\Gamma$ and $\Stage :
\Tm_{\msf{2LTT}}\,\emb{\Gamma}\,\emb{A} \to \Tm_{\msf{Obj}}\,\Gamma\,A$. We call
$\Stage$ \emph{stable} if $\Stage\,\circ\,\emb{\blank} = \id$, and \emph{sound}
if $\emb{\blank}\,\circ\,\Stage = \id$. Hence, conservativity is the same as
having a sound and stable staging algorithm. In \cite{twolevel}, only a weak
form of conservativity is shown, which corresponds to staging without soundness.
\\\\
\textbf{Staging by evaluation.} We define $\Stage$ as the evaluation of 2LTT
types and terms in the presheaf model over the syntactic category of the object
theory. We call this model $\whobj$. The object-level 2LTT fragment is
interpreted using sets of types and terms in the object theory. Operationally,
this yields closed evaluation for the meta-level 2LTT fragment, and we get naive
weakening for object-theoretic terms. Naive weakening can be inefficient, but in
practice it can be optimized using De Bruijn levels and delayed variable
renamings. The same efficiency issue arises in presheaf-based
normalization-by-evaluation, and the same solution applies there. However,
staging can be overall more efficient, because meta-level evaluation is closed
(no free variables occur in values).
\\\\
\textbf{Soundness.} Stability of staging follows by straightforward induction on the object theory;
soundness requires more effort. We define a \emph{restriction} morphism, which
maps from 2LTT to $\whobj$, restricting the meta-level syntactic fragment so
that it can only depend on object-level typing contexts, that is, contexts given
as $\emb{\Gamma}$ for some $\Gamma$. Then, we show soundness of staging by a
proof-relevant logical relation between the evaluation morphism from 2LTT to
$\whobj$ and the restriction morphism. We define this logical relation in the
internal language of $\whobj$, to avoid the deluge of boilerplate for showing
stability under object-theoretic substitution.

Alternatively, staging together with its soundness can be established in a
single step, by gluing along the restriction morphism. This interpretation can
be more compactly defined using synthetic Tait computability
\cite{sterlingthesis}.
\\\\
\textbf{Intensional analysis.} This means analyzing the internal structure of
object-level terms, i.e.\ values with type $\Lift\,A$. While intensional
analysis can be often simulated with deeply embedded inductive syntaxes at the
meta level, it may be more concise and convenient to use native intensional
analysis features instead. We consider the interpretation of such features in
the presheaf models.

If the object theory has parallel substitutions as syntactic morphisms,
then $\whobj$ does not support intensional analysis. For illustration, consider
decidable equality of $\Lift A$ as an intensional meta-level axiom; this is
essentially decidability of definitional equality of object terms. This
axiom does not hold in $\whobj$, because inequality of object-level terms
is not stable under substitution: inequal variables can be mapped to equal
terms.

However, we may choose to only have \emph{weakenings} as morphisms in the object
syntax. In this case, decidable equaity for $\Lift A$ holds in $\whobj$, since
term inequality is stable under weakening. As a trade-off, if there is no notion
of substitution in the specification of the object theory, it is not possible to
specify dependent or polymorphic types there. This is still sufficient for many
practical use cases, for example when the object theory is simply-typed, in
which case staging also performs \emph{monomorphization}. In this setup, it makes
sense to only have weakening in the equational theory of the object theory, but
no $\beta\eta$-rules and no substitution. The reason is that we do not want
to equate programs with different performance characteristics, when we do
staging in order to improve runtime code performance.
\\\\
\textbf{Future work.} A major line of future work is to connect 2LTT to existing
literature on staged compilation. This would involve formalizing existing tricks
and techniques, such as let-insertion techniques
\cite{DBLP:journals/corr/abs-2201-00495}, fusion, various binding-time
improvements and CPS conversions \cite{partial-evaluation}. Another line is
fleshing out practical details for staging and intensional analysis. In staging,
a production-strength solution should include some form of caching, to reduce
code duplication. In intensional analysis, some form of induction or pattern
matching would be more ergonomic than plain decidability of conversion.

Also, 2LTT could be extended to more general \emph{multimodal}
\cite{gratzer20multimodal} type theories, where modalities represent morphisms
between different object-theoretic syntactic categories, in possibly different
object theories. A simple example is the closed or ``crisp'' modality
\cite{licata2018internal} which can be used to represent closed object terms.


\bibliographystyle{plain}
\bibliography{references}

\end{document}
