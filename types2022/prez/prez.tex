
\documentclass[dvipsnames]{beamer}
\usetheme{Madrid}

%% kill footline
\setbeamertemplate{footline}[frame number]{}
\setbeamertemplate{navigation symbols}{}
\setbeamertemplate{footline}{}

%% bibliography
\bibliographystyle{alpha}
\setbeamerfont{bibliography item}{size=\footnotesize}
\setbeamerfont{bibliography entry author}{size=\footnotesize}
\setbeamerfont{bibliography entry title}{size=\footnotesize}
\setbeamerfont{bibliography entry location}{size=\footnotesize}
\setbeamerfont{bibliography entry note}{size=\footnotesize}
\setbeamertemplate{bibliography item}{}

%% kill ball enumeration
\setbeamertemplate{enumerate items}[circle]
\setbeamertemplate{section in toc}[circle]

%% kill block shadows
\setbeamertemplate{blocks}[rounded][shadow=false]
\setbeamertemplate{title page}[default][colsep=-4bp,rounded=true]

%% kill ball itemize
\setbeamertemplate{itemize items}[circle]

%% --------------------------------------------------------------------------------

\usepackage[utf8]{inputenc}
\usepackage{hyperref}
\usepackage{amsmath}
\usepackage{cite}
\usepackage{amsthm}
\usepackage{amssymb}
\usepackage{amsfonts}
\usepackage{mathpartir}
\usepackage{scalerel}
\usepackage{stmaryrd}
\usepackage{bm}
\usepackage{graphicx}

\hypersetup{
  colorlinks=true,
  linkcolor=blue,
  urlcolor=blue}

%% --------------------------------------------------------------------------------

\renewcommand{\mit}[1]{\mathit{#1}}
\newcommand{\msf}[1]{\mathsf{#1}}
\newcommand{\mbb}[1]{\mathbb{#1}}
\newcommand{\mbf}[1]{\mathbf{#1}}
\newcommand{\bs}[1]{\boldsymbol{#1}}
\newcommand{\wh}[1]{\widehat{#1}}
\newcommand{\ext}{\triangleright}
\newcommand{\Code}{\msf{Code}}
\newcommand{\El}{\msf{El}}
\newcommand{\lam}{\msf{lam}}
\newcommand{\app}{\msf{app}}
\newcommand{\NatElim}{\msf{NatElim}}
\newcommand{\y}{\msf{y}}

\newcommand{\Lift}{{\Uparrow}}
\newcommand{\spl}{{\sim}}
\newcommand{\qut}[1]{{<}#1{>}}

\newcommand{\mbbc}{\mbb{C}}
\newcommand{\mbbo}{\mbb{O}}
\newcommand{\ob}{_\mbbo}

\newcommand{\U}{\msf{U}}
\newcommand{\Con}{\msf{Con}}
\newcommand{\Sub}{\msf{Sub}}
\newcommand{\Ty}{\msf{Ty}}
\newcommand{\Tm}{\msf{Tm}}
\newcommand{\Cono}{\msf{Con}_{\mbbo}}
\newcommand{\Subo}{\msf{Sub}_{\mbbo}}
\newcommand{\Tyo}{\msf{Ty}_{\mbbo}}
\newcommand{\Tmo}{\msf{Tm}_{\mbbo}}
\newcommand{\hCon}{\wh{\msf{Con}}}
\newcommand{\hSub}{\wh{\msf{Sub}}}
\newcommand{\hTy}{\wh{\msf{Ty}}}
\newcommand{\hTm}{\wh{\msf{Tm}}}

\newcommand{\p}{\mathsf{p}}
\newcommand{\q}{\mathsf{q}}

\newcommand{\refl}{\msf{refl}}
\newcommand{\Bool}{\msf{Bool}}
\newcommand{\true}{\msf{true}}
\newcommand{\false}{\msf{false}}
\newcommand{\True}{\msf{True}}
\newcommand{\False}{\msf{False}}
\newcommand{\List}{\msf{List}}
\newcommand{\nil}{\msf{nil}}
\newcommand{\cons}{\msf{cons}}
\newcommand{\Nat}{\msf{Nat}}
\newcommand{\zero}{\msf{zero}}
\newcommand{\suc}{\msf{suc}}
\renewcommand{\tt}{\msf{tt}}
\newcommand{\fst}{\msf{fst}}
\newcommand{\snd}{\msf{snd}}
\newcommand{\mylet}{\msf{let}}
\newcommand{\emptycon}{\scaleobj{.75}\bullet}
\newcommand{\id}{\msf{id}}

\newcommand{\Set}{\mathsf{Set}}
\newcommand{\Prop}{\mathsf{Prop}}
\newcommand{\Rep}{\msf{Rep}}
\newcommand{\blank}{{\mathord{\hspace{1pt}\text{--}\hspace{1pt}}}}
\newcommand{\emb}[1]{\ulcorner#1\urcorner}

\newcommand{\Stage}{\msf{Stage}}
\newcommand{\hato}{\bm\hat{\mbbo}}
\newcommand{\ev}{\mbb{E}}
\newcommand{\re}{\mbb{R}}

\theoremstyle{remark}
\newtheorem{notation}{Notation}

\newcommand{\whset}{\wh{\Set}}
\newcommand{\rexti}{\re_{\ext_1}^{-1}}
\newcommand{\rextizero}{\re_{\ext_0}^{-1}}

\newcommand{\rel}{^{\approx}}
\newcommand{\yon}{\msf{y}}

\newcommand{\Vect}{\msf{Vec}}

%% --------------------------------------------------------------------------------


\title{Conservativity of Two-Level Type Theory Corresponds to Staged Compilation}
\author{András Kovács}
\institute{
  {Eötvös Loránd University}
}
\date{21 June 2022, TYPES, Nantes}
\begin{document}

\frame{\titlepage}

\begin{frame}{Overview}

Two-level TT:
\begin{itemize}
  \item \emph{Voevodsky: A simple type system with two identity types}
  \item \emph{Annekov, Capriotti, Kraus, Sattler: Two-Level Type Theory and Applications}
  \item Goal: synthetic homotopy theory
\end{itemize}
\vspace{1em}
\pause

Staged compilation:
\begin{itemize}
  \item Template Haskell, MetaOCaml
  \item Goal: code generation (for performance, code reuse)
  \pause
  \item \emph{Remark: staged compilation $\neq$ staged computation}
\end{itemize}
\vspace{1em}

\end{frame}

\begin{frame}{Rules}

  \begin{enumerate}
    \item Two universes $\U_0$, $\U_1$, closed under arbitrary type formers.
    \pause
    \item No elimination allowed from one universe to the other.
    \pause
    \item \emph{Lifting:} for $A : \U_0$, we have $\Lift A : \U_1$.
    \pause
    \item \emph{Quoting:} for $A : \U_0$ and $t : A$, we have $\qut{t} : \Lift A$.
    \pause
    \item \emph{Splicing:} for $t : \Lift A$, we have $\spl{t} : A$.
    \pause
    \item Quoting and splicing are definitional inverses.
  \end{enumerate}

\end{frame}

\begin{frame}{Inlined definitions}

Staging input:
\begin{alignat*}{4}
  &\msf{two} : \Lift \Nat_0 \\
  &\msf{two} = \qut{\suc_0\,(\suc_0\,\zero_0)} \\
  & \\
  & \msf{f} : \Nat_0 \to \Nat_0 \\
  & \msf{f} = \lambda\,x.\, x + \spl{\msf{two}}
\end{alignat*}
\pause
Output:
\begin{alignat*}{4}
  & \msf{f} : \Nat_0 \to \Nat_0 \\
  & \msf{f} = \lambda\,x.\, x + \suc_0\,(\suc_0\,\zero_0)
\end{alignat*}


\end{frame}

\begin{frame}{Compile-time functions}

Input:
\begin{alignat*}{4}
  & \id : (A : \U_1) \to A \to A\\
  & \id = \lambda\,A\,x.\,x     \\
  & \\
  & \msf{idBool_0} : \Bool_0 \to \Bool_0\\
  & \msf{idBool_0} = \lambda\,x.\,\spl(\id\,(\Lift \Bool_0)\,\qut{x})
\end{alignat*}
\pause
Output:
\begin{alignat*}{4}
  & \msf{idBool_0} : \Bool_0 \to \Bool_0\\
  & \msf{idBool_0} = \lambda\,x.\,x
\end{alignat*}

\end{frame}


\begin{frame}{Inlined $\msf{map}$ arguments}

Input:
\begin{alignat*}{4}
  & \msf{inlMap} : \{A\,B : \Lift \U_0\} \to (\Lift\spl A \to \Lift\spl B) \to \Lift(\msf{List_0}\,\spl A) \to \Lift(\msf{List_0}\,\spl B) \\
  & \msf{inlMap} = \lambda\,f\,\mit{as}.\,\qut{\msf{foldr_0}\,
    (\lambda\,a\,\mit{bs}.\,\cons_0\,\spl(f\,\qut{a})\,\mit{bs})\,
    \nil_0\,
    \spl{\mit{as}}
    }\\
  & \\
  & \msf{f} : \List_0\,\Nat_0 \to \List_0\,\Nat_0\\
  & \msf{f} = \lambda\,\mit{xs}.\,\,\spl(\msf{inlMap}\,(\lambda\,n.\,\qut{\spl n + 2})
     \,\qut{\mit{xs}})
\end{alignat*}
\pause
Output:
\begin{alignat*}{4}
  & \msf{f} : \List_0\,\Nat_0 \to \List_0\,\Nat_0\\
  & \msf{f} = \lambda\,\mit{xs}.\, \msf{foldr}_0\,(\lambda\,a\,\mit{bs}.\,\cons_0\,(a + 2)\,\mit{bs})\,\nil_0\,\mit{xs}
\end{alignat*}

\end{frame}

\begin{frame}{Staging Types}

Input:
\begin{alignat*}{4}
  & \rlap{$\Vect : \Nat_1 \to \Lift \U_0 \to \Lift \U_0$}\\
  & \Vect\,&&\zero_1    \,&&A = \qut{\top_0}\\
  & \Vect\,&&(\suc_1\,n)\,&&A = \qut{\spl A \times_0 \spl(\Vect\,n\,A)}\\
  & \\
  & \rlap{$\msf{Tuple3} : \U_0 \to \U_0$} \\
  & \rlap{$\msf{Tuple3}\,A = \spl(\Vect\,3\,\qut{A})$}
\end{alignat*}
\pause
Output:
\begin{alignat*}{4}
  &\msf{Tuple3} : \U_0 \to \U_0 \\
  &\msf{Tuple3}\,A = A \times_0 (A \times_0 (A \times_0 \top_0))
\end{alignat*}

\end{frame}

\begin{frame}{$\msf{map}$ for $\Vect$}

Input:
\begin{alignat*}{3}
  &\msf{map} : \{A\,B : \Lift \U_0\} \to
  (n : \Nat_1) \to (\Lift\spl A \to \Lift\spl B) \\
  & \hspace{7.8em} \to \Lift(\Vect\,n\,A) \to \Lift(\Vect\,n\,B)\\
  &\msf{map}\,\zero_1\hspace{1.3em}f\,\mit{as} = \qut{\tt_0} \\
  &\msf{map}\,(\suc_1\,n)\,f\,\mit{as} =
    \qut{(\spl(f\,\qut{\fst_0\,\spl as}),\,\spl(\msf{map}\,n\,f\,\qut{\snd_0\,\spl as}))} \\
  & \\
  & \msf{f} : \spl(\Vect\,2\,\qut{\Nat_0}) \to \spl(\Vect\,2\,\qut{\Nat_0}) \\
  & \msf{f}\,\mit{xs} = \spl(\msf{map}\,2\,(\lambda\,x.\,\qut{\spl x + 2})\,\qut{\mit{xs}}
\end{alignat*}
\pause
Output:
\begin{alignat*}{3}
  & \msf{f} : \Nat_0 \times_0 (\Nat_0 \times_0 \top_0) \to \Nat_0 \times_0 (\Nat_0 \times_0 \top_0) \\
  & \msf{f}\,\mit{xs} = (\fst_0\,\mit{xs} + 2,\,(\fst_0\,(\snd_0\,\mit{xs}) + 2,\,\tt_0))
\end{alignat*}

\end{frame}

\begin{frame}{Ergonomics}

In the demo implementation:
\vspace{1em}

\begin{itemize}
  \item Bidirectional elaboration
  \item Coercive subtyping for $\Lift$ and type formers
  \item Standard unification techniques
\end{itemize}
\vspace{1em}

Almost all quotes and splices are inferable in practice.

\end{frame}

\begin{frame}{Staging as Conservativity}

The \textbf{object theory} is the TT supporting only $\U_0$ and its type formers.
\vspace{1em}
\pause

The \textbf{object-level fragment} of 2LTT contains types in $\U_0$, their terms,
and only allows contexts with entries in $\U_0$.
\vspace{1em}
\pause

\textbf{Conservativity of 2LTT means}
\begin{itemize}
  \item There's a bijection between object-theoretic types and object-fragment 2LTT types.
  \item There's also a bijection between object-theoretic terms and object-fragment 2LTT terms.
  \item (Both up to $\beta\eta$-conversion).
\end{itemize}
\vspace{1em}

(See proof in the preprint)


\end{frame}




\begin{frame}{}

ICFP preprint, implementation, tutorial: \url{github.com/AndrasKovacs/staged}
\vspace{2em}

\begin{center}
  \Large {Thanks for your attention!}
\end{center}


\end{frame}






\end{document}
