
\documentclass{easychair}
\raggedbottom

\usepackage{doc}
\usepackage{xcolor}
\usepackage{mathpartir}
\usepackage{todonotes}
\presetkeys{todonotes}{inline}{}
\usepackage{scalerel}
\usepackage{bm}
\usepackage{amssymb}

\renewcommand{\mit}[1]{{\mathit{#1}}}
\newcommand{\msf}[1]{{\mathsf{#1}}}
\newcommand{\mbf}[1]{{\mathbf{#1}}}
\newcommand{\mbb}[1]{\mathbb{#1}}
\newcommand{\bs}[1]{\boldsymbol{#1}}
\newcommand{\ext}{\triangleright}
\newcommand{\blank}{{\mathord{\hspace{1pt}\text{--}\hspace{1pt}}}}

\newcommand{\MetaTy}{\msf{MetaTy}}
\newcommand{\Base}{\msf{Base}}
\newcommand{\PSh}{\msf{PSh}}
\newcommand{\El}{\msf{El}}
\newcommand{\Cat}{\msf{Cat}}
\newcommand{\In}{\msf{In}}
\newcommand{\base}{\msf{base}}
\newcommand{\Tm}{\msf{Tm}}
\newcommand{\U}{\msf{U}}
\newcommand{\Con}{\msf{Con}}
\newcommand{\Sub}{\msf{Sub}}
\newcommand{\Y}{\msf{Y}}
\renewcommand{\S}{\msf{S}}
\newcommand{\Fib}{\msf{Fib}}
\newcommand{\Fin}{\msf{Fin}}
\newcommand{\Tree}{\msf{Tree}}
\newcommand{\Disc}{\msf{Disc}}

\title{A Generalized Logical Framework}
\author{Andr\'as Kov\'acs$^{1}$ \and Christian Sattler$^{2}$}
\titlerunning{A Generalized Logical Framework}

\institute{
  Chalmers University of Technology \& University of Gothenburg, Sweden \\
  $^{1}$\,\email{andrask@chalmers.se}\,\,\,$^{2}$\,\email{sattler@chalmers.se}
}

\pagenumbering{gobble}
\begin{document}
\maketitle

Logical frameworks (LFs \cite{Harper93lf}) and the closely related two-level type
theories (2LTTs \cite{twolevel}) let us work in a mixed syntax of a metatheory and a
chosen object theory. Here, we have a second-order view on the object theory,
where contexts, variables and substitutions are implicit, and binders are
represented as meta-level functions. There are some well-known limitations to
LFs. First, we have to pick a model of the object theory externally. Second,
since we only have a second-order view on that model, many constructions cannot
be expressed; for example, the induction principle for the syntax of an object
theory requires a notion of first-order model, where contexts and substitutions
are explicit. Various ways have been described to make logical frameworks more
expressive by extending them with modalities (e.g.\ \cite{sterlingthesis,DBLP:conf/lics/Hofmann99,DBLP:journals/corr/abs-1901-03378,orton_et_al:LIPIcs:2016:6564}). In the current work we
describe an LF with the following features:
\begin{itemize}
\item We can work with multiple models of multiple object theories at the same
  time. By ``theory'' we mean a second-order generalized algebraic theory
  (SOGAT \cite{uemura,DBLP:conf/fscd/KaposiX24}); this includes all type theories and programming languages that
  only use structural binders.
\item We have both an ``external'' first-order view and an ``internal''
  second-order view on each model, and we can freely switch between
  perspectives. All models of object theories are defined internally in the LF.
\item The LF is fully structural as a type theory; no substructural modalities
  are used.
\end{itemize}

\noindent\textbf{The Logical Framework.} The basic structure is as follows.
\begin{itemize}
\item We have a universe $\MetaTy$\footnote{More precisely, a $\mbb{N}$-indexed universe
hierarchy, but we shall omit ``sizing'' levels in this abstract.} closed under the type formers
  of extensional type theory.
\item We have $\Base : \MetaTy$, $1 : \Base$, $\PSh : \Base \to \MetaTy$ and $\El : \{i :
  \Base\} \to \PSh\,i \to \MetaTy$ such that each $\PSh\,i$ and $\El$ constitutes
  a Tarski-style universe closed under ETT type formers.
\item Let us define $\Cat\,i : \PSh\,i$ as the type of categories internally to
  $\PSh\,i$. Then, we have $\In : \{i : \Base\} \to \El\,(\Cat\,i) \to \MetaTy$ and $\msf{base} : \In\,C \to \Base$.
\end{itemize}
We give some semantic intuition in the following. Each $\PSh\,i$ is a universe
of presheaves over some base category. In the empty context, only $\PSh\,1$ is
available, which is the universe of sets. Internally to $\PSh\,1$, we can define
some $C : \El\,(\Cat\,1)$. Now, if we have $i : \In\,C$, we can form $\PSh\,(\base\,i)$
as the universe of presheaves over $C$.
\begin{enumerate}
\item We can define $\msf{PShExt\,C} : \PSh\,1$ as the \emph{external} type of presheaves over $C$.
\item Our semantics supports the isomorphism $\El\,(\msf{PShExt}\,C) \simeq ((i
  : \In\,C) \to \PSh\,(\base\,i))$.  In other words, external and internal
  notions of presheaves coincide. More generally, we have this isomorphism for
  any $C : \El\,(\Cat\,j)$, i.e.\ starting from a category that's internal to any
  previously defined presheaf universe.
\end{enumerate}
\textbf{Yoneda embeddings}. Our semantics actually supports a more general
notion of internalization than the above one, which we don't describe here. We
have not yet finalized which operations to enshrine in the LF's syntax, but the
special case of \emph{Yoneda embeddings} seems to be especially useful. This
works in the generality of SOGATs but we shall focus on the example of pure
lambda calculus. A second-order model of pure LC in some universe $\U$ is simply
$\Tm : \U$ together with an isomorphism $\Tm \simeq (\Tm \to \Tm)$. A
first-order model is a unityped category with families \cite{cwfs}, where we
write $\msf{Con} : \U$ for the type of contexts, $\Tm : \Con \to \U$ for the
type of terms, $\Gamma+ : \Con$ for the extension of $\Gamma : \Con$ with a
binding, and we have a natural isomorphism $\Tm\,\Gamma \simeq \Tm\,(\Gamma+)$.
\begin{itemize}
\item For each $M$ a first-order model in $\PSh\,i$ and $j : \In\,M$\footnote{We
implicitly take the underlying category of $M$ here.}, we have $\msf{S}_j$ as a
  second-order model in $\PSh\,j$. In other words, internally to presheaves over
  a model of lambda calculus, we have a second-order model of lambda
  calculus. In fact, this is the standard semantics of traditional LFs/2LTTs,
  and we get all such LFs/2LTTs as syntactic fragments of our generalized LF, by
  working under an assumption of $j : \In\,M$.
\item Yoneda embedding has action on contexts, substitutions and terms:
  \begin{alignat*}{3}
    & \Y : \El\,\Con_M \to (\{j : \In\,M\} \to \PSh\,j)\\
    & \Y : \El\,(\Sub_M\,\Gamma\,\Delta) \simeq (\{j : \In\,M\} \to \El\,(\Y\,\Gamma\,\{j\}) \to \El\,(\Y\,\Delta\,\{j\}))\\
    & \Y : \El\,(\Tm_M\,\Gamma) \simeq (\{j : \In\,M\} \to \El\,(\Y\,\Gamma) \to \El\,\Tm_{\S_j})
  \end{alignat*}
\end{itemize}
Additionally, $\Y$ preserves empty contexts and extended contexts up to
isomorphism and preserves all other structure strictly. $\Y$ allows ad-hoc
switching between first-order and second-order syntax. For example, the identity
substitution $\msf{id} : \El\,(\Sub_M\,\Gamma\,\Gamma)$ can be alternatively defined as
$\Y^{-1}(\lambda\,\gamma.\,\gamma)$, where we use $\Y^{-1}$ to externalize
$(\lambda\,\gamma.\,\gamma) : (\{j : \In\,M\} \to \El\,(\Y\,\Gamma) \to
\El\,(\Y\,\Gamma))$. More generally, by using a modest amount of syntactic sugar
and elaboration, we can develop $\Y$ and $\Y^{-1}$ into a ``second-order
notation'' for any SOGAT, which constitutes a rigorous and nicely readable
alternative to De Bruijn indices and explicit substitution operations.
\\

\noindent \textbf{Sketch of the semantics}. The model of LF is constructed in
two steps. First, we give a model for the theory that has $\PSh$, $\Base$ and
$\In$ as sorts but does not support $\MetaTy$, and then take presheaves over
that model to obtain a model of a 2LTT where $\MetaTy$ represents the outer
layer. In the inner model, we start with an inductive definition of certain
``trees in categories'':
\begin{alignat*}{3}
  & \mbf{data}\,\Tree\,(B : \Cat) : \msf{Set}\,\mbf{where}\\
  & \hspace{1em} \msf{node} : (\Gamma : \PSh\,B)(n : \mbb{N})(C : \Fin\,n \to \Fib\,(B \ext \Disc\,\Gamma))\\
  & \hspace{3.8em} ((i : \Fin\,n) \to \Tree\,(B \ext \Disc\,\Gamma \ext C\,i)) \to \Tree\,B
\end{alignat*}
Here, $\PSh$ means presheaves in sets, $\Fib$ is cartesian fibrations, $\Disc$
creates a discrete fibration from a presheaf and $\blank\!\ext\!\blank$ takes
the total category of a fibration. Now, the objects of
the semantic base category are elements of $\Tree\,1$, and morphisms between
trees are level-wise natural transformations between the $\Gamma$ components
together with $\Fin\,n \to \Fin\,m$ renamings of subtree indices. The
non-discrete $\Fib$ components are preserved by morphisms. A semantic $\Base$
points to a subtree of a context, an $\In$ is a $\Fin\,n$ index pointing to a
child of a given node, and a $\PSh$ is a dependent presheaf over a $\Gamma$ inside a
given node. Extending a context with an $\In$ binding adds a new empty subtree
to a given node. Extending with an $\El$ binding extends the $\Gamma$ presheaf
in a node with a dependent presheaf.

\bibliographystyle{plain}
\bibliography{references}

\end{document}
