%% build: latexmk -pdf -pvc prez.tex

\documentclass[dvipsnames,aspectratio=169]{beamer}
\usetheme{Madrid}

%% kill footline
\setbeamertemplate{footline}[frame number]{}
\setbeamertemplate{navigation symbols}{}
\setbeamertemplate{footline}{}

%% bibliography
\bibliographystyle{alpha}
\setbeamerfont{bibliography item}{size=\footnotesize}
\setbeamerfont{bibliography entry author}{size=\footnotesize}
\setbeamerfont{bibliography entry title}{size=\footnotesize}
\setbeamerfont{bibliography entry location}{size=\footnotesize}
\setbeamerfont{bibliography entry note}{size=\footnotesize}
\setbeamertemplate{bibliography item}{}

%% kill ball enumeration
\setbeamertemplate{enumerate items}[circle]
\setbeamertemplate{section in toc}[circle]

%% kill block shadows
\setbeamertemplate{blocks}[rounded][shadow=false]
\setbeamertemplate{title page}[default][colsep=-4bp,rounded=true]

%% kill ball itemize
\setbeamertemplate{itemize items}[circle]

%% --------------------------------------------------------------------------------

\usepackage[utf8]{inputenc}
%% \usepackage[hidelinks]{hyperref}
\usepackage{amsmath}
\usepackage{cite}
\usepackage{amsthm}
\usepackage{amssymb}
\usepackage{amsfonts}
\usepackage{mathpartir}
\usepackage{scalerel}
\usepackage{stmaryrd}
\usepackage{bm}
\usepackage{graphicx}

%% --------------------------------------------------------------------------------

%% HoTT style composition
\makeatletter
\DeclareRobustCommand{\sqcdot}{\mathbin{\mathpalette\morphic@sqcdot\relax}}
\newcommand{\morphic@sqcdot}[2]{%
  \sbox\z@{$\m@th#1\centerdot$}%
  \ht\z@=.33333\ht\z@
  \vcenter{\box\z@}%
}
\makeatother

\newcommand{\mi}[1]{\mathit{#1}}
\newcommand{\ms}[1]{\mathsf{#1}}
\newcommand{\mbb}[1]{\mathbb{#1}}
\newcommand{\mbf}[1]{\mathbf{#1}}
\newcommand{\bs}[1]{\boldsymbol{#1}}
\newcommand{\ap}{\ms{ap}}
\newcommand{\apd}{\ms{apd}}
\newcommand{\tr}{\ms{tr}}
\newcommand{\happly}{\ms{happly}}
\newcommand{\funext}{\ms{funext}}
\newcommand{\toind}{\to^{\ms{int}}}

\newcommand{\Tys}{\ms{Ty_{sig}}}
\newcommand{\Tms}{\ms{Tm_{sig}}}
\newcommand{\Us}{\ms{U_{sig}}}
\newcommand{\Els}{\ms{El_{sig}}}

\newcommand{\Ix}{\mi{Ix}}

\newcommand{\zero}{\ms{zero}}
\newcommand{\suc}{\ms{suc}}
\newcommand{\J}{\ms{J}}
\newcommand{\UIP}{\ms{UIP}}

\newcommand{\refl}{\mathsf{refl}}
\newcommand{\reflect}{\mathsf{reflect}}
\newcommand{\Reflect}{\mathsf{Reflect}}
\newcommand{\id}{\mathsf{id}}
\newcommand{\Con}{\mathsf{Con}}
\newcommand{\Sub}{\mathsf{Sub}}
\newcommand{\Tm}{\mathsf{Tm}}
\newcommand{\Ty}{\mathsf{Ty}}
\newcommand{\U}{\mathsf{U}}
\newcommand{\El}{\mathsf{El}}
\newcommand{\Id}{\mathsf{Id}}
\newcommand{\ID}{\mathsf{ID}}
\newcommand{\proj}{\mathsf{proj}}
\renewcommand{\tt}{\mathsf{tt}}
\newcommand{\blank}{\mathord{\hspace{1pt}\text{--}\hspace{1pt}}}
\newcommand{\ra}{\rightarrow}

\newcommand{\Y}{\mathsf{Y}}
\newcommand{\Set}{\mathsf{Set}}
\newcommand{\CatTel}{\mathsf{CatTel}}
\newcommand{\Fib}{\mathsf{Fib}}
\newcommand{\Fin}{\mathsf{Fin}}
\newcommand{\Tree}{\mathsf{Tree}}
\newcommand{\Disc}{\mathsf{Disc}}
\newcommand{\Base}{\mathsf{Base}}
\newcommand{\In}{\mathsf{In}}
\newcommand{\PSh}{\mathsf{PSh}}
\newcommand{\Cat}{\mathsf{Cat}}
\newcommand{\base}{\mathsf{base}}
\newcommand{\SMod}{\mathsf{SMod}}
\newcommand{\FMod}{\mathsf{FMod}}
\newcommand{\bBase}{\mathbf{Base}}
\newcommand{\bIn}{\mathbf{In}}
\newcommand{\bPSh}{\mathbf{PSh}}
\newcommand{\bCat}{\mathbf{Cat}}
\newcommand{\bbase}{\mathbf{base}}

\newcommand{\Lift}{\Uparrow}
\newcommand{\ToS}{\mathsf{ToS}}
\newcommand{\ext}{\triangleright}
\newcommand{\emptycon}{\scaleobj{.75}\bullet}

\newcommand{\Pii}{\Pi}
\newcommand{\funi}{\Rightarrow}
\newcommand{\appi}{\mathsf{app}}
\newcommand{\lami}{\mathsf{lam}}

\newcommand{\fune}{\Rightarrow^{\ms{Ext}}}
\newcommand{\Pie}{\Pi^{\mathsf{Ext}}}
\newcommand{\appe}{\mathsf{app^{Ext}}}
\newcommand{\lame}{\mathsf{lam^{Ext}}}
\newcommand{\toe}{\to^{\ms{Ext}}}
\newcommand{\arre}{\Rightarrow^{\mathsf{Ext}}}
\newcommand{\lambdae}{\lambda^{\ms{Ext}}}

\newcommand{\Piinf}{\Pi^{\mathsf{ext}}}
\newcommand{\appinf}{\mathsf{app^{ext}}}
\newcommand{\laminf}{\mathsf{lam^{ext}}}
\newcommand{\laminfprime}{\mathsf{lam^{ext'}}}
\newcommand{\toinf}{\to^{\ms{ext}}}
\newcommand{\lambdainf}{\lambda^{\ms{ext}}}
\newcommand{\arrinf}{\Rightarrow^{\mathsf{ext}}}
\newcommand{\bPiinf}{\bs{\Piinf}}

\newcommand{\appitt}{\mathop{{\scriptstyle @}}}
\newcommand{\Refl}{\mathsf{Refl}}
\newcommand{\IdU}{\mathsf{IdU}}
\newcommand{\ReflU}{\mathsf{ReflU}}
\newcommand{\Sig}{\mathsf{Sig}}
\newcommand{\ToSSig}{\mathsf{ToSSig}}
\newcommand{\Subtype}{\mathsf{Subtype}}
\newcommand{\subtype}{\mathsf{subtype}}
\newcommand{\NatSig}{\mathsf{NatSig}}
\newcommand{\Sg}{\Sigma}
\newcommand{\flcwf}{\mathsf{flcwf}}
\newcommand{\SigTy}{\mathsf{SigTy}}
\newcommand{\SigTm}{\mathsf{SigTm}}
\newcommand{\SigU}{\mathsf{SigU}}
\newcommand{\tm}{\ms{tm}}
\newcommand{\ty}{\ms{ty}}

\newcommand{\Kfam}{\mathsf{K}}
\newcommand{\lamK}{\mathsf{lam}_{\K}}
\newcommand{\appK}{\mathsf{app}_{\K}}

\newcommand{\p}{\mathsf{p}}
\newcommand{\q}{\mathsf{q}}
\newcommand{\K}{\mathsf{K}}
\newcommand{\A}{\mathsf{A}}
\newcommand{\D}{\mathsf{D}}
\renewcommand{\S}{\mathsf{S}}
\newcommand{\arri}{\Rightarrow}
\newcommand{\syn}{\mathsf{syn}}
\newcommand{\SynSig}{\mathsf{SynSig}}
\newcommand{\bCon}{\bs{\Con}}
\newcommand{\bTy}{\bs{\Ty}}
\newcommand{\bSub}{\bs{\Sub}}
\newcommand{\bTm}{\bs{\Tm}}
\newcommand{\bGamma}{\bs{\Gamma}}
\newcommand{\bDelta}{\bs{\Delta}}
\newcommand{\bsigma}{\bs{\sigma}}
\newcommand{\bdelta}{\bs{\delta}}
\newcommand{\bepsilon}{\bs{\epsilon}}
\newcommand{\bt}{\bs{t}}
\newcommand{\bu}{\bs{u}}
\newcommand{\bA}{\bs{A}}
\newcommand{\ba}{\bs{a}}
\newcommand{\bb}{\bs{b}}
\newcommand{\bB}{\bs{B}}
\newcommand{\bid}{\bs{\id}}
\newcommand{\bemptycon}{\scaleobj{.75}{\bs{\bullet}}}
\newcommand{\bSet}{\bs{\Set}}
\newcommand{\bU}{\bs{\U}}
\newcommand{\bEl}{\bs{\El}}
\newcommand{\bPii}{\bs{\Pi}}
\newcommand{\bPie}{\bs{\Pie}}
\newcommand{\bappi}{\bs{\mathsf{app}}}
\newcommand{\blami}{\bs{\mathsf{lam}}}
\newcommand{\bId}{\bs{\Id}}
\newcommand{\bM}{\bs{\mathsf{M}}}
\newcommand{\bT}{\bs{\mathsf{T}}}
\newcommand{\bS}{\bs{\mathsf{S}}}
\newcommand{\bP}{\bs{\mathsf{P}}}
\newcommand{\bD}{\bs{\mathsf{D}}}
\newcommand{\bI}{\bs{\mathsf{I}}}
\newcommand{\bK}{\bs{\mathsf{K}}}

\newcommand{\ul}[1]{\underline{#1}}
\newcommand{\ulGamma}{\ul{\Gamma}}
\newcommand{\ulDelta}{\ul{\Delta}}
\newcommand{\ulgamma}{\ul{\gamma}}
\newcommand{\ulOmega}{\ul{\Omega}}
\newcommand{\uldelta}{\ul{\delta}}
\newcommand{\ulsigma}{\ul{\sigma}}
\newcommand{\ulnu}{\ul{\nu}}
\newcommand{\ulepsilon}{\ul{\epsilon}}
\newcommand{\ulemptycon}{\ul{\emptycon}}
\newcommand{\ult}{\ul{t}}
\newcommand{\ulu}{\ul{u}}
\newcommand{\ulA}{\ul{A}}
\newcommand{\ula}{\ul{a}}
\newcommand{\ulB}{\ul{B}}
\newcommand{\tos}{\mathsf{tos}}
\newcommand{\coe}{\mathsf{coe}}
\newcommand{\coh}{\mathsf{coh}}
\newcommand{\llb}{\llbracket}
\newcommand{\rrb}{\rrbracket}
\newcommand{\sem}[1]{\llb#1\rrb}

\newcommand{\Var}{\ms{Var}}
\newcommand{\var}{\ms{var}}
\newcommand{\app}{\ms{app}}
\newcommand{\vz}{\ms{vz}}
\newcommand{\vs}{\ms{vs}}
\newcommand{\Alg}{\ms{Alg}}
\newcommand{\Mor}{\ms{Mor}}
\newcommand{\DispAlg}{\ms{DispAlg}}
\newcommand{\Section}{\ms{Section}}
\newcommand{\Initial}{\ms{Initial}}
\newcommand{\Inductive}{\ms{Inductive}}
\newcommand{\TmAlg}{\ms{TmAlg}}
\newcommand{\Rec}{\ms{Rec}}
\newcommand{\Ind}{\ms{Ind}}
\newcommand{\Obj}{\ms{Obj}}
\newcommand{\Nat}{\ms{Nat}}
\newcommand{\Bool}{\ms{Bool}}
\newcommand{\mbbC}{\mbb{C}}
\newcommand{\hmbbC}{\hat{\mbb{C}}}
\newcommand{\mbbD}{\mbb{D}}
\newcommand{\lam}{\ms{lam}}

\newcommand{\true}{\ms{true}}
\newcommand{\false}{\ms{false}}
\newcommand{\up}{\uparrow}
\newcommand{\down}{\downarrow}

\newcommand{\lab}{\langle}
\newcommand{\rab}{\rangle}
\newcommand{\defn}{:\equiv}
\newcommand{\yon}{\ms{y}}
\newcommand{\lub}{\,\sqcup\,}
\newcommand{\bmsA}{\bs{\ms{A}}}


%% --------------------------------------------------------------------------------

\title{A Generalized Logical Framework}
\author{\textbf{András Kovács\inst{1}},\,\,Christian Sattler\inst{1}}
\institute{
  \inst{1}%
       {University of Gothenburg \& Chalmers University of Technology}
}
\date{18 Apr 2025, EuroProofNet WG6 meeting, Genoa}
\begin{document}


\frame{\titlepage}

\begin{frame}{Overview}

\begin{enumerate}
\item Two-level type theories (2LTT):
  \begin{itemize}
    \item metaprogramming over a \textbf{single model} of a \textbf{single type theory}.
    \item the chosen model is defined \textbf{outside the system}.
    \item \textbf{only a second-order (``internal'')} view on the model.
  \end{itemize}
\pause
\item Generalized logical framework (GLF):
  \begin{itemize}
  \item metaprogramming over \textbf{any number of models} of \textbf{any number of type theories}.
  \item models are defined \textbf{inside the system}.
  \item both a \textbf{first-order/external} and a \textbf{second-order/internal} view on each model.
  \pause
  \item \emph{No substructural modalities}.
  \end{itemize}
\end{enumerate}
\pause
\vspace{1em}

\emph{In this talk}:
\begin{enumerate}
\item A syntax of GLF + examples + increasing amount of syntactic sugar.
\item A short overview of semantics.
\end{enumerate}

\end{frame}

\begin{frame}{GLF basic universes \& type formers}

\begin{block}{}
\vspace{-1em}
\begin{alignat*}{3}
  & \bU     && : \bU           \hspace{6em}&& \text{A universe of that supports ETT.}\\
  & \bBase  && : \bU                 && \text{Type of ``base categories''.} \\
  & \mbf{1} && : \bBase                && \text{The terminal category as a base category.} \\
  & \bPSh   && : \bBase \to \bU      && \text{Universes of presheaves. Cumulativity: $\PSh_i \subseteq \U$. Supports ETT.}\\
  &        &&                          && \text{We can only eliminate from $\PSh_i$ to $\PSh_i$.} \\
  & \bCat_i && : \bPSh_i               && := \text{\emph{type of categories in $\PSh_i$}} \\
  & \bIn    &&: \bCat_i \to \bU      && \text{``Permission token'' for working in presheaves over some $\mbbC : \Cat_i$.} \\
  & \bbase  &&: \bIn\,\mbbC \to \bBase     && \text{``Using the permission''}.
\end{alignat*}
\end{block}
\vspace{1em}

{\small We use type-in-type everywhere for simplicity, i.e.\ $\U : \U$ and $\PSh_i : \PSh_i$.}

\end{frame}

\begin{frame}{Basic things we can do}
\begin{block}{}
\vspace{-0.6em}
{\small
  \[ \U : \U \hspace{1.5em} \Base : \U \hspace{1.5em} 1 : \Base \hspace{1.5em} \PSh : \Base \to \U \]
  \[ \Cat_i : \PSh_i := \text{\emph{type of cats in $\PSh_i$}} \hspace{1.5em} \In : \Cat_i \to \U \hspace{1.5em} \base : \In\,\mbbC \to \Base \]
}
\end{block}
\vspace{1em}

$\PSh_1$ is a universe supporting ETT. Semantically, $\PSh_1$ is a universe of sets.
\vspace{1em}
\pause

We can define some $\mbbC : \Cat_1$, where $\ms{Obj}(\mbbC) : \PSh_1$.
\vspace{1em}
\pause

Now, \alert{under the assumption} of $i : \In\,\mbbC$, we can form the universe $\PSh_{(\base\,i)}$, which is semantically the universe of
presheaves over $\mbbC$.
\vspace{1em}
\pause

At this point, we have no interesting interaction between $\PSh_1$ and $\PSh_i$.
\vspace{1em}

{\small
\emph{Syntactic sugar:} we'll omit ``$\base$'' in the following.
\vspace{3em}
}

\end{frame}

\begin{frame}{Example: embedding pure lambda calculus}
%% \begin{block}{}
%% \vspace{-1.1em}
%% {\footnotesize
%% \begin{alignat*}{5}
%%   & \Set   &&: \Set                 && \Cat_i &&: \PSh_i := \text{\emph{type of cats in $\PSh_i$}} \\
%%   & \Base  &&: \Set                 && \In    &&: \Cat_i \to \Set \\
%%   & \PSh   &&: \Base \to \Set \quad && \base  &&: \In\,C \to \Base
%% \end{alignat*}
%% }
%% \end{block}

\vspace{1em}
A \textbf{second-order model of pure LC} in $\PSh_i$ consists of:
\begin{alignat*}{3}
  & \Tm &&: \PSh_i \\
  & \lam &&: (\Tm \to \Tm) \to \Tm \\
  & \blank\!\$\!\blank &&: \Tm \to \Tm \to \Tm \\
  & \beta && : \lam\,f\,\$\,t = f\,t \\
  & \eta && : \lam\,(\lambda x.\,t\,\$\,x) = t
\end{alignat*}

We define $\SMod_i : \PSh_i$ as the above $\Sigma$-type.
\vspace{2em}
\end{frame}

\begin{frame}{Example: embedding pure lambda calculus}
%% \begin{block}{}
%% \vspace{-1.1em}
%% {\footnotesize
%% \begin{alignat*}{5}
%%   & \Set   &&: \Set                 && \Cat_i &&: \PSh_i := \text{\emph{type of cats in $\PSh_i$}} \\
%%   & \Base  &&: \Set                 && \In    &&: \Cat_i \to \Set \\
%%   & \PSh   &&: \Base \to \Set \quad && \base  &&: \In\,C \to \Base
%% \end{alignat*}
%% }
%% \end{block}
%% \vspace{1em}

A \textbf{first-order model of pure LC} consists of:
\begin{itemize}
\item A category of contexts and substitutions with $\Con : \PSh_i$, $\Sub : \Con \to \Con \to \PSh_i$ and terminal
      object $\emptycon$.
\item $\Tm : \Con \to \PSh_i$, plus a term substitution operation.
\item A context extension operation $\blank\ext : \Con \to \Con$ such that $\Sub\,\Gamma\,(\Delta\,\ext) \simeq \Sub\,\Gamma\,\Delta \times \Tm\,\Gamma$.
\item A natural isomorphism $\Tm\,(\Gamma\,\ext) \simeq \Tm\,\Gamma$ whose components are $\lambda$ and application.
\end{itemize}
\vspace{1em}
We define $\FMod_i : \PSh_i$ as the above $\Sigma$-type.

\vspace{1em}
$\FMod$ is mechanically derivable from $\SMod$.\footnote{Ambrus Kaposi \& Szumi Xie: \emph{Second-Order Generalised Algebraic Theories}.}

\end{frame}

\begin{frame}{Example: embedding pure lambda calculus}

\begin{block}{GLF Axiom 1}
  Assuming $M : \FMod_i$ and $j : \In\,M$, we have $\S_j : \SMod_j$.

  {\footnotesize (In ``$\In\,M$'' we implicitly convert $M$ to its underlying category.)}
\end{block}
\vspace{0.5em}

Now we have a 2LTT inside $\PSh_j$:
\begin{itemize}
\item ETT type formers in $\PSh_j$ comprise the outer level.
\item $S_j$ comprises the inner level.
\end{itemize}
\vspace{0.5em}
\pause
Y-combinator as example:
\begin{alignat*}{3}
  & \ms{YC} : \Tm_{\S_j} \\
  & \ms{YC} := \lam_{\S_j}(\lambda\,f.\,(\lam_{\S_j} (\lambda x.\, x\,\$_{\S_j}\,x))\,\$_{\S_j}\,
               (\lam_{\S_j} (\lambda f.\,\lam_{\S_j} (\lambda x.\, f \,\$_{\S_j}\, (x \,\$_{\S_j}\, x)))))
\end{alignat*}
\pause
With a reasonable amount of sugar:
\begin{alignat*}{3}
  & \ms{YC} : \Tm_{\S_j} \\
  & \ms{YC} := \lam\,f.\,(\lam\,x.\,x\,x)\,(\lam\,f.\,\lam\,x.\,f\,(x\,x))
\end{alignat*}

\end{frame}

\begin{frame}{}

\begin{itemize}
\item More generally, we have the previous axiom for every second-order generalized algebraic theory.
\item Hence: all 2LTTs are syntactic fragments of GLF.
\item (For each 2LTT, the semantics of GLF restricts to the standard presheaf semantics of the 2LTT.)
\end{itemize}


\end{frame}

\begin{frame}{Yoneda: conversion between internal \& external views}

\begin{block}{GLF Axiom: Yoneda embedding for pure LC}
Assuming $M : \FMod_i$ and writing $\simeq$ for definitional isomorphism, we have
\begin{alignat*}{4}
  & \Y && : \Con_M                 &&\to\,  &&((j : \In_M) \to \PSh_j) \\
  & \Y && : \Sub_M\,\Gamma\,\Delta &&\simeq &&((j : \In_M) \to \Y\,\Gamma\,j \to \Y\,\Delta\,j)\\
  & \Y && : \Tm_M\,\Gamma          &&\simeq &&((j : \In_M) \to \Y\,\Gamma\,j \to \Tm_{\S_j})
\end{alignat*}
such that $\Y$ preserves empty context and context extension:
\begin{alignat*}{4}
  & \Y\,\emptycon\,j &&\simeq \top \\
  & \Y\,(\Gamma\,\ext)\,j &&\simeq \Y\,\Gamma\,j \times \Tm_{\S_j}
\end{alignat*}
and $\Y$ preserves all other structure strictly.
\end{block}
\textbf{Notation}: we write $\Lambda$ for inverses of $\Y$.
\end{frame}

\begin{frame}{LC examples, sugar}

$\Y$ and $\Lambda$ allow ad-hoc switching between first-order and second-order notation.
Let's redefine some operations using second-order notation:
\begin{alignat*}{6}
  & \ms{id} : \Sub_M\,\Gamma\,\Gamma && \ms{comp} : \Sub_M\,\Delta\,\Theta \to \Sub_M\,\Gamma\,\Delta \to \Sub_M\,\Gamma\,\Theta\\
  & \ms{id} := \Lambda\,(\lambda\,j\,\gamma.\,\gamma)\quad\quad && \ms{comp}\,\sigma\,\delta := \Lambda\,(\lambda\,j\,\gamma.\,\Y\,\sigma\,(\Y\,\delta\,\gamma\,j)\,j
\end{alignat*}
\pause
With reasonable amount of sugar:
\begin{alignat*}{6}
  & \ms{id} := \Lambda\,\gamma.\,\gamma\quad\quad && \ms{comp}\,\sigma\,\delta := \Lambda\,\gamma.\,\Y\,\sigma\,(\Y\,\delta\,\gamma)
\end{alignat*}
\pause
Or even:
\begin{alignat*}{6}
  &\ms{comp}\,\sigma\,\delta := \Lambda\,\gamma.\,\sigma\,(\delta\,\gamma)
\end{alignat*}
\pause
Example for ``pattern matching'' notation:
\begin{alignat*}{3}
  & \ms{p} : \Sub_M\,(\Gamma\,\ext)\,\Gamma \\
  & \ms{p} := \Lambda\,(\gamma,\,\alpha).\,\gamma \quad\quad \text{\emph{Note: $\Y\,(\Gamma\,\ext) \simeq \Y\,\Gamma \times \Tm_{\S_j}$}}
\end{alignat*}

\end{frame}

\begin{frame}{Second-order named notation}
  \begin{itemize}
  \item When working with CwF-s, De Bruijn indices and substitutions
        can be hard to read.
  \item Handwaved ``named'' binders in CwFs have been used a couple of times.
  \item GLF provides a rigorous implementation of such notation.
  \end{itemize}
\end{frame}

\begin{frame}{Embedding dependent type theories}

\begin{columns}
\begin{column}{0.5\textwidth}
In a first order model, we have:
\begin{alignat*}{3}
  &\Con &&: \PSh_i \\
  &\Sub &&: \Con \to \Con \to \PSh_i \\
  &\Ty  &&: \Con \to \PSh_i \\
  &\Tm  &&: (\Gamma : \Con) \to \Ty\,\Gamma \to \PSh_i \\
  & ... &&
\end{alignat*}
\end{column}
\begin{column}{0.5\textwidth}
In a second order model, we have
\begin{alignat*}{3}
  &\Ty  &&: \PSh_i \\
  &\Tm  &&: \Ty \to \PSh_i \\
  & ... && \\
  & && \\
  & &&
\end{alignat*}
\end{column}
\end{columns}
\vspace{0.5em}
\pause
Yoneda embedding:
\begin{alignat*}{4}
  & \Y : \Con_M &&\to\,&&((j : \In\,M) \to \PSh_j) \\
  & \Y : \Sub_M\,\Gamma\,\Delta &&\simeq &&((j : \In\,M) \to \Y\,\Gamma\,j \to \Y\,\Delta\,j) \\
  & \Y : \Ty_M\,\Gamma &&\simeq &&((j : \In\,M) \to \Y\,\Gamma\,j \to \Ty_{\S_j})\\
  & \Y : \Tm_M\,\Gamma\,A &&\simeq &&((j : \In\,M) \to (\gamma : \Y\,\Gamma\,j) \to \Tm_{\S_j}\,(\Y\,A\,j\,\gamma))
\end{alignat*}

\end{frame}

\begin{frame}{Embedding dependent type theories}

\textbf{Sugar for contexts}:
\[(\Gamma \ext A \ext B) : \Con_M \quad \text{is equal to}\quad  \Gamma \ext (\Lambda\,\gamma. \Y A\,\gamma) \ext (\Lambda\,(\gamma,\,\alpha). \Y B\,(\gamma,\,\alpha)) \]
\pause
This suggests the notation:
\[(\gamma : \Gamma,\,\alpha : \Y A\,\gamma,\,\beta : \Y B\,(\gamma,\,\alpha)) : \Con_M\]
With implicit $\Y$:
\[(\gamma : \Gamma,\,\alpha : A\,\gamma,\,\beta : B\,(\gamma,\,\alpha)) : \Con_M\]
\pause
\textbf{Sugar for $\bs{\Tm_M$}}. We have
\[  \Tm_M\,(\Gamma\,\ext\,A\,\ext\,B)\,C \,=\, \Tm_M\,(\Gamma\,\ext\,A\,\ext\,B)\,(\Lambda\,(\gamma,\,\alpha,\,\beta).\,B\,(\gamma,\,\alpha,\,\beta)) \]
which suggests the notation
\[  \Tm_M\,(\gamma : \Gamma,\,\alpha : A\,\gamma,\,\beta : B\,(\gamma,\,\alpha))\,(B\,(\gamma,\,\alpha,\,\beta)) \]

\end{frame}

\begin{frame}{Embedding dependent type theories}

Example: a construction which looks awful in explicit CwF notation\footnote{Kaposi, Huber, Sattler: \emph{Gluing for Type Theory}, Section 5}
{\small
\begin{alignat*}{5}
  &\Con^{\circ}\,\Gamma && := \Ty\,(F\,\Gamma)\\
  &\Ty^{\circ}\,\Gamma^{\circ}\,A && := \Ty\,(F\,\Gamma\,\ext\,\Gamma^{\circ}\,\ext\,F\,A[\ms{p}])\\
  &\Tm^{\circ}\,\Gamma^{\circ}\,A^{\circ}\,t && := \Tm\,(F\,\Gamma\,\ext\,\Gamma^{\circ})\,(A^{\circ}[\id,\,F\,t[\ms{p}]))\\
  & \Gamma^{\circ}\,\ext^{\circ}\,A^{\circ} && := \Sigma(\Gamma^{\circ}[\ms{p}\circ F_{\ext.1}])(A^{\circ}[\ms{p} \circ F_{\ext.1} \circ \ms{p},\,\ms{q},\,\ms{q}[F_{\ext.1} \circ \ms{p}]])\\
  & ... &&
\end{alignat*}
\vspace{-0.5em}
{\normalsize but is reasonable in sugary GLF notation:}
\begin{alignat*}{5}
  &\Con^{\circ}\,\Gamma && := \Ty\,(\gamma : F\,\Gamma)\\
  &\Ty^{\circ}\,\Gamma^{\circ}\,A && := \Ty\,(\gamma : F\,\Gamma,\,\gamma^{\circ} : \Gamma^{\circ}\,\gamma,\,\alpha : F\,A\,\gamma)\\
  &\Tm^{\circ}\,\Gamma^{\circ}\,A^{\circ}\,t && := \Tm\,(\gamma : F\,\Gamma,\,\gamma^{\circ} : \Gamma^{\circ}\,\gamma)\,(A^{\circ}\,(\gamma,\,\gamma^{\circ},\,F\,t\,\gamma))\\
  & \Gamma^{\circ}\,\ext^{\circ}\,A^{\circ} &&:= \Lambda\,(F_{\ext.1}(\gamma,\,\alpha)).\, \Sigma(\gamma^{\circ} : \Gamma^{\circ}\,\gamma) \times A^{\circ}\,(\gamma,\,\gamma^{\circ},\,\alpha)
\end{alignat*}
}
\emph{It's a fair amount of sugar, but we can always rigorously desugar when it doubt!}

\end{frame}

\begin{frame}{Sketch of semantics}

Each $\PSh_i$ should be an universe of internal presheaves over an internal category.
\vspace{1em}

We should work with $\bCat$ somehow, but there are issues with that:
\begin{itemize}
\item There's no general $\Pi$.
\item $\Pi$-types of presheaves and universes of presheaves are not stable under
      reindexing by arbitrary functors.
\end{itemize}
\vspace{1em}

In GLF, the categorical part ($\Base$, $\In$) is purely for bookkeeping, we can't do synthetic
category theory. We can only do interesting things with presheaves.
\vspace{1em}

GLF contexts are \emph{trees of categories} where tree morphisms only have interesting
action on ``discrete'' parts of the tree.

\end{frame}

\begin{frame}{Sketch of semantics}

\textbf{Notation}:
\begin{itemize}
\item For a category $C$ and a split fibration $A$ over it, we write $C \ext A$ for the total category.
\item For a presheaf $A$, we write $\Disc\,A$ for the derived discrete fibration.
\end{itemize}
\vspace{1em}

\textbf{Definition}. A \emph{category telescope} is either the terminal category, or it is
(inductively) of the form $C\,\ext\,\Disc\,A\,\ext\,B$ where $C$ is a category telescope. We write
$C : \CatTel$ for a category telescope.
\vspace{1em}

\textbf{Definition}. A tree of categories is inductively defined as:
\begin{alignat*}{3}
  & \mbf{data}\,\Tree\,(B : \CatTel) : \Set\,\mbf{where}\\
  & \hspace{1em} \ms{node} : (\Gamma : \PSh\,B)\\
  & \hspace{2.2em} \to  (n : \mbb{N})\\
  & \hspace{2.2em} \to  (C : \Fin\,n \to \Fib\,(B \ext \Disc\,\Gamma))\\
  & \hspace{2.2em} \to ((i : \Fin\,n) \to \Tree\,(B \ext \Disc\,\Gamma \ext C\,i))\\
  & \hspace{2.2em} \to \Tree\,B
\end{alignat*}

\end{frame}

\begin{frame}{Sketch of semantics}

\begin{block}{}
\vspace{-1em}
{\small
\begin{alignat*}{3}
  & \ms{node} &&: (\Gamma : \PSh\,B)(n : \mbb{N})(C : \Fin\,n \to \Fib\,(B \ext \Disc\,\Gamma)) \to ((i : \Fin\,n) \to \Tree\,(B \ext \Disc\,\Gamma \ext C\,i)) \\
  & &&\to \Tree\,B
\end{alignat*}
}
\end{block}
\vspace{0.5em}

A GLF context is an element of $\Tree\,1$. Some examples for semantic contexts. We have $\mbb{N}_i : \PSh_i$. We use $\blank\ext\blank$ for ``context extension'' in presheaves as well.
\begin{alignat*}{3}
  & \emptycon &&:= \ms{node}\,1\,0\,[]\,[]\\
  & (\emptycon\,\ext\,\mbb{N}_1) &&:= \ms{node}\,(1\,\ext\,\mbb{N})\,0\,[]\,[]\\
  & (\emptycon\,\ext\,\mbb{N}_1\,\ext\,\In\,C) &&:= \ms{node}\,(1\,\ext\,\mbb{N})\,1\,[C]\,[\ms{node}\,1\,0\,[]\,[]]\\
  & (\emptycon\,\ext\,\mbb{N}_1\,\ext\,i : \In\,C\,\ext\,\mbb{N}_{(\base\,i)}) &&:= \ms{node}\,(1\,\ext\,\mbb{N})\,1\,[C]\,[\ms{node}\,(1 \ext \mbb{N})\,0\,[]\,[]]
\end{alignat*}
\vspace{-1em}
\pause

\begin{itemize}
\item A $\Base$ points to a node of the tree.
\item An $\In$ points to a subtree of a node.
\item Extending a context with $A : \PSh_i$ extends the presheaf in node $i$.
\item Extending a context with $j : \In\,C$ for $C : \Cat_j$ adds a new subtree at node $j$.
\end{itemize}


\end{frame}


\end{document}
